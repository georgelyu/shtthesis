\begin{abstract}[flattitle]
  流体仿真技术在诸多现实领域有着重要的指导意义,同时也是计算机图形学、计算流体力学等学科中重要的研究内容。然而在大规模的复杂场景下,使用现有方法对湍流进行高效、精准仿真,通常需要巨大的计算资源和时间开销。如何提升高雷诺数湍流仿真的效率与精度依然是极具挑战性的问题。同时由于不同领域的侧重点不同,不同的流体仿真方法开始分化,并逐渐只被应用于特定领域,这在一定程度上限制了流体仿真方法的灵活性与应用范围。针对这一系列问题,本文在格子玻尔兹曼方法的框架下,针对不同领域的应用提出了新的介观流体仿真方法,进一步提升格子玻尔兹曼方法的精度和稳定性,同时通过自动化的前处理,与图形处理单元 (Graphics Processing Unit, GPU) 优化算法,实现湍流的准确、高效仿真,以应用于视觉动画、工业产品设计、气动声分析等多个领域,弥合不同领域间流体仿真方法的差异。本文的主要技术创新点如下:
  \begin{enumerate}
      \item 本文提出面向视觉动画的通用流体仿真及流固耦合方法。针对于计算机图形学领域追求高效、稳定的流体仿真方法的特点,本文提出了基于速度修正和简单反弹边界方法的混合边界方法。该方法在精度和稳定性上相比现有的简单反弹边界方法都有所提升。并且可以在在湍流情况下同时处理亚网格尺度物体和正常尺度物体,展现出很强的通用性。
      \item 本文提出更加通用的高精度流体仿真方法。对精度要求更严格的领域,如工业产品设计,本文提出了新的单点插值反弹边界处理方法。该边界处理可以同时处理静态与动态物体,并利用单点处理提升了计算效率。同时,本文还提出了新的基于熵优化的累积量碰撞模型,该模型在高雷诺数流体仿真中依然保持稳定,并有着相比现有方法更高的精确度。通过与现实物理实验的对比,本文验证了该套流体仿真方法有着非常高的精准度,同时维持了很高的计算效率,使其可以应用在计算机图形学、计算流体力学、计算气动声学等不同领域,成为统一的流体仿真方法。
      \item 本文提出高效、自动化的流体仿真框架。为了使上述的流体仿真方法得到足够精确的物理量,流体仿真需要在极高的分辨率下进行,这势必要求使用非常高的分辨率构建计算网格。而目前这一个过程对于使用者来说非常繁琐冗长。本文在系统实现层面,阐述了自动化流体仿真的整体框架。其中包括了自动的计算网格构建方法、与自动的流体区域划分。该方法可以在有复杂几何的同时,完成高效的内、外流体区分及包含自动细分的多分辨率网格构建,从而支持高精度的流体仿真。
      \item 本文提出相关的算法优化以及高效GPU实现。本文还提出了针对上述方法的一些算法优化,如快速的几何计算算法,以降低额外的计算消耗。并且,针对上述每一部分,本文都阐述了相关的GPU优化算法,以优化整体的运行效率。借助现代的GPU硬件,利用本文的方法可以在几小时内完成高精度的流体仿真。
  \end{enumerate}
  
  除了提出新的流体计算方法与框架,本文还在计算机图形学、计算流体力学、计算气动声学等领域中,进行了大量的验证与实验,以证明我们新的流体仿真方法的准确性与应用潜力,并有利地推动格子波尔兹曼方法在这些领域的进一步应用与发展。
\end{abstract}

\begin{abstract*}[flattitle]

\end{abstract*}