\begin{abstract}[flattitle]
  流体仿真技术在诸多现实领域有着重要的指导意义,同时也是计算机图形学、计算流体力学等学科中重要的研究内容。然而在大规模的复杂场景下,使用现有方法进行湍流仿真,通常需要巨大的计算资源和时间开销。如何提升高雷诺数湍流仿真的效率与精度依然是极具挑战性的问题。同时由于不同领域的侧重点不同,不同的流体仿真方法开始分化,并逐渐只被应用于特定领域,这在一定程度上限制了流体仿真方法的灵活性与应用范围。针对这一系列问题,本文在格子玻尔兹曼方法的框架下,针对不同领域的应用提出了新的介观流体仿真方法,进一步提升格子玻尔兹曼方法的精度和稳定性,同时通过自动化的前处理,与图形处理单元 (Graphics Processing Unit, GPU) 优化算法,实现湍流的准确、高效仿真,以应用于视觉动画、工业产品设计、气动声分析等多个领域,弥合不同领域间流体仿真方法的差异。本文的主要技术创新点如下:
  \begin{enumerate}
      \item 针对于复杂几何的流固耦合问题,本文提出了基于速度修正和简单反弹边界方法的混合边界方法。该方法在精度和稳定性上相比简单反弹边界方法都有所提升。并且现有的流体仿真方法中并不能很好地求解薄板、细棒等亚网格尺度物体与流体的耦合,尤其是在湍流情况下。本文提出的新的混合边界方法可以同时适用于亚网格尺度物体和正常尺度物体,展现出很强的通用性。
      \item 本文提出更加通用的高精度流体仿真方法。针对精度要求更严格的领域,如工业产品设计,本文提出了新的单点插值反弹边界处理方法。该边界处理可以同时处理静态与动态物体,并利用单点处理提升了计算效率。同时,本文还提出了新的基于熵优化的累积量碰撞模型,该模型在极高雷诺数流体仿真中依然保持稳定,并有着相比现有方法更高的精确度。通过与现实物理实验的对比,本文验证了该套流体仿真方法有着非常高的精准度,同时维持了很高的计算效率,使其可以应用在计算机图形学、计算流体力学、计算气动声学等不同领域,成为统一的流体仿真方法。
      \item 本文提出高效、自动化的流体仿真框架。为了使上述的流体仿真方法得到足够精确的物理量,流体仿真需要在极高的分辨率下进行,这势必要求使用非常高的分辨率构建计算网格。而目前这一个过程难以自动化完成,对于使用者来说非常繁琐冗长。本文阐述了自动化流体仿真的整体框架,其中包括了自动的计算网格构建方法与流体区域划分。该方法可以在有复杂几何的同时,完成高效的内、外流体区分及包含自动细分的多分辨率网格构建,从而支持高精度的流体仿真。
      \item 针对上述方法,本文提出相关的算法优化以及高效GPU实现。如快速的几何计算与其在GPU上的并行方法,以降低额外的计算消耗,优化整体的运行效率。借助现代的GPU硬件,利用本文的方法可以在几小时内完成高精度的流体仿真。
  \end{enumerate}
  
  除了提出新的流体计算方法与框架,本文还在计算机图形学、计算流体力学、计算气动声学等领域中,进行了大量的验证与实验,以证明我们新的流体仿真方法的准确性与应用潜力,并有利地推动格子波尔兹曼方法在这些领域的进一步应用与发展。
\end{abstract}

\begin{abstract*}[flattitle]
  Fluid simulation technique provides important guiding significance for many real-world applications, and is also an important research topic in computer graphics, computational fluid dynamics, and other disciplines.
  However, in large-scale complex scenarios, existing turbulent simulation methods usually require huge computational resources and temporal cost.
  How to improve the efficiency and accuracy of high Reynolds number turbulent simulation is still a very challenging problem.
  Meanwhile, due to the different focuses of different disciplines, fluid simulation methods have begun to diverge and are gradually applied only to specific fields, which limit the flexibility and application scope of fluid simulation methods to a certain extent.
  To address the aforementioned problems, in this thesis we propose a novel mesoscopic fluid simulation method under the framework of the lattice Boltzmann method for different applications, which improves the accuracy and stability of existing lattice Boltzmann method. The method can accurately and efficiently simulate turbulent flow with automated pre-processing and GPU optimization. It can be applied to computer animation, industrial product design, aeroacoustic analysis and other fields, bridging the gap in fluid simulation methods among different fields. The main technical innovations of the thesis are as follows:
  \begin{enumerate}
    \item A hybrid boundary method based on velocity correction and simple bounce-back boundary method is proposed in this thesis for fluid-structure interaction with complex geometries. This method is improved in accuracy and stability compared with the simple bounce-back method. Moreover, the existing fluid simulation methods cannot well solve the coupling between fluid and sub-grid scale objects such as thin plates and thin rods, especially in turbulent flow. Out new hybrid boundary method can be applied to both sub-grid scale objects and normal scale objects, showing strong generalization.
    \item We propose a general-purpose fluid simulation framework with high accuracy. For fields with more stringent accuracy requirements, such as industrial product design, we propose a new single-node interpolated bounce-back boundary method. The boundary method can handle both static and dynamic objects, and the single-node algorithm can effectively improve computational efficiency. We also propose a new cumulatant collision model with entropy optimization, which is stable in extremely high Reynolds number fluid simulation, and more accurate than the existing methods. By comparing with real-world experiments, we verify that the fluid simulation method is highly accurate and efficient, so that it can be applied in different fields such as computer graphics, computational fluid dynamics, computational aeroacoustics, etc., to potentially become a unified fluid simulation method.
    \item We propose a framework for efficient and automated fluid simulation. In order to obtain sufficiently accurate physical quantities from fluid simulation, the resolution needs to be sufficiently high, which inevitably requires a very dense mesh for computation. This process is currently very cumbersome and tedious for users. In this thesis we introduce an overall framework for automated fluid simulation at the system implementation level. It includes an automated computational mesh construction and fluid region partitioning method. The method can efficiently partition fluid into different connected components to construct adaptive multi-resolution mesh for scenes with complex geometries, enabling highly accurate fluid simulation.
    \item We also present relevant algorithmic optimizations as well as efficient GPU implementations in the thesis. We propose some algorithmic optimizations for the aforementioned methods, such as efficient geometry processing algorithms, to reduce computational cost. Moreover, for each of the above sections, we describe relevant GPU optimization algorithms to optimize the overall efficiency. With the help of modern GPU hardware, highly accurate fluid simulation can be accomplished in a few hours using the proposed methods.
  \end{enumerate}

  In addition to proposing new fluid computation methods and framework, we also carry out a large number of validations and experiments for computer graphics, computational fluid dynamics, and computational aeroacoustics to demonstrate the accuracy and potential for real applications of our new fluid simulation method, and to promote the further application and development of the lattice Boltzmann method in these fields.
\end{abstract*}