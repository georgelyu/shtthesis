\chapter{引言}

% Sec 1.1
\section{研究背景与意义}
流体因为它的普遍性与复杂性,从17世纪经典流体力学形成至今,一直是人们乐此不疲的研究课题。从19世纪末开始,人们开始细致地研究流体粘性运动与高速运动,使得流体力学开始有了指导现实的能力,也就是与此同时,航空事业开始飞速的进步。流体力学的发展解决了人类为实现飞行梦想所面临的关键技术问题,另一方面航空领域和其他诸多领域的发展,也推动了流体力学自身的发展。随着计算机的出现,从20世纪下半叶开始,计算流体力学 (Computational Fluid Dynamics, CFD)学科建立,至今几十年其已发展成为一门成熟的、横跨流体力学、数值分析、偏微分方程数值理论与计算机科学等多个领域的交叉学科,并日益在生产实践中发挥越来越重要的作用。计算流体力学主要对流体进行直接的数值仿真,从而直观地展示出流体的特征,如整个计算域中的速度、温度、旋度、湍流程度与压力等,并且可以响应流体中物体造型的变化。这一点对于工业产品的外形设计有极大的指导意义。如今,计算流体力学被大量的应用于实际工程问题,如飞机外形设计~\cite{JOHNSON20051115}、换热器与空调的传热研究~\cite{ASLAMBHUTTA20121}、风力发电机的叶片设计~\cite{Shourangiz-Haghighi2020-mo}等。

同时,另一个需求流体仿真的热门领域是计算机图形学 (Computer Graphics, CG),即基于物理的流体动画技术。在近几十年中,随着娱乐工业的蓬勃发展,计算机图形学中的流体仿真也得到了长足的发展,并应用到了游戏、电影、虚拟现实与增强现实等等环境中。如Bridson 等~(\citeyear{doi:10.1126/science.1198769})所指出,通过流体动画技术制作出的动画,其视觉效果已达到了以假乱真的程度。然而对于基于物理的流体动画,大量的计算使得制作这样的动画需要庞大的算力并且必须离线计算,这对于艺术家迭代设计造成很大的困难。所以流体动画开始倾向于增强实时性而降低真实性的算法,即通过流体动画加速技术,创造出视觉上难以分辨的流体动画,但并不一定符合物理世界的实际规律。

在这一背景下,计算流体力学与计算机图形学中的流体仿真方法开始向不同方向发展。计算
流体力学研究更多关注计算精度和问题本质,而计算机动画方面更关注效率与视觉效果。而近来,格子玻尔兹曼方法 (Lattice Boltzmann Method, LBM) 的出现,使其成为一个可以替代传统的求解纳维-斯托克斯 (Navier-Stokes,N-S) 方程的流体仿真方法,并在CFD与CG领域均被广泛的应用。LBM通过求解格子玻尔兹曼方程 (Lattice Boltzmann Equation, LBE) 从而等效求解N-S方程,进行流体仿真。在这一过程中,LBM可以通过每个离散点局部的求解来迭代时间步长,即无需施加全局的约束。这一点使得LBM非常适合大规模并行计算,从而相比传统的基于N-S方程的仿真方法有着非常高的求解效率。而LBE可以通过Chapman-Enskog分析与N-S方程建立联系,从而证明LBE在弱可压流体中,与N-S方程在宏观上等同\cite{Y.H.Qian_1993}。这表明LBM可以在物理上满足精度的需求。这使得LBM有着满足工业与娱乐领域等不同需求的潜力。

除此之外,LBM还被越来越多的应用于计算气动声学 (Computational Aeroacoustics, CAA)。计算气动声学通常用于研究飞行器、汽车、列车等的噪音管理。由于这些情况下的仿真域很大,模拟声波的传播需要很小的时间步长,从而计算消耗很大。比较常用的数值方法是混合方法和直接数值模拟 (Direct Numerical Simulation, DNS) 方法~\cite{doi:10.2514/1.15993}。混合方法通常是对流体进行仿真后,再通过对线性欧拉 (Euler) 方程\cite{doi:10.1080/10618560410001673498,Bogey:2002:1610-1928:463,doi:10.2514/1.18933}或 Lighthill 非齐次波动方程\cite{doi:10.1098/rspa.1952.0060}来计算声场。然而因为混合方法中流场与声场是分别求解的,流体无法与其产生的声音进行交互,所以并不能适用于所有的情况。DNS虽然没有这个限制,但是由于计算声波传播所需求的精度很高,在空间和时间上都需要足够的精准,直接仿真的代价很高。而LBM由于其可直接求解弱可压流体,并且有着求解高效的特点,成为CAA领域的一个非常有前景的直接仿真方法。

虽然在多个领域都有很强的应用价值,但格子玻尔兹曼方法目前依然面临着计算网格构建复杂、边界条件精度及稳定性不足等问题。本文基于现有的格子玻尔兹曼方法,提升其求解精度和稳定性,构建了仿真框架,以自动化地完成高效、高精度地流体仿真,并展示其在不同领域中的应用。首先,我们提出了基于速度修正和简单反弹 (Simple Bounce-back, SBB) 边界方法的混合边界方法。该方法可以求解流体与轻薄物体的交互,如薄片、细棒等。这些物体的某些尺度通常比计算域的离散尺度还要小,在传统的流体仿真方法中极易引起流体泄露。我们提出的新的混合边界方法在解决流体泄露问题的同时,提升了求解的稳定性。同时,本文提出了针对该方法的快速几何计算算法,从而依然保持了求解的高效性,使该方法非常适合于CG领域的动画制作。本文还提出了新的插值反弹 (Interpolated Bounce-back, IBB) 边界方法,与基于熵优化的累积量碰撞模型。通过与物理实验的比较,我们验证了新的方法相比现有方法有着进一步的精度提升,并可以模拟出物理上的阻力危机现象。在实现与效率上,我们提出了基于距离场的自动计算网格构建方法,该方法可以自动识别出有效的流体区域,并依据与固体距离的远近自动细分网格,实现了快速、简便的网格生成。最后,我们提出了整个框架的GPU优化算法,在代码层面进一步提升运行效率。

本文的核心目标为缩小CG、CFD、CAA等不同领域中计算方法的区别,提出一个统一、高效、精准的流固耦合仿真框架,并完善相关实现方法,使流体仿真技术在现实中有着更强的指导意义。


% Sec 1.2
\section{研究现状}
本章节将从四个方面介绍相关研究工作。首先于章节1.2.1回顾CG领域中的流体仿真方法,之后于章节1.2.2回顾CFD领域中的流体仿真方法,最后于章节1.2.3介绍LBM方法中相关技术的发展。

% Sec 1.2.1
\subsection{计算机图形学中的流体仿真方法}
\paragraph{流体仿真方法}
相对于CFD领域中流体仿真对精度的追求,CG中的流体仿真方法更加注重效率与灵活性。例如早期Stam~(\citeyear{Stam-1999}) 提出的简单的无条件稳定的半拉格朗日方法。该方法开创了CG领域流体仿真的研究,但该方法因为有较强的数值耗散误差而无法进行湍流的仿真。之后的许多工作提出了新的数值求解方法从而改善这一情况,如BFECC方法~\cite{Kim-2005},无条件稳定的MacCormack方法~\cite{Selle-2008},对流-反射 (advection-reflection) 方法~\cite{Zehnder-2018}与BiMocq$^2$方法~\cite{Qu-2019}。但这些方法通常会带来额外的数值色散误差。
因为烟雾动画更加注重其表面涡量的模拟,涡方法应运而生。涡方法将N-S方程中重新构造为了基于涡量的形式~\cite{Park-2005, Selle-2005},并使用涡丝~\cite{Angelidis-2005, Weissmann-2010}或涡片~\cite{Pfaff-2012, Zhang-2014, Zhang-2015}进行仿真。
除去上述基于网格的方法,也有部分基于粒子的方法被应用于CG领域的流体仿真。如光滑粒子流体力学 (Smoothed Particle Hydrodynamics, SPH) 由于其易于实现的特性与良好的视觉效果,是CG领域中十分流行的液体仿真方法~\cite{Desbrun-1996,Muller-2003,Adams-2007,Becker-2007,Ihmsen-2014-1}。为了更好地保证流体的不可压缩性,基于密度~\cite{Solenthaler-2009, Bender-2015, Ihmsen-2014-2}或空间位置~\cite{Macklin-2013}的修正被提出来改善SPH方法的视觉效果。
此外,还有混合方法尝试结合网格与粒子方法的优点~\cite{Harlow-1962, Brackbill-1986, Foster-1996, Zhu-2005}。基于此,有工作~\cite{Jiang-2015, Fu-2017}进一步通过更精准的网格与粒子间的转换提升了精度。近来,Fei 等~(\citeyear{Fei-2021}) 使用了分离的积分方法降低了数值耗散,Qu 等~(\citeyear{Qu-2022}) 提出了新的网格与粒子间的转换方法以更好的保证流体体积的一致性。

\paragraph{流固耦合方法}
早期的工作在流固耦合仿真时,会将固体看作拉格朗日方法中的离散点,将每个点的速度在流体中对应为诺伊曼边界条件,并在固体表面受到相反的压力~\cite{Yngve:2000,Foster:2001,Carlson-2004,Takahashi-2002,Genevaux-2003}。
这些方法的耦合有显式格式也有半隐式格式,之后使用完全隐式格式来耦合流体固体速度的方法也被提出~\cite{Klingner-2006,Chentanez:2006:SCP,Batty-2007}。Teng 等~(\citeyear{Teng-2016}) 提出了完全使用欧拉方法处理流固耦合的方法。近来,单一框架的方法~\cite{takahashi-2020,fang-2020}被提出以支持流固之间的双向强耦合,但由于计算的复杂性,这类方法只能处理较小的尺度。
相比于常见固体,与薄片耦合的流体仿真难度更大,大多数需要特殊处理,如单边的基于求交的修正以防止流体泄露,或进行两次压力求解以计算应用于固体上的耦合作用力~\cite{Guendelman-2005}。
也有方法提出了更精准的动量转移与虚拟单元方法~\cite{Robinson-2008,Robinson:2009}。理论上,基于自适应网格的欧拉方法~\cite{Feldman:DF:2005,Feldman-2005,dai-2005,Klingner-2006,Elcott-2007}或拉格朗日方法~\cite{Misztal:2010,Clausen-2013}可以处理任意形状的固体,但这些方法需要频繁地调整网格结构,从而降低计算效率。
为了降低计算难度,部分基于切削网格的方法使用了类有限体积的形式,在单元内切割出贴合边界的区域以防止流体穿过物体~\cite{Roble-2005,Batty-2007,Ng-2009,gibou-2012,weber-2015,Edwards-2014,Liu:2015:MVF,Azevedo-2016}。
对于基于粒子的流体仿真方法,通常会在流体上施加惩罚力以实现边界条件~\cite{peer-2015,Ihmsen-2013}。Gao等~(\citeyear{Gao:2017}) 在流体隐式粒子 (Fluid-Implicit-Particle, FLIP) 方法中加入了贴合物体形状的约束, 但该方法在湍流中有稳定性问题。并且,绝大部分的基于粒子的方法无法保证流体压力的一致性~\cite{band-2018}。
此外,也有部分混合方法,使用网格来完成耦合的同时,利用粒子进行流体仿真~\cite{Zhang-2016,Fei-2018,hu-2018,Fei-2019}。

% Sec 1.2.2
\subsection{计算流体力学中的流体仿真方法}
\paragraph{有限差分方法}
有限差分方法是最简单而易于实现的偏微分方程求解方法。应用于N—S方程时,有限差分方法同样只需要较小的计算量即可求解~\cite{vreman2014comparison, kooij2018comparison}。但是只有极少数情况中,有限差分方法被应用于湍流仿真。在求解不可压缩N-S方程时,需要额外求解全局的泊松方程来满足不可压条件 (使速度场散度为0),这也是使用有限差分方法进行流体仿真的难点。对于拉普拉斯算子,通常需要使用迭代的求解方法 (如多重网格方法~\cite{golub2013matrix}) 来求解线性系统。其次,对于非线性平流项,有限差分方法往往采取半拉格朗日方法~\cite{smolarkiewicz1992class}。但这会带来很大的数值粘度误差。现代CFD中比较重要的差分格式为总变差减小 (Total Variation Diminishing, TVD) 格式~\cite{HARTEN1983357, osher1986very, YEE1985327}。在TVD格式中,已经通过对左右模板 (stencil) 的导数值的比较来进行自适应的选择适当的值,这一思想也被扩展到了之后的基本无震荡 (Essentially Non-oscillatory, ENO) 格式~\cite{HARTEN19973, SHU1988439, SHU198932}与加权基本无震荡 (Weighted Essentially Non-oscillatory, WENO) 格式~\cite{LIU1994200}。但是由于有限差分方法在多维问题上只能在各个维度上进行一维近似,所以其只能应用于结构化网格,从而对复杂几何体的处理能力有限。

\paragraph{有限体积方法}
相对于有限差分方法从微分形式的方程出发,有限体积方法通过从积分形式的方程出发出发构造算法,这使得有限体积方法有着天然的守恒性。也因为它直接计算积分项,使其可以直接应用于非结构网格上。这些特点使得有限体积方法能够运用于复杂几何体的计算。目前,有限体积方法被广泛应用于商业流体仿真软件中,如Ansys FLUENT,Ansys CFX等~\cite{JEONG201419}。 
对于不可压缩的流体,典型的显式有限体积法必须与压力投影阶段相结合,以得到无旋的速度场~\cite{pember1996higher, https://doi.org/10.1002/fld.310}。为了提高稳定性,对速度和压力进行离散时,通常会使用交错网格。虽然在结构化网格中实现这一点并不难,但非结构网格中这依然有一定的挑战~\cite{bermudez1998upwind, herbin2012staggered, gao2012unstructured}。此外,也有工作提出了更高阶的有限体积方法,如k-exact有限体积方法~\cite{barth1990higher}与WENO有限体积方法~\cite{HU199997}等。

\paragraph{有限元方法}
有限元方法将计算域离散为许多小区域 (即有限元),每个单元可以看作是整个解的一个分段的近似。通过有限元方法求解偏微分方程通常先要将方程改写为弱形式,这个求解的方法称作伽辽金 (Galerkin) 方法。而因为其采用连续函数空间,使得每个单元之间无法完全独立,在求解时需要求解一个全局的庞大线性方程组。所需的计算消耗使得该方法并未在流体仿真领域得到广泛应用。而随后,间断伽辽金 (Discontinuous Galerkin,DG) 方法被提出,最早其被用于求解中子运输方程~\cite{reed1973triangular},并随后通过与Runge-Kutta方法的结合,被推广并应用到双曲守恒律方程的求解中~\cite{cockburn1989tvb2, cockburn1989tvb3, cockburn1990runge, cockburn2001runge}。在此之后该方法也开始被应用于CFD领域~\cite{Zienkiewicz-2013, bassi1997high, lomtev1999discontinuous}。为了改进DG方法计算量大等缺点,无积分型 (quadrature-dree) DG~\cite{atkins1998quadrature}与节点型 (nodal) DG~\cite{hesthaven2007nodal}也被相应提出。

% Sec 1.2.3
\subsection{格子玻尔兹曼方法}
\label{sec:1_related_works_LBM}
\paragraph{碰撞模型}
在LBM中最简洁也最易于实现的碰撞模型为BGK (Bhatnagar-Gross-Krook) 碰撞模型~\cite{Bhatnagar-1954, Chen-1998},然而在高雷诺数流体中,BGK模型有着很大的截断误差,从而在数值上非常不稳定。多松弛时间 (multiple relaxation time, MRT) 碰撞模型~\cite{dHumieres-1992, Lallemand-2000, Coveney-2002}尝试通过在矩 (moment) 空间进行松弛,来提升整体的数值稳定性与准确性。最初MRT碰撞模型通过原始矩 (raw moments) 进行构造,而这一形式违反了伽利略不变性 (Galilean invariance)。而之后为了进一步提升精度与稳定性,基于中心矩 (central moments)~\cite{Geier-2006, Geier-2009}、埃尔米特矩 (Hermite moments)~\cite{Shan-2007, Chen-2014, Adhikari-2008}、中心埃尔米特矩 (central Hermite moments)~\cite{Mattila-2017, Shan-2019}和累积量 (cumulants)~\cite{Geier-2015, Geier-2017}的MRT碰撞模型被提出,并满足了伽利略不变性。
还有一类碰撞模型是对分布函数进行埃尔米特多项式 (Hermite polynomial) 展开后截断来去除非水动力学动理学的模态,之后进行碰撞。这类碰撞模型被称为正则化碰撞模型~\cite{Zhang-2006, Latt-2006}。后期有工作通过递归形式改进了正则化碰撞模型,被称为递归正则化碰撞模型~\cite{Malaspinas-2015, Coreixas-2017}。作者证明该方法有着更好的数值耗散和色散性质。Jacob 等~(\citeyear{Jacob-2018}) 之后又提出了混合递归正则化模型,对二阶应力张量的重构做了调整,提高了方法的稳定性。
另一类对碰撞模型的改进路线是动态调整松弛时间,主要的方法包括大涡模拟 (large-eddy simulations, LES) 亚网格模型~\cite{Eggels-1996, Sagaut-2010}和基于熵优化的碰撞模型~\cite{Karlin-1999, Ansumali-2003}。
对于LES亚网格模型,在LBM中最早被应用的是Smagorinsky模型~\cite{Hou-1994, Krafczyk-2003}。之后的工作通过包含动态的参数估计~\cite{Premnath-2009}、van Driest阻尼函数~\cite{Malaspinas-2014}与改进的应力项~\cite{Leveque-2007},来提升壁面流体描述的精度。
除了使用涡流粘度之外,还有基于近似反卷积的模型~\cite{Malaspinas-2011, Nathen-2018}和壁面自适应大涡模型 (Wall-adaptive Large Eddy, WALE)~\cite{Weickert-2010}被应用在LBM中。
同时,基于熵优化的碰撞模型通过最大化局部熵值,为高阶矩的松弛时间提供了一个有效的约束。然而,在碰撞过程中求解带约束的熵优化问题在数值上耗费很大。所以有不少方法尝试优化求解效率,如KBC模型~\cite{Karlin-2014}、基于伪熵的方法~\cite{Kramer-2019}和使用解析解进行优化的方法~\cite{Tang-2022}。
除了亚网格模型与熵优化之外,近来也有方法通过离线的回归方法来优化松弛时间~\cite{Li-2020}。

\paragraph{迁移算法}
在LBM中一个潜在的瓶颈是迁移过程中分布函数的互相依赖,从而降低了数据传输效率。一般来说,在迁移过程中会使用两套分布函数交替更新来避免冲突,但是这种方法非常的浪费存储。针对该问题,一系列的迁移算法试图通过修改访问顺序等,在使用单套分布函数的情况下避免数据冲突,来提升空间和时间效率。这些方法包括AA模式~\citet{Bailey-2009}、Shift-and-Swap方法~\cite{Mohrhard-2019}、Periodic-shift方法~\cite{Adrian-2023}、Esoteric Twist方法~\cite{Geier-2017-c}与Esoteric Pull and Push方法~\cite{Moritz-2022}等。

\paragraph{边界处理}
边界处理在LBM中是一个非常重要的研究方向~\cite{Marson-2022}。
在LBM中最常用的实现无滑移条件的边界方法是反弹边界条件~\cite{Ladd-1994, Bouzidi-2001, Ginzburg-2003, Chun-2007}和浸没边界法 (immersed boundary method, IBM)~\cite{Peskin-1972, Lu-2012, Kang-2011, Patel-2018, Seo-2011, Chen-2013}。
简单反弹边界因为十分易于实现,而成为最常见的边界处理方法。但是由于它实质上将边界形状看作了体素化的边界,所以一般情况下只有一阶精度。为了解决这一问题,一个常用的方案是根据网格点到边界的距离对分布函数进行插值。这一方法被称为插值反弹边界~\cite{Bouzidi-2001, Yu-2003}。
Ginzburg和d'Humières (~\citeyear{Ginzburg-2003}) 通过构造多次反射边界处理 (Multi-reflection Boundary Treatment),在弯曲边界处理上取得了更高阶的精度。
虽然这些方法可以有效地处理复杂几何,但是它们通常包含非局部的计算,即需要依赖于其相邻点的分布函数。这样首先造成了当网格点处于狭窄边界时,可能由于没有相邻点而无法处理的现象。其次,由于需要访问相邻点的数据,在并行计算中也会对效率产生影响。
所以有方法通过考虑边界自身的速度来部分避免这一问题~\cite{Chun-2007},并由此激发了单点插值反弹边界方法~\cite{Zhao-2017, Geier-2015, Tao-2018-b}与参数化的单点插值反弹边界方法~\cite{Zhao-2019, Chen-2021-b, Marson-2021}的提出。

\paragraph{网格细化}
为提升仿真的精度和效率,网格细化是非常重要的一步。网格细化可以令更多的计算资源集中在物体边界周围和尾流区域等重要部分,从而提高整体仿真的效率~\cite{Sandoval-2012}。在LBM中,网格细化可以主要分为两种:基于格点的 (Node-based) 和基于网格的 (Cell-based)。
基于格点的方法最早由Filippova和Hänel~(\citeyear{Filippova-1998})提出,并被Dupuis和Chopard~(\citeyear{Dupuis-2003})改进。这类方法将计算的数值储存在网格的格点上 (包括分布函数与宏观量等),并在格点上完成不同分辨率网格的转换(分布函数的重缩放),以满足质量与动量的守恒。在从粗网格向细网格转换时,对分布函数进行空间滤波可以在高雷诺数流体中提高稳定性~\cite{Lagrava-2012}。
与之相对,基于网格的方法~\cite{Rohde-2006, Chen-2006}将计算的数值储存在网格的中心点上,这类方法不需要对分布函数进行重缩放即可满足质量与动量的守恒~\cite{Schornbaum-2016, Hasert-2014, Latt-2021}。
为了使流体的过渡更加自然,紧凑插值 (Compact Interpolation) 法~\cite{Fard-2015}、方向分割 (Directional Splitting) 法~\cite{Gendre-2017}与直接耦合 (Direct Coupling) 法~\cite{Astoul-2021}被提出,这些方法可以使涡流更好的通过不同格子的边界区域。
上述的两类方法一般均约束粗网格以固定尺度细分为细网格,通常细网格的网格长度是粗网格的二分之一。而也有工作提出了连续尺度的网格细分方法~\cite{Li-2019},打破了这一限制。


% Sec 1.3
\section{研究内容和贡献}
虽然流体仿真已经被广泛地运用于多个现实领域,然而对于一些情境,现有的流体仿真方法还有一定的不足。比如在CG领域,随着算力的提升,精度也被逐渐地重视。并且像薄板、细棒等亚网格尺度的物体并不能被很好的与流体一起仿真。而在CFD领域,想要做到十分精准的流体仿真,需要大量的前期工作,包括面网格的处理、体网格的生成等等,其中很大部分需要人力的投入,十分影响整个仿真流程的效率。同时,现有的流体仿真方法对一些物理现象无法很好的重现,这意味着现有方法的精度也有待提升。这些问题都需求有更高精度、更高效、更统一且更自动化的流体仿真框架的出现。因此,本文的研究内容主要是围绕上述痛点和难点展开,主要贡献如下:

\paragraph{面向CG领域的多用途边界处理}
针对于CG领域追求高效、稳定的流体仿真方法的特点,本文提出了基于速度修正和简单反弹边界方法的混合边界方法。该方法在精度和稳定性上相比简单反弹边界方法都有所提升。并且现有CG领域中的流体仿真方法中并不能很好地求解薄板、细棒等亚网格尺度物体与流体的耦合,尤其是在湍流情况下。本文提出的新的混合边界方法可以同时适用于亚网格尺度物体和正常尺度物体,展现出很强的通用性。本文还提出了针对该方法的快速几何计算算法与相关GPU优化,降低额外的计算消耗,使该方法非常适合于CG领域的动画制作。

\paragraph{通用且准确的流体仿真及流固耦合方法}
本文提出了新的基于熵优化的累积量碰撞模型,该模型在高雷诺数流体仿真中依然保持稳定,并有着相比现有方法更高的精确度。同时,本文还提出了新的单点插值反弹边界处理方法。该边界处理可以同时处理静态与动态物体,并利用单点处理提升了计算效率。通过与现实物理实验的对比,我们验证了该套流体仿真方法有着非常高的精准度。同时,由于该方法依然维持了很高的计算效率,使其可以应用在CFD、CG、CAA等不同领域,成为统一的流体仿真框架。

\paragraph{构建高效、自动化的流体仿真框架}
本文在系统实现层面,阐述了自动化流体仿真的全流程。其中包括了自动的计算网格构建方法、与自动的流体区域划分。该方法可以在有复杂几何的同时,完成高效的内、外流体区分及包含自动细分的多分辨率网格构建,从而支持高精度的流体仿真。同时本文还阐述了相关的图形处理单元 (Graphics Processing Unit, GPU) 优化算法,以优化整体的运行效率。借助现代的GPU硬件,高精度的流体仿真可以在几小时内完成。


% Sec 1.4
\section{论文的组织结构}
本文的具体章节安排如下:

第一章介绍了本文工作的研究背景和意义,以及目前相关的研究现状,阐述了研究内容和贡献点。

第二章介绍了相关的背景知识,首先介绍了LBM的起源与相关的基础知识,之后介绍了如何从LBM推导至N-S方程,并最后介绍累积量碰撞模型。

第三章介绍了面向CG领域的多用途边界处理,与如何在有薄板、细棒等亚网格尺度物体时进行流固耦合,及相关的GPU优化。

第四章介绍了自动化的流体仿真框架,包括基于熵优化的累积量碰撞模型、单点插值反弹边界处理方法与自动的计算网格构建方法,同时介绍了相关的GPU优化。

第五章介绍了当前方法在计算流体声学中的验证案例,以显示我们提出的流体仿真方法在计算流体声学中的潜力。

第六章总结了现有的LBM流体仿真研究,并讨论流体仿真在CG、CFD、CAA等不同领域中更广泛的未来工作和未来研究方向。