\chapter{引言}

% Sec 1.1
\section{研究背景与意义}
流体因为它的普遍性与复杂性,从17世纪经典流体力学形成至今,一直是人们乐此不疲的研究课题。从19世纪末开始,人们开始细致地研究流体粘性运动与高速运动,使得流体力学开始有了指导现实的能力,也就是与此同时,航空事业开始飞速的进步。流体力学的发展解决了人类为实现飞行梦想所面临的关键技术问题,另一方面航空领域和其他诸多领域的发展,也推动了流体力学自身的发展。随着计算机的出现,从20世纪下半叶开始,计算流体力学(Computational Fluid Dynamics,CFD)学科建立,至今几十年其已发展成为一门成熟的、横跨流体力学、数值分析、偏微分方程数值理论与计算机科学等多个领域的交叉学科,并日益在生产实践中发挥越来越重要的作用。计算流体力学主要对流体进行直接的数值仿真,从而直观地展示出流体的特征,如整个计算域中的速度、温度、旋度、湍流程度与压力等,并且可以响应流体中物体造型的变化。这一点对于工业产品的外形设计有极大的指导意义。如今,计算流体力学被大量的应用于实际工程问题,如飞机外形设计、换热器与空调的传热研究、风力发电机的叶片设计等。

同时,另一个需求流体仿真的热门领域是计算机图形学(Computer Graphics,CG),即基于物理的流体动画技术。在近几十年中,随着娱乐工业的蓬勃发展,计算机图形学中的流体仿真也得到了长足的发展,并应用到了游戏、电影、虚拟现实与增强现实等等环境中。如Bridson等在《科学》上指出的那样,通过流体动画技术制作出的动画,其视觉效果已达到了以假乱真的程度。以至于流体特效技术获得2007年奥斯卡技术成就奖和科学技术奖。然而对于基于物理的流体动画,大量的计算使得制作这样的动画需要庞大的算力并且必须离线计算,这对于艺术家迭代设计造成很大的困难。所以流体动画开始倾向于增强实时性而降低真实性的算法,即通过流体动画加速技术,创造出视觉上难以分辨的流体动画,但并不一定符合物理世界的实际规律。

在这一背景下,计算流体力学与计算机图形学中的流体仿真方法开始向不同方向发展。计算
流体力学研究更多关注计算精度和问题本质,而计算机动画方面更关注效率与视觉效果。而近来,格子玻尔兹曼方法(Lattice Boltzmann Method,LBM)的出现,使其成为一个可以替代传统的求解纳维-斯托克斯(Navier-Stokes,N-S)方程的流体仿真方法,并在CFD与CG领域均被广泛的应用。LBM通过求解格子玻尔兹曼方程(Lattice Boltzmann Equation,LBE)从而等效求解N-S方程,进行流体仿真。在这一过程中,LBM可以通过每个离散点局部的求解来迭代时间步长,即无需施加全局的约束。这一点使得LBM非常适合大规模并行计算,从而有着非常高的求解效率。而LBM被证明可以以二阶精度求解N-S方程,从而有着精度的保证。这使得LBM有着满足工业与娱乐领域等不同需求的潜力。然而LBM目前依然面临着计算网格构建复杂、边界条件精度及稳定性不足等问题。

本文

% Sec 1.2
\section{研究现状}

% Sec 1.2.1
\subsection{计算机图形学中的流体仿真方法}

% Sec 1.2.2
\subsection{计算流体力学中的流体仿真方法}

% Sec 1.2.3
\subsection{格子玻尔兹曼方法及其边界处理}

% Sec 1.3
\section{研究内容和贡献}

% Sec 1.4
\section{论文的组织结构}