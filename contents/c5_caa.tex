\chapter{方法的扩展与后处理技术}
\label{chap:caa}

在本章中我们讨论将本文中介绍的流体仿真技术扩展应用至不同领域时,一些值得注意的要点,及所需的一些后处理技术。

\section{应用于视觉动画}
在视觉动画中,我们主要需求真实感的烟雾渲染。而这一过程所需的烟雾数据一般为烟雾的密度场 (表示各个位置烟雾的浓淡),而我们通过仿真得到的是流体的速度场,这自然需要模拟一个烟雾的传播 (advection) 过程。比较常见的传播方法是欧拉方法,即基于网格的。但是这有诸多的问题,其中最重要的一点是湍流仿真中有非常多的高频信号,而网格的空间采样频率很难捕捉到这些信号~\cite{bridson2015fluid},以至于传播得到的烟雾十分平滑,失去湍流的特征。而粒子则不受这一限制。对于一个被传播的物理量$q$,粒子的传播方程$Dq/Dt=0$表明粒子上所携带的值不会受到任何的滤波或信息丢失,十分适合我们的需求。所以我们在这里介绍一种通过粒子与网格的混合方法生成烟雾密度场的方法。该方法主要分为三个部分:粒子的生成、粒子的传播、粒子到网格的转换。

\paragraph{粒子的生成}
首先,我们先在场中确定一块区域,作为粒子的发射端。我们需要对该区域进行采样以真正决定粒子的初始位置。出于不同的需求,可能该区域不同位置的初始烟雾密度会不同,那么我们需要根据密度来进行采样。假如密度是均匀的,我们只需要进行均匀的随机采样。接下来我们要考虑一次性注入与连续注入两种情况。如果是一次性注入,我们只需要确定场中存在的粒子数量$N$,并一次性采样完毕,进行传播得到烟雾的动态。而连续注入时一个需要被确定的量是发射速率 (emission rate) $d$$N/dt$。即每一个时间步长,我们都应该发射$n=\Delta t \text{ }$$d$$N/dt$个粒子。我们要注意的是,连续注入时需要避免每个时间步长粒子的初始化位置相同的情况。那样会形成多条脉线 (streakline),而不是形成一个烟雾一样的连续体。

此外,还有一个问题是,在连续注入时,如果我们在仿真每一帧的开始注入固定量的粒子,令其随速度场传播,当速度足够快时,我们会看到一团团离散的烟雾,这非常影响视觉效果。为了解决这一问题,我们可以给每个粒子一个随机的生成时间,这个随机时间来源于在一个时间步长$dt$内的均匀分布。之后每个粒子只会沿着速度场传播剩余的时间,以达到真正的位置。

% \begin{figure}[htb]
%   \centering
%     \includegraphics[width=0.7\columnwidth]{figures/smoke_particles.png}
%   \bicaption[粒子传播时的不连续问题]{粒子传播时的不连续问题。在粒子相隔固定时间步长注入,且速度场中速度较快时,会出现不连续的烟雾团 (上图),而将粒子的生成时间随机化,可以避免这一问题 (下图)。}{Discontinuous proplem when advecting density using particles. If particles are emitted with fixed time interval and the velocities are fast, there will be discrete puffs moving through space (upper), while randomizing the emission time of particles can prevent the problem (bottom).}
%   \label{img:smoke_particles}
% \end{figure}

\paragraph{粒子的传播}
粒子的传播是影响精度的核心操作。在这一过程中我们需要求解一个微分方程来确定粒子的位置:
\begin{equation}
    \frac{d \mathbf{x}}{d t}=\mathbf{u}(\mathbf{x}),
    \label{eq:momentum_ode}
\end{equation}
其中$\mathbf{x}$为粒子的位置,$\mathbf{u}(\mathbf{x})$为速度场中$\mathbf{x}$点的速度。如果我们将离散的时间序列中第$n$个时刻记为$t^n$,此时的粒子位置记为$\mathbf{x}^{n}=\mathbf{x}(t^{n})$,而时间步长则为$\Delta t=t^{n+1}-t^{n}$,公式~\ref{eq:momentum_ode} 所表达的方程的初始条件为$\mathbf{x}(0)=\mathbf{x}^{0}$。则待求的解可以通过如下积分求得:
\begin{equation}
    \mathbf{x}(t)=\int_{0}^{t} \, \mathbf{u}(\mathbf{x}(\tau)) \, \mathrm{d}\tau.
\end{equation}

求解该方程最简单也是代价最小的时间积分方法即为前向欧拉 (forward Euler) 方法:
$\mathbf{x}^{n+1}=\mathbf{x}^{n}+\Delta t \, \mathbf{u}(\mathbf{x}^{n})$,
然而该方法只有一阶的精度。在时间步长不够小时,这一方法会带来很大的误差。而这一误差会随时间累积,最终使数值解与真实解偏差巨大。更精准的时间积分方法,如龙格-库塔 (Runge-Kutta,RK) 方法,可以得到更高阶精度的解。如二阶精度的龙格-库塔方法 (RK2) 可以写作
\begin{align}
    \begin{split}
    \mathbf{x}^{n+1 / 2} & =\mathbf{x}^n+\frac{1}{2} \Delta t \, \mathbf{u}(\mathbf{x}^n), \\
    \mathbf{x}^{n+1} & =\mathbf{x}^n+\Delta t \, \mathbf{u}(\mathbf{x}^{n+1 / 2}) .
\end{split}
\label{eq:2nd_rk}
\end{align}
而三阶精度的龙格-库塔方法 (RK3) 可以写作
\begin{align}
    \begin{split}
k_1 & =\mathbf{u}(\mathbf{x}^n), \\
k_2 & =\mathbf{u}(\mathbf{x}^n+\frac{1}{2} \Delta t \, k_1), \\
k_3 & =\mathbf{u}(\mathbf{x}^n+\frac{3}{4} \Delta t \, k_2), \\
\mathbf{x}^{n+1} & =\mathbf{x}^n+\frac{2}{9} \Delta t \, k_1+\frac{3}{9} \Delta t \, k_2+\frac{4}{9} \Delta t \, k_3 .
    \end{split}
    \label{eq:3rd_rk}
\end{align}
我们注意在公式~\ref{eq:2nd_rk} 与~\ref{eq:3rd_rk} 中,我们都忽略了速度场随时间的变化,这也导致这样的求解在时间上的精度仅为一阶的,但是我们在空间上可以获得更高的精度。

当然我们还要考虑一些特殊的情况,如在物体边界处的流体速度是非常低的。此时粒子很可能会以非常低的速度运动,或者在极端情况下粘滞在物体上。此外,即使我们使用了RK3方法,也依然可能受到时间步长的影响获得不精确的解,此时烟雾有可能穿过物体进入物体内部,造成视觉上的失真。为了处理这些问题,我们可以对粒子施加一个额外的惩罚力,惩罚力的大小与粒子距离物体边界的远近有关系,方向为远离物体的方向。可以抽象地认为,在粒子距离物体边界足够近的时候,物体会对粒子施加一个排斥力,拒绝粒子过度接近物体边界,甚至穿过物体边界。虽然这样做一定程度影响了粒子传播的精度,但这可以视为一个经验性的防止出现视觉失真现象的方法。

\paragraph{粒子到网格的转换}
我们使用粒子生成烟雾的目的,是为了制作视觉动画 (渲染)。虽然粒子也可以被直接渲染,但是庞大的粒子数量对高渲染效率造成很大挑战。而成熟的渲染管线更多采用基于网格的数据,许多全局光照以及动态效果也可以在基于网格的数据上做得更好。所以我们这里要讨论如何将粒子采样点转换回网格数据。

我们假设每一个粒子对烟雾密度的贡献为$s$,在最简单的情况下我们可以假设$s$对于所有粒子都是同一个值。然而基于不同的视觉效果,我们可以假设例如$s$是随时间递减的,而当$s$低于某一阈值后被视为消失 (即被删除)。假设对于粒子p,它的位置为$\mathbf{x}_p$,对烟雾密度的贡献为$s_p$。而对于网格的格点,每个点的位置为$\mathbf{x}_{i,j,k}$,其中$i$、$j$、$k$分别表达该点在网格中的坐标索引。这里$\mathbf{x}_{i,j,k}$与$\mathbf{x}_p$处于同一空间。则每个格点上的烟雾密度可以记为
\begin{equation}
    s_{i,j,k}=\sum_p s_p \, K(\mathbf{x}_p-\mathbf{x}_{i,j,k}).
    \label{eq:particle_sijk}
\end{equation}
在公式~\ref{eq:particle_sijk} 中$K$是一个核函数 (kernel function),这里该函数的具体形式没有严格的要求,举例来说我们可以使用二次B样条 (quadratic B-spline) 来进行构造:
\begin{align}
\label{eq:particle_kernel}
& k(x, y, z)= h\left(\frac{x}{\Delta x}\right) h\left(\frac{y}{\Delta x}\right) h\left(\frac{z}{\Delta x}\right), \\
& \quad h(r)=\left\{\begin{array}{cl}
\frac{1}{2}\left(r+\frac{3}{2}\right)^{2} & ,-\frac{3}{2} \leq r<-\frac{1}{2}, \\
\frac{3}{4}-r^{2} & ,-\frac{1}{2} \leq r<\frac{1}{2}, \\
\frac{1}{2}\left(\frac{3}{2}-r\right)^{2} & ,\frac{1}{2} \leq r<\frac{3}{2}, \\
0 & , \text {其他情况,}
\end{array}\right.
\end{align}
其中$\Delta x$是网格的大小。公式~\ref{eq:particle_kernel} 中对网格大小做了自适应处理,使得每个粒子对周边的网格格点的贡献是适当的,而不会使烟雾过度模糊或颗粒化。

\section{应用于物理量计算}
\label{sec:pp_surface_pressure}
在使用流体仿真进行物理量 (如升力系数、阻力系数、物体表面压力系数等) 计算时,为了尽可能地提升仿真的物理精度,在仿真时应注意以下几点:
\begin{itemize}
    \item 仿真域:由于目前的仿真域边界条件无法完美地模拟开放边界条件 (open boundary condition),即所有的仿真域边界都会对域内的流体产生影响,所以为了进一步减轻这一影响,通常仿真域在各个维度的大小应设置为物体的6-12倍;
    \item 网格分辨率:为了保证精度,一般网格大小$\Delta x$的尺度在物体特征长度的1000分之1左右。显然在这种分辨率下,单分辨率网格是不现实的,所以需要采用多分辨率网格;
    \item 网格构建:网格的构建应保证关键的结构和区域有足够的分辨率,如靠近物体边界的区域和物体的尾流区域。其中一些特殊结构需要额外加密,如细密格栅结构或曲率较大的结构。
  \end{itemize}

  \begin{figure}[!htb]
    \centering
        \includegraphics[width=0.9\columnwidth]{figures/pressure_projection.png}
    \bicaption[物体表面物理量投影的示意图]{物体表面物理量投影的示意图。图中蓝色网格为流体网格,红色为切削网格,绿色为固体网格,加点的区域为固体覆盖的区域。对于物体表面上的$s_a$与$s_b$点,如果直接从流体网格点插值物理量,则$x_a$与$x_b$点分别有着最大的影响,因为它们分别是沿法向投影后距离$s_a$与$s_b$点最近的流体点。然而因为$x_a$与$x_b$点到物体边界的距离差别较大,这两个点的物理量也会有很大区别,造成物体表面物理量的不连续。而沿着$s_a$与$s_b$点法向向外走一定距离后,分别从$p_a$与$p_b$点插值物理量,可以减轻这一问题。}{Schematics of physical quantities projection to surface. The blue, red and green cells in the firgue represent fluid, cut and solid cells, respectively. The dotted area is covered by solid object. For point $s_a$ and $s_b$ on object surface, if the physical quantities are directly interpolated from the nodes, node $x_a$ and $x_b$ would have the biggest impact, because they are the closest nodes to $s_a$ and $s_b$ after being projected to the surface along the normals, respectively. However, since node $x_a$ and $x_b$ have different distance to the object surface, resulting in largely different physical quantities which is discontinuous on surface. While interpolating from $p_a$ and $p_b$, which are points away from surface along their normals with a fixed distance can lift the problem.}
    \label{img:pressure_projection}
  \end{figure}

同时,因为我们需要的仿真输出是不同的物理量,这里使用的后处理方式也和上述不同。除了将数值直接输出外,为了使结果看起来更加直观,很多时候会将物理量进行颜色映射后以图片或动画的形式展现。将仿真域中的物理量以截面的形式提取出来,之后通过颜色映射进行后处理,是一个比较直接的过程。而对于物体表面的物理量 (如物体表面的压力),这并不是一个直接的过程。原因是在LBM中我们使用了切削网格来处理边界,这使得在有边界存在时,切削网格部分是不完整的 (被固体覆盖的网格点是不存在的,见图~\ref{img:pressure_projection} 中红色网格),这造成我们无法使用常规的插值方法来获取表面上点的物理量。

一个简单的想法是忽略不存在的固体点,而只对物体表面点周围以某个核函数对存在的流体点进行插值。然而这也会带来问题,一个最重要的问题是在切削网格内,物体表面距其法向方向最近的切削网格点的距离也不相同。以图~\ref{img:pressure_projection} 中的情况为例,对于物体表面上的$s_a$与$s_b$点,如果直接从流体网格点插值物理量,则$x_a$与$x_b$点分别有着最大的影响,因为它们分别是沿法向投影后距离$s_a$与$s_b$点最近的流体点。然而即使在同一网格内,在高雷诺数环境下,边界层依然比网格大小要薄很多,这造成这两个点的物理量也会有很大区别。所以在投影到物体表面后会造成物理量的不连续。

为了减轻这一问题,我们不以物体表面的点为中心进行插值,而是沿着物理表面的采样点的法向向外走一定距离后,再进行插值。依然以图~\ref{img:pressure_projection} 中为例,为了获取$s_a$与$s_b$点的物理量,分别从$s_a$与$s_b$点,沿表面的法向走一定距离后,从$p_a$与$p_b$点插值物理量。这样可以避免造成因为切削网格点本身的不连续带来的投影不连续问题。

\section{应用于气动声学分析}
\begin{figure}[!htb]
  \centering
      \includegraphics[width=0.85\columnwidth]{figures/acoustic_problems.png}
  \bicaption[不同可视化范围下对流涡经过多分辨率网格边界的对比]{不同可视化范围下对流涡经过多分辨率网格边界的对比。其中虚线表示瞬时速度的等高线,左图与右图分别使用了基于格点的与基于网格的多分辨率处理方法。在观察气动现象时,我们看到无论采用何种方法,该涡从定性的角度看均可以顺利地通过多分辨率网格边界 (上图)。但在观察更小尺度的声波抖动时,我们可以看到在网格边界产生了不应有的声波 (下图)。图片来自~\citep{ASTOUL2020109645}。}{Comparison of convected vortex crossins a multi-resolution grid interface with a colormap of different scales. Dashed lines are isocontours of transversal velocity. Left and right figures adopt cell-vertex and cell-centered algorithms, respectively. From an aerodynamic point of view, the convected vortex is qualitatively very well transmitted from one mesh to the other one whatever the algorithm used. However, when focused on the acoustic scales, there are obvious spurious acoustic fluctuations. Images from \citep{ASTOUL2020109645}.}
  \label{img:acoustic_problems}
\end{figure}

使用LBM进行气动声学分析是一个较为特殊的问题。虽然在第~\ref{sec:background} 节中我们已经讨论了,使用LBM进行气动声学仿真是LBM的一大优势,因为LBM可以非常高效地进行可压缩非稳态湍流求解。并且因为LBM本身求出的是弱可压过程,求解气动声与其它物理量并无区别。然而声音是由于非常微弱的气压波动导致,这种波动对精度是十分敏感的。虽然已有一些工作使用LBM进行了气动声分析,并且可以在一定程度上确定声源~\cite{doi:10.2514/6.2013-2256, doi:10.2514/1.J052365, doi:10.2514/6.2015-2993},但同时也会产生一些错误的声源。这些问题的根本原因是,在分析气动现象时,我们观察物理量的尺度是较大的,而在分析气流导致的声音现象时,观察的尺度会小很多,此时许多之前注意不到的误差均会产生很大的影响 (见图~\ref{img:acoustic_problems})。\citet{ASTOUL2020109645} 主要讨论了两个误差的来源,分别为多分辨率网格的边界,与碰撞模型本身的精度。然而由于声波对误差的敏感性,包括物体边界与计算域边界也同样是误差的来源。我们可以认为在目前阶段看,彻底消除这样的误差是十分困难的。本文在第~\ref{sec:acoustic_results} 节中展示了通过本文所介绍的方法进行气动声分析的一些初步成果,以验证本文中的方法应用于计算气动声学的能力。而基于此,在提升气动声学分析的精度上,还可以有更多未来的工作。

在后处理上,除了气动特征,通常我们还会记录某些采样点压力波动的时间序列,以定量地进行频谱分析。这一过程中,因为采样点也会位于物体表面,所以可以使用与上一节中所述的在物体表面取物理量时同样的处理。在获取到压力的时间序列后,通常使用加窗平均周期图法 (Welch法~\citep{1161901}) 进行频谱分析。具体来说该方法将时间信号序列分割成了多个块 (blocks),对每个块使用周期图法 (periodogram) 进行谱密度估计后再进行平均。

具体来说,一个时间信号序列先被分割为$K$个块,且这些块之间是有部分重叠的。
\begin{equation}
  x_i(n) = w(n) \, x(n+i D), \quad n=0,1, \ldots, M-1, \, i=0,1, \ldots, K-1,
\end{equation}
其中$x_i$为第$i$块的序列,$x$为原序列,$w$是窗函数 (window function),$D$是窗的跳长 (hop size)。我们注意到当$D=M$时,这些所有的块之间是没有重叠的。当$D=M/2$时,相邻的块之间会有50\%的重叠。一个常用的窗函数是汉宁窗 (Hann window):
\begin{equation}
  w(n)=\sin^2 \left(\frac{\pi n}{M-1}\right), \quad 0 \leq n \leq M-1 .
\end{equation}
此时第$i$块的周期图为
\begin{equation}
  P_{x_i}\left(k\right)=\frac{1}{MU}\left|\mathrm{FFT}\left(x_i\right)\right|^2 = \frac{1}{MU}\left|\sum_{n=0}^{M-1} x_i(n) e^{-j 2 \pi n k / M}\right|^2,
\end{equation}
其中$j$表示虚数单位,$U$是一个归一化系数,它的目的是抵消窗函数的平方带来的影响,具体的公式为
\begin{equation}
  U=\frac{1}{M}\sum^{M-1}_{n=0} w^2(n).
\end{equation}
则通过加窗平均周期图法获得的功率谱密度 (Power Spectral Density, PSD) 为
\begin{equation}
  \hat{S}_x^W\left(w_k\right) = \frac{1}{K} \sum_{i=0}^{K-1} P_{x_i}\left(k\right).
\end{equation}