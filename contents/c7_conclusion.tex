\chapter{总结与展望}
\label{chap:conclusion}

\section{总结}
本文在格子玻尔兹曼方法的框架下,提出了新的碰撞模型、边界处理及相关的前处理方法与GPU加速,进一步提升格子玻尔兹曼方法的精度、稳定性与易用性,实现湍流的准确、高效仿真。本文所提出的方法可以被应用于包括计算机图形学、计算流体力学及计算气动声学等不同领域,弥合不同领域仿真方法的差异,并推动LBM在这些领域中的应用与发展。

\section{展望}
我们考虑基于目前的工作,可以在以下三个方面进一步提升方法的仿真能力,并将其影响扩展到更多的研究和行业。

\paragraph{进一步的精度及性能提升}
虽然本文展示了一系列的验证案例,如物理上的阻力危机现象,但我们可以看到目前的方法与实际实验结果依然存在一定的误差。为了改善这一情况,我们认为未来可以从以下方面,在不大幅增加计算资源的前提下,对精度及性能进一步提升。首先,目前的多分辨率网格构建没有考虑到局部的单独网格细化,对尺寸小但对流场有重要影响的结构进行局部的细化可以进一步优化计算资源的分配以提升精度。此外,目前的多分辨率网格数据过渡仍会在网格边界产生不正确的流场,这对物理量的计算,尤其是气动声学这种对扰动极其敏感的应用,有着一定的影响。提升多分辨率网格过渡的连续性,对精度的提升是十分重要的。最后,针对于性能,我们认为LBM仍在实现层面有进一步的效率提升空间,具体地讲可以通过减少仿真时需求数据的吞吐量,来减少读写存储的开销,从而加速计算。在这一方向上使用更好的迁移算法~\citep{Moritz-2022} 与浮点数表示方法~\citep{PhysRevE.106.015308} 均是值得探索的方向。

\paragraph{与可变形物体耦合} 
我们目前的方法主要专注于与刚性物体的耦合,而许多场景下必须考虑物体的形变对仿真的影响。如布料与弹性物体在流体中的动态,血流仿真中血管的变形等。现有的方法中大多考虑使用浸没边界法结合拉格朗日点对边界进行近似~\citep{doi:10.1142/S0219876221500705, WU2017103},这一想法与我们在第~\ref{chap:siga21} 章中的方法是类似的。但是如何在动态物体上快速并均匀地采样以提升仿真效率,及如何将更高精度的方法如插值反弹边界应用于可变形物体,仍需要更深入的研究。

\paragraph{扩展到不同流体}
本文主要展示了气体 (烟雾) 与物体耦合时的仿真,但是我们也注意到。许多仿真场景还需求其它不同种类的流体,如研究汽车涉水的车身污染问题,或水下机器人在虚拟环境中的训练时,就涉及到液体的仿真。LBM在解两相流时,也同样有着高效、准确的优势,并且已经在视觉动画领域展现出一定的优势~\cite{Wei:2022, Wei:2023}。我们认为可以基于这些工作进一步将现有方法扩展至多相流体,并验证其在应用于工业制造领域的精度及性能表现,以进一步提升现有流体仿真方法的适用范围。