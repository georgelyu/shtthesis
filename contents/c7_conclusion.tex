\chapter{总结与展望}
\label{chap:conclusion}
本文在格子玻尔兹曼方法的框架下,提出了新的介观流体仿真方法及相关的前处理方法与GPU加速,进一步提升格子玻尔兹曼方法的精度、稳定性与易用性,实现湍流的准确、高效仿真。本文首先提出基于速度修正和简单反弹边界方法的混合边界方法,该方法可以有效地求解亚网格物体与正常尺度物体,展示出很强的求解复杂几何的能力。
  \begin{enumerate}
      \item 针对于复杂几何的流固耦合问题,本文提出了基于速度修正和简单反弹边界方法的混合边界方法。该方法在精度和稳定性上相比简单反弹边界方法都有所提升。并且现有的流体仿真方法中并不能很好地求解薄板、细棒等亚网格尺度物体与流体的耦合,尤其是在湍流情况下。本文提出的新的混合边界方法可以同时适用于亚网格尺度物体和正常尺度物体,展现出很强的通用性。
      \item 本文提出更加通用的高精度流体仿真方法。对精度要求更严格的领域,如工业产品设计,本文提出了新的单点插值反弹边界处理方法。该边界处理可以同时处理静态与动态物体,并利用单点处理提升了计算效率。同时,本文还提出了新的基于熵优化的累积量碰撞模型,该模型在极高雷诺数流体仿真中依然保持稳定,并有着相比现有方法更高的精确度。通过与现实物理实验的对比,本文验证了该套流体仿真方法有着非常高的精准度,同时维持了很高的计算效率,使其可以应用在计算机图形学、计算流体力学、计算气动声学等不同领域,成为统一的流体仿真方法。
      \item 本文提出高效、自动化的流体仿真框架。为了使上述的流体仿真方法得到足够精确的物理量,流体仿真需要在极高的分辨率下进行,这势必要求使用非常高的分辨率构建计算网格。而目前这一个过程对于使用者来说非常繁琐冗长。本文在系统实现层面,阐述了自动化流体仿真的整体框架。其中包括了自动的计算网格构建方法、与自动的流体区域划分。该方法可以在有复杂几何的同时,完成高效的内、外流体区分及包含自动细分的多分辨率网格构建,从而支持高精度的流体仿真。
      \item 本文提出相关的算法优化以及高效GPU实现。本文还提出了针对上述方法的一些算法优化,如快速的几何计算算法,以降低额外的计算消耗。并且,针对上述每一部分,本文都阐述了相关的GPU优化算法,以优化整体的运行效率。借助现代的GPU硬件,利用本文的方法可以在几小时内完成高精度的流体仿真。
  \end{enumerate}