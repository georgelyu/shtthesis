% shtthesis, an unofficial LaTeX thesis template for ShanghaiTech University.
% Copyright (C) 2022 Li Rundong <rundong.001@gmail.com>
%
% This program is free software: you can redistribute it and/or modify
% it under the terms of the GNU General Public License as published by
% the Free Software Foundation, either version 3 of the License, or
% (at your option) any later version.
%
% This program is distributed in the hope that it will be useful,
% but WITHOUT ANY WARRANTY; without even the implied warranty of
% MERCHANTABILITY or FITNESS FOR A PARTICULAR PURPOSE.  See the
% GNU General Public License for more details.
%
% You should have received a copy of the GNU General Public License
% along with this program.  If not, see <https://www.gnu.org/licenses/>

% graduate setup
\documentclass[doctor]{shtthesis}
\shtsetup{
  degree-name = {工学博士},
  degree-name* = {Doctor~of~Philosophy},
  %secret-level = {白给},
  title = {用于复杂流固耦合的高性能介观流体仿真方法},
  title* = {High performance mesoscopic method for fluid\\simulation with complex fluid-solid coupling},
  keywords = {高性能流体仿真,格子玻尔兹曼方法,流固耦合,碰撞模型,边界处理,GPU优化,并行计算,自动化前处理,流体数据渲染与可视化},
  keywords* = {High-performance Fluid Simulation,Lattice Boltzmann Method,Fluid-solid Coupling, Collision Model, Boundary Treament, GPU Optimization, Parallel Computing, Automatic Preprocessing, Fluid Data Rendering and Visualization},
  author = {吕超阳},
  author* = {Lyu~Chaoyang},
  institution = {上海科技大学信息科学与技术学院},
  institution* = {School~of~Information~Science~and~Technology\\%
                  ShanghaiTech~University},
  supervisor = {刘晓培~副教授},
  supervisor* = {Professor~Liu~Xiaopei},
  %supervisor-institution = {上海科技大学信息科学与技术学院},
  discipline-level-1 = {计算机科学与技术},
  discipline-level-1* = {Computer~Science~and~Technology},
  date = {2023~年~12~月},
  date* = {December,~2023},
  bib-resource = {reference.bib},
}

% undergraduate setup
% \documentclass[bachelor, comfort]{shtthesis}
% \shtsetup{
%   title = {\ShtThesis{}~v\version{}\\使用说明},
%   title* = {A~User's~Guide~to\\\ShtThesis{}~v\version{}},
%   keywords = {上海科技大学,学位论文,\LaTeX{}},
%   keywords* = {ShanghaiTech~University, Thesis, \LaTeX{}},
%   date = {2021~年~02~月},
%   date* = {02~/~2021},
%   author = {李润东},
%   author* = {Rundong~Li},
%   author-id = {36273800},
%   entrance-year = {2017},
%   institution = {信息科学与技术学院},
%   institution* = {School~of~Information~Science~and~Technology},
%   supervisor = {范睿},
%   supervisor* = {Rui~Fan},
%   discipline = {计算机科学与技术},
%   discipline* = {Computer~Science~and~Technology},
%   bib-resource = {reference.bib},
% }

% `latex' and `shell' environments are adapted from `thuthesis'
\usepackage{listings}
\newcommand\prompt{\textup{\$}}
\lstdefinestyle{lstStyleBase}{%
  basicstyle=\small\ttfamily,
  aboveskip=\medskipamount,
  belowskip=\medskipamount,
  lineskip=0pt,
  boxpos=c,
  showlines=false,
  extendedchars=true,
  upquote=true,
  tabsize=2,
  showtabs=false,
  showspaces=false,
  showstringspaces=false,
  numbers=none,
  linewidth=\linewidth,
  xleftmargin=4pt,
  xrightmargin=0pt,
  resetmargins=false,
  breaklines=true,
  breakatwhitespace=false,
  breakindent=0pt,
  breakautoindent=true,
  columns=flexible,
  keepspaces=true,
  gobble=0,
  framesep=3pt,
  rulesep=1pt,
  framerule=1pt,
  frame=l,
  rulecolor=\color{ShtRed},
  backgroundcolor=\color{gray!5},
  stringstyle=\color{green!40!black!100},
  keywordstyle=\bfseries\color{blue!50!black},
  commentstyle=\slshape\color{black!60},
  escapeinside={`'},
}
\lstdefinestyle{lstStyleShell}{%
  style=lstStyleBase,
  language=bash}
\lstdefinestyle{lstStyleLaTeX}{%
  style=lstStyleBase,
  language=[LaTeX]TeX}
\lstnewenvironment{latex}{\lstset{style=lstStyleLaTeX}}{}
\lstnewenvironment{shell}{\lstset{style=lstStyleShell}}{}

\providecommand{\TODO}[1]{{\color{red}{#1}}}

\usepackage{hologo}
\ifluahbtex
  \usepackage{emoji}
\else
  \providecommand{\emoji}[1]{ \fbox{\emph{#1}} }
\fi

\usepackage{subcaption}
\usepackage{ctable}
\usepackage[list=off]{bicaption}
\usepackage{lipsum}
\captionsetup[figure][bi-second]{name=Figure}
\captionsetup[table][bi-second]{name=Table}

\makeatletter
  \def\ifundergraduate{\ifsht@undergraduate}
  \def\ifgraduate{\ifsht@graduate}
\makeatother

\newcommand{\citepen}[1]{{\defcounter{gbcitelocalcase}{2}\citep{#1}}}
\newcommand{\citeten}[1]{{\defcounter{gbcitelocalcase}{2}\citet{#1}}}

\begin{document}

\maketitle

\frontmatter
% A - 中英文摘要
\begin{abstract}[flattitle]
  流体仿真技术在诸多现实领域有着重要的指导意义,同时也是计算机图形学、计算流体力学等学科中重要的研究内容。然而在大规模的复杂场景下,使用现有方法对湍流进行高效、精准仿真,通常需要巨大的计算资源和时间开销。如何提升高雷诺数湍流仿真的效率与精度依然是极具挑战性的问题。同时由于不同领域的侧重点不同,不同的流体仿真方法开始分化,并逐渐只被应用于特定领域,这在一定程度上限制了流体仿真方法的灵活性与应用范围。针对这一系列问题,本文在格子玻尔兹曼方法的框架下,针对不同领域的应用提出了新的介观流体仿真方法,进一步提升格子玻尔兹曼方法的精度和稳定性,同时通过自动化的前处理,与图形处理单元 (Graphics Processing Unit, GPU) 优化算法,实现湍流的准确、高效仿真,以应用于视觉动画、工业产品设计、气动声分析等多个领域,弥合不同领域间流体仿真方法的差异。本文的主要技术创新点如下:
  \begin{enumerate}
      \item 本文提出面向视觉动画的通用流体仿真及流固耦合方法。针对于计算机图形学领域追求高效、稳定的流体仿真方法的特点,本文提出了基于速度修正和简单反弹边界方法的混合边界方法。该方法在精度和稳定性上相比现有的简单反弹边界方法都有所提升。并且可以在在湍流情况下同时处理亚网格尺度物体和正常尺度物体,展现出很强的通用性。
      \item 本文提出更加通用的高精度流体仿真方法。对精度要求更严格的领域,如工业产品设计,本文提出了新的单点插值反弹边界处理方法。该边界处理可以同时处理静态与动态物体,并利用单点处理提升了计算效率。同时,本文还提出了新的基于熵优化的累积量碰撞模型,该模型在高雷诺数流体仿真中依然保持稳定,并有着相比现有方法更高的精确度。通过与现实物理实验的对比,本文验证了该套流体仿真方法有着非常高的精准度,同时维持了很高的计算效率,使其可以应用在计算机图形学、计算流体力学、计算气动声学等不同领域,成为统一的流体仿真方法。
      \item 本文提出高效、自动化的流体仿真框架。为了使上述的流体仿真方法得到足够精确的物理量,流体仿真需要在极高的分辨率下进行,这势必要求使用非常高的分辨率构建计算网格。而目前这一个过程对于使用者来说非常繁琐冗长。本文在系统实现层面,阐述了自动化流体仿真的整体框架。其中包括了自动的计算网格构建方法、与自动的流体区域划分。该方法可以在有复杂几何的同时,完成高效的内、外流体区分及包含自动细分的多分辨率网格构建,从而支持高精度的流体仿真。
      \item 本文提出相关的算法优化以及高效GPU实现。本文还提出了针对上述方法的一些算法优化,如快速的几何计算算法,以降低额外的计算消耗。并且,针对上述每一部分,本文都阐述了相关的GPU优化算法,以优化整体的运行效率。借助现代的GPU硬件,利用本文的方法可以在几小时内完成高精度的流体仿真。
  \end{enumerate}
  
  除了提出新的流体计算方法与框架,本文还在计算机图形学、计算流体力学、计算气动声学等领域中,进行了大量的验证与实验,以证明我们新的流体仿真方法的准确性与应用潜力,并有利地推动格子波尔兹曼方法在这些领域的进一步应用与发展。
\end{abstract}

\begin{abstract*}[flattitle]

\end{abstract*}

\makeindices

\ifgraduate
% B - 符号列表
% \begin{nomenclatures}
%   \header[单位]{符号}{说明}
%   \item[$\symup{{m^{2} \cdot s^{-2} \cdot K^{-1}}}$]{$R$}{the gas constant}
%   \item[$\symup{{m^{2} \cdot s^{-2} \cdot K^{-1}}}$]{$C_v$}{specific heat capacity at constant volume}
%   \item[$\symup{{m^{2} \cdot s^{-2} \cdot K^{-1}}}$]{$C_p$}{specific heat capacity at constant pressure}
%   \item[$\symup{{m^{2} \cdot s^{-2}}}$]{$E$}{specific total energy}
%   \item[$\symup{{kg \cdot m \cdot s^{-3} \cdot K^{-1}}}$]{$k$}{thermal conductivity}
%   \item[$\symup{{kg \cdot m^{-1} \cdot s^{-2}}}$]{$S_{ij}$}{deviatoric stress tensor}
%   \item[$\symup{{kg \cdot m^{-1} \cdot s^{-2}}}$]{$\tau_{ij}$}{viscous stress tensor}
%   \item[$\symup{{1}}$]{$\delta_{ij}$}{Kronecker tensor}
% \end{nomenclatures}

\begin{nomenclatures}[缩写]
  \header{缩写}{全称}
  \item{CFD}{Computational Fluid Dynamics}
  \item{CG}{Computer Graphics}
  \item{LBM}{Lattice Boltzmann Method}
  \item{N-S}{Navier-Stokes}
  \item{LBE}{Lattice Boltzmann Equation}
  \item{CAA}{Computational Aeroacoustics}
  \item{DNS}{Direct Numerical Simulation}
  \item{SBB}{Simple Bounce-back}
  \item{IBB}{Interpolated Bounce-back}
\end{nomenclatures}

% \begin{nomenclatures}[算子 \& 说明]
%   \item{$\Delta$}{difference}
%   \item{$\nabla$}{gradient operator}
% \end{nomenclatures}
\fi

\mainmatter
% C - 正文
\chapter{引言}

% Sec 1.1
\section{研究背景与意义}

% Sec 1.2
\section{研究现状}

% Sec 1.2.1
\subsection{计算机图形学中的流体仿真方法}

% Sec 1.2.2
\subsection{计算流体力学中的流体仿真方法}

% Sec 1.2.3
\subsection{格子玻尔兹曼方法及其边界处理}

% Sec 1.3
\section{研究内容和贡献}

% Sec 1.4
\section{论文的组织结构}
\chapter{背景知识}

% Sec 2.1
\section{研究背景与意义}
\chapter{面向CG的通用边界处理方法}
\label{sec:siga21}

% Sec 3.1
\section{背景与动机}
流固耦合对于复杂的视觉现象仿真有很重要的作用,流体和固体之间的相互作用对两者的运动皆有影响,从而成为一个时间上的迭代过程。这种相互作用在数值上的计算是非常复杂的,并且,在有薄壳或细棒的场景中,流固耦合的计算是更加艰巨的。我们这里定义的薄壳或细棒是在某一个或某两个维度上非常狭窄的物体,以至于在这些维度上,物体的大小是远小于网格的尺度的。由于薄壳或细棒的这种特性,它们的特征在网格中很难被捕捉到,以至于经常发生泄漏或穿透等现象。

虽然有工作展示了薄壳~\cite{DiscreteShells,Bridson:2003} 或细棒~\cite{DiscreteRods} 的仿真,也包括它们在黏性流体~\cite{Fei-2018,Takahashi:2019,Fei-2019} 或非湍流~\cite{Azevedo-2016} 中的耦合,在动力学方法中使用扩散界面浸入边界法可以在湍流中或者更加稳定、高效的仿真结果~\cite{Li-2018,Li-2020}。

在本章中,我们介绍一个在LBM框架下的高效且通用的双向流固耦合边界处理方法,以可同时求解固体任意维度为亚网格尺度的情况。我们的方法将简单反弹边界方法与一个速度修正方法混合,以克服之前方法的缺点,提升仿真的稳定性与视觉效果。并且,通过几何上的近似,与实现层面的GPU优化,我们的方法相比Chen等~(\citeyear{Chen-2021}) 提出的LBM优化方案,在仿真效率上有着数倍的提升。

因为LBM使用笛卡尔网格来离散空间空间,而固体的边界一般不与网格对齐,于是出现了切削网格的概念,即网格被固体边界所切割,如图~\ref{img:cutcell_and_interpolation} 中的绿色网格。一般在LBM中需要对切削网格进行特殊处理,以刻画流体与固体间的相互作用。现有的各类基于切削网格的边界处理方法虽不能在效率、准确度、稳定性上都尽善尽美,但也有各自的优势。如扩散界面浸入边界法不需要追踪网格随物体的变化,从而降低了求解的复杂度。但它通过施加惩罚力来描述边界对流体的作用,不能准确刻画边界的形状,从而不适用于薄物体的仿真,如图~\ref{img:cutcell_and_interpolation} 所示。而简单反弹边界方法虽然可以通过分布函数的反弹阻止流体泄露 (即使是薄物体),但是因为它的准确性有限,在仿真中会产生不正确的速度分布,影响仿真的稳定性,如图~\ref{img:Immersed_Bounce_back}所示。

我们提出混合方法的一个重要动机是,这两类方法实质上是互补的。在反弹边界方法中,介观尺度的分布函数反弹从宏观上看,构成了部分浸没边界法中所需的惩罚力。而其本身又有着可以防止流体穿过薄物体的性质,所以我们可以施加一个额外的辅助惩罚力,对反弹边界法进行修正。我们注意到这个辅助惩罚力会比浸没边界法中原本的惩罚力要小很多。这样的混合方法可以满足我们在正确求解薄物体的同时,对精度与效率的需要。

\begin{figure}[htb]
    \centering
      \includegraphics[width=0.95\columnwidth]{figures/cutcell_and_interpolation.png}
    \bicaption{切削网格中的速度插值。左图:因为固体边界存在而产生的切削网格 (图中标为绿色)。右图:流体中的切削网格格点$\bm{x}$ (图中标为红色圆圈) 的速度,需要通过固体边界的投影点$\bm{x}{'}$ (图中标为蓝色圆圈),与该投影点到格点的延长线与下一个网格的交点$\bm{x}{''}$ (图中标为绿色圆圈) 进行速度的线性插值得到。}{Interpolation of velocity on cut-cell nodes. Left: cut-cell (marked in green) intersecting a solid boundary; Right: the velocity on a cut-cell node $\bm{x}$ inside the fluid region (red circle) needs to be interpolated using the velocities of its projected point $\bm{x}{'}$ onto the solid boundary (blue circle) and the intersected point between the ray from the projected point to the cut-cell node and the interpolated fluid point $\bm{x}{''}$ (green circle), where the velocity can be reliably evaluated through simple linear interpolation.}
    \label{img:cutcell_and_interpolation}
\end{figure}

\begin{figure}[htb]
    \centering
      \includegraphics[width=0.97\columnwidth]{figures/comparison_with_ib.png}
    \bicaption{浸没边界法造成的泄露。当有薄板存在时,Li等~(\citeyear{Li-2020}) 所使用的浸没边界法 (左图) 与我们的方法的对比 (右图)。因为惩罚力会分布到薄板两侧,扩散边界浸没边界法可能会产生泄露 (见图中红色方框),而我们的方法可防止此现象。}{Leakage of immersed boundary. For a kinetic fluid simulation coupled with a thin plate, the immersed boundary method used by Li et al.~(\citeyear{Li-2020}) (left) is compared with our method (right). Due to force spreading in both sides of the thin plate, this recent coupling approach employing a diffuse-interface immersed boundary method may generate leakage through the plate (see red box), while our method prevents this issue by design.}
    \label{img:comparison_with_ib}
\end{figure}

\begin{figure}[htb]
    \centering
      \includegraphics[width=0.95\columnwidth]{figures/ib-mbb.png}
    \bicaption{不同雷诺数下的简单反弹边界处理方法。(a) 在低雷诺数下,简单反弹边界没有造成视觉不正确的现象;(b) 但在高雷诺数下,固体边界会有很强的锯齿现象,从而对稳定性有较大影响;(c) 我们的方法在与 (b) 同样的高雷诺数情况下看不到视觉不正确的现象。}{Simple bounce-back under different Reynolds numbers. (a) a plain simple bounce-back scheme generates no visual artifacts for  flows with low Reynolds number; (b) however, it exhibits strong aliasing artifacts near the solid boundary for high Reynolds number flows, which may seriously influence stability; (c) our new boundary treatment for the same high Reynolds number as in (b) is artifact-free, generating the type of vortices expected from such an example.}
    \label{img:Immersed_Bounce_back}
\end{figure}

% Sec 3.2
\section{方法}
我们的混合方法主要包含下面三个部分:
\begin{itemize}
\item 双面简单反弹边界:我们首先介绍一个简单反弹边界方法的变种,即切削网格边界的两侧都会发生分布函数的反弹,而不只在一侧发生,从而不再需要追踪格子随物体运动的状态变化;
\item 切削网格的速度修正:我们通过在切削网格内进行单边的速度修正,大幅提升仿真的稳定性,同时压制边界上的速度误差;
\item 固体受力:最后我们通过介观与宏观的混合,求出流体转移至固体的动量,从而获得固体受力以驱动固体运动。
\end{itemize}

我们将在下面依次介绍这三部分内容。

\subsection{双面简单反弹边界}
在第~\ref{sec:boundary_treatment} 节中,我们回顾了简单反弹边界方法。一般来说,反弹边界都只应用于流体点。然而随着物体的运动,流体点可能被固体覆盖成为固体点,固体点也可能重新变为流体点。这个转换过程不止需要追踪格点状态,并且需要进行速度的插值,甚至外插。这样会在高雷诺数下在固体边界上产生很强的耗散误差。为了避免这一现象,我们对所有切削网格内的点都进行分布函数反弹,而不只针对流体点。我们称该方法为\emph{双面简单反弹边界方法}。该方法依旧可以避免流体泄露,即使有亚网格物体存在。但这并没有改善固体边界的速度锯齿现象与大速度梯度造成的误差,所以我们引入下一节中介绍的速度修正方法。

\subsection{切削网格的速度修正}
因为我们谋求的是低阶 (线性) 精度的边界处理,我们提出在双面简单反弹边界之后,通过沿固体边界的法向方向对速度线性插值,来对速度场进行修正。修正的方式为对流场施加惩罚力。更具体地说,我们先通过切削网格点周围的固体和流体速度插值,得到一个切削网格点上的期望速度。之后通过当前速度与期望速度的差计算惩罚力,从而修正速度场。

\paragraph{切削网格点上的期望速度}
切削网格点上的期望速度可通过其邻近的可靠速度插值得来。对于处于固体内部的切削网格点,它们的期望速度必须与那一点的固体速度一致,这个速度可以通过固体的运动状态直接获得。
对于流体中的切削网格点,我们通过一个线性插值,获得其期望速度。该插值方法如图~\ref{img:cutcell_and_interpolation} 中右图所示。对于切削网格点$\bm{x}$点,我们首先计算它在固体边界的投影点位置$\bm{x}'$,之后从$\bm{x}'$向$\bm{x}$打一条射线,该射线与下一个网格面的交点为$\bm{x}''$。如果这个面上的所有点都不是切削网格点,$\bm{x}''$点的速度就可以通过该面上的点线性插值得来 (二维中是线性插值,三维中是双线性插值)。如果这个面上存在点是切削网格点,则可寻找下一个交点,直到找到满足条件的交点。此时$\bm{x}$点的期望速度则可通过线性插值得到:
\begin{equation}  \label{eq:vel_lerp}
\hat{\bm{u}}(\bm{x})=(1-\alpha)\bm{u}_s + \alpha \bm{u}(\bm{x}'')\;,
\end{equation}
其中$\alpha=\|\bm{x}-\bm{x}'\|/\|\bm{x}''-\bm{x}'\|$,$\bm{u}_s$是固体边界上投影点$\bm{x}'$的速度。

\paragraph{基于惩罚力的速度修正}
由于切削网格点$\bm{x}$上的期望速度$\hat{\bm{u}}(\bm{x})$可能与分布函数反弹后的宏观速度$\bm{u}(\bm{x})$有偏差,我们构造如下的惩罚力$\bm{F}_p(\bm{x})$
\begin{equation} \label{eq:penaltyForce}
\bm{F}_p(\bm{x}) = \hat{\bm{u}}(\bm{x})-\bm{u}(\bm{x})\;,
\end{equation}
施加在点$\bm{x}$上,做法与Li等~(\citeyear{Li-2020}) 中相同。我们将惩罚力投影至分布函数空间,分布函数使用最高阶的埃尔米特多项式展开,以维持精度与稳定性。由于反弹边界已经承担了大部分的边界处理,这里的惩罚力会比传统浸没边界法中的惩罚力小很多,只作为一个速度的修正出现,于是我们也不需要采用Li等~(\citeyear{Li-2020}) 中的时间重缩放来缩短物理时间步长从而提升稳定性。这可以使仿真效率进一步提升。

\begin{figure}[htb]
    \centering
      \includegraphics[width=0.95\columnwidth]{figures/cutcell_special_cases.png}
    \bicaption{两种邻近物体的情况。左图:从固体边界点$\bm{x}'$起始的射线经过$\bm{x}$点,在交到网格面之前,便相交于另一个物体上。这种情况,$\bm{x}''$是另一个固体边界点。右图:一个切削网格点可能完全被切削网格面包围,所以找不到任何非切削网格面。}{Two cases of solid proximity. Left: the ray starting from a boundary point $\bm{x}'$ through a grid node $\bm{x}$ may hit another boundary surface point before it intersects with the non-cut-cell face; in this case, $\bm{x}''$ locates at another boundary point. Right: a cut-cell node may be surrounded by all cut-cell faces, so a ray may not hit any non-cut-cell face nearby.}
    \label{img:handling_proximity}
\end{figure}

\paragraph{一些特殊情况}
当流体中的物体间距离接近网格尺度的时候,有时$\bm{x}''$的速度会无法从周围的非切削网格点插值得到。一种情况是从$\bm{x}'$点射出的射线在交到网格面之前,便相交于另一个物体上 (图~\ref{img:handling_proximity} 中左图)。那么由于这个交点的速度是已知的,我们可以用这一速度替代先前提到的非切削网格点的速度。另一种情况是,因为物体之间过于接近,射线无法找到任何非切削网格面。这个时候我们只得抛弃速度修正过程,而只采用双面简单反弹边界作为边界处理。

\begin{figure}[htb]
  \centering
    \includegraphics[width=0.9\columnwidth]{figures/sub_grid_approximation.png}
  \bicaption{亚网格近似。在亚网格物体存在时,我们将每个切削网格点投影至其最近的边界采样点。这等同于使用一个包围体积 (橘色区域) 来近似一个网格中的多个亚网格物体。}{Sub-grid approximation. To handle sub-grid scale solid structures, we project each cut-cell node onto its nearest boundary sample point. This basically amounts to using bounding volumes (orange regions) to approximate multiple thin structures within a cut-cell.}
  \label{img:sub_grid_approximation}
\end{figure}

\paragraph{亚网格近似}
当数量众多的薄板或细棒存在时,有可能一个流体网格中会包含多个这样的物体,尤其在网格较粗的情况中 (见图~\ref{img:sub_grid_approximation} ),此时边界的精度是无法保证的。此时,我们使用一个包围体积,来近似这些存在于一个网格中的多个物体。所以我们依然可以将流体点投影至距离最近的固体边界点来计算惩罚力。

\paragraph{讨论}
我们的惩罚力修正是基于法向方向的线性速度插值。这个修正可被认为是一个滤波器,来压制在高雷诺数下,简单反弹边界造成的固体边界周围的速度震荡。这个修正可以有效地提升在湍流流固耦合仿真的稳定性,使得我们在相对粗糙的网格中也能获得视觉上可信的仿真结果。然而,这依然只是一个一阶的线性修正,所以要获得高物理精度的仿真需要高分辨率网格作为基础。但考虑到我们主要面向图形学应用,我们不考虑在构造更加复杂的速度修正方法。

\subsection{固体受力}
因为要做到流固的双向耦合,我们需要讨论流体中固体的受力,也即流体传递给固体的动量。我们在边界中使用了反弹边界与速度修正的结合,所以我们在讨论固体受力时也分为两部分来讨论。我们首先讨论双面反弹边界对固体受力的贡献。在双面反弹边界之后,我们先将由公式~\ref{eq:bounce-back} 带来的动量交换进行累计。更具体地说,迁移过程后,$t$时刻$\bm{x}$点的动量可以表达为
\begin{equation}
  \bm{p}(\bm{x}) = \sum_{i}\bm{c}_if^*_i(\bm{x},t).
\end{equation}
如果$L$是$\bm{x}$点切削速度方向的集合 (即与固体边界相交的速度方向),每一个$L$中的速度方向都会发生分布函数反弹,那么$\bm{c}_j$方向的流固动量交换是
\begin{equation}
  \smash{-\bm{c}_{j}\,(f^*_{j}(\bm{x}) + f^*_{j'}(\bm{x}))}.
\end{equation}
从而,在$\bm{x}$点交换的动量和$\Delta \bm{p}$为
\begin{equation}
\Delta \bm{p}(\bm{x})= - \sum_{j\in L} \bm{c}_{j}\,(f^*_{j}(\bm{x}) + f^*_{j'}(\bm{x})),
\end{equation}
其中我们不将$(\Delta x)^3$显式写出 ($\Delta x$是网格大小),因为在正则化LBE空间中这一项为1。
那么反弹边界部分所贡献的固体受力是所有点动量交换的和,即
\begin{equation}
\bm{F}_{B}\equiv \sum_{\bm{x}} \Delta \bm{p}(\bm{x}),
\end{equation}
其中除以$\Delta t$ (时间步长) 也被隐去,因为其在正则化LBE空间中也为1。类似地,我们可以获得总的力矩
\begin{equation}
\bm{\tau}_{B}\equiv \sum_{\bm{x}} (\bm{x}-\bm{x}_{C})\times\Delta \bm{p}(\bm{x}),
\end{equation}
其中$\bm{x}_{C}$是物体质心的位置。

接下来我们继续讨论速度修正对固体受力的贡献。我们认为这个速度修正的来源依旧是固体的存在,所以这个力的来源是固体本身。那么固体理应受到一个与惩罚力大小相同,方向相反的反作用力。那么这一部分固体所受到的合力$\bm{F}_{C}$与合力矩$\bm{\tau}_{C}$可以表达为
\begin{equation}
\bm{F}_{C} = - \sum_{\bm{x}}\bm{F}_p(\bm{x}), \quad \bm{\tau}_{C} = - \sum_{\bm{x}} (\bm{x}-\bm{x}_{C})\times\bm{F}_p(\bm{x}).
\end{equation}

所以固体受到的合力$\bm{F}_s$与合力矩$\bm{\tau}_s$是上述两部分贡献的和:
\begin{equation}
\bm{F}_s = \bm{F}_{B} + \bm{F}_{C} \text{ and } \bm{\tau}_s = \bm{\tau}_{B} + \bm{\tau}_{C}.
\end{equation}

\section{算法优化}
\subsection{几何近似}
我们的混合边界处理中需要许多的几何计算,如将流体点投影至固体表面,或将速度方向与固体边界进行求交等。我们还需要识别切削网格点,以及它们的状态 (属于流体区域或固体区域)。许多工作阐述了如何准确地进行这样的几何计算~\cite{Azevedo-2016,Robinson:2009}。但由于固体边界的形状可能十分复杂,这一过程可能非常耗时,不利于维持LBM高并行度的优势。
为了在效率与精度间取得平衡,我们提出一系列的几何近似计算方法,以在不影响视觉可信度的情况下提升计算效率。

\begin{figure}[!htbp]
  \centering
    \includegraphics[width=0.97\columnwidth]{figures/geometric_computing.png}
  \bicaption{高效几何近似计算。对于一个固体,我们首先对它的边界采样 (a),然后切削网格点的投影点可以被近似为距其最近的采样点 (b)。为了检验某个速度方向 (图中标为黄色) 是否与固体边界相交,我们可以检查这个方向的两端是否有固体点存在。如果一端是固体点另一端是流体点,则我们认为其跨过了固体边界 (c) (图中绿色点为固体点,橘色点为流体点)。对于两端都是流体点的情况,我们检查这两个流体点投影点的法向,如果它们有类似的朝向 (d) 则认为该方向没有跨过边界,相反则认为其跨过边界 (e)。在GPU实现时,我们从所有的固体采样点出发,令每个固体采样点寻找其周围的流体切削网格点,而不是从流体点出发。}{Efficient geometric approximation. Given a solid geometry, we first sample its boundary (a); then the projected points of cut-cell nodes can be simply approximated by the nearest sample (b). To examine whether a link (yellow) crosses the solid boundary, we first check if solid nodes are involved. If the ends of the link are fluid and solid nodes, then it crosses the solid boundary (c) (green cut-cell nodes are inside the solid, whereas orange ones are in the fluid). For links connecting fluid nodes, we check the normals of the two projected cut-cell nodes forming the link; if the two normals have the same orientation (d), the link does not cross the solid boundary; otherwise, it intersects the solid boundary (e). For GPU implementation, we parallelize over solid samples instead of cut-cell nodes, where each solid sample will check its surrounding region of cut-cell nodes.}
  \label{img:geometric-computing}
\end{figure}

\paragraph{表面采样与切削网格识别}
为了加速计算,我们可以使用采样点而不是实际的几何来表达边界。这一过程需要对固体表面进行均匀采样,如使用泊松圆盘采样 (Poisson-disk Sampling)~\cite{dunbar2006spatial}。注意这也要求固体模型应该是密封的。每一个采样点$\bm{x}_s$在采样时可获得一个由边界朝外的法向$\bm{n}(\bm{x}_s)$。之后如果有网格中包含至少一个边界采样点,这个网格就可以被识别为切削网格 (如图~\ref{img:geometric-computing} (a))。

\paragraph{高效切削网格投影}
在几何计算中一个最重要的 (也可能是最耗时的) 部分是将流体点$\bm{x}$投影至固体表面上,以得到投影点位置$\bm{x}'$。传统方法可能会构建一个树形结构,然后搜索距离$\bm{x}$点最近的三角形~\cite{wang-2012}。在这里我们使用一个高效的近似算法来替代这一过程,我们在固体表面选择一个距离$\bm{x}$点最近的采样点$\bm{x}_s$,并满足
\begin{equation}\label{eq:is_in_fluid}
(\bm{x}-\bm{x}_s)\cdot \bm{n}(\bm{x}_s) \geq 0\;
\end{equation}
来近似采样点$\bm{x}'$。公式~\ref{eq:is_in_fluid} 这个约束可以避免我们选择到错误朝向的采样点。如果没有满足这个约束的采样点,那么这个点应该被标记为固体点 (如图~\ref{img:geometric-computing} (b))。这个基于采样点的技术可以达到很高的并行度,但精度很显然依赖于采样密度。所以在实现中,我们需要保证每个切削网格内有足够的采样点数。一般我们设置泊松圆盘采样的采样距离不超过网格大小的一半。

\paragraph{近似反弹边界处理}
在双面反弹边界处理中,我们必须要识别出与固体边界相交的速度方向 (见图~\ref{img:bounce_back_scheme} )。对于厚物体,由于我们已经对切削网格点属于流体点或固体点有所标记,我们只许对速度方向的两端点的状态进行识别。如果两个端点分别为流体点和固体点,则这个速度方向与固体边界有相交。但是对于薄物体,可能存在两端均是流体点,但依然与固体表面相交的情况。所以我们要进行特殊识别。
对于从$\bm{x}$至$\bm{y}$点的速度方向,我们检查这两个点的投影点$\bm{x}'$与$\bm{y}'$的法向朝向是否不一致,即$\bm{n}(\bm{x}')\!\cdot\!\bm{n}(\bm{y'})\!<\!0$ (如图~\ref{img:geometric-computing} (d)与(e))。
如果朝向不一致 (图~\ref{img:geometric-computing} (e) 即为此情况),则我们认为$\bm{x}$点与$\bm{y}$点处于边界的不同侧,所以这个速度方向是与固体表面相交的。则我们应在这个方向上进行反弹边界处理。

\subsection{GPU实现}
目前为止所描述的我们的算法,包括几何近似,我们都希望它们尽可能的简单,且只涉及局部计算。这样可以令GPU的实现较为简单直接,且有尽可能高的计算效率。在实现时我们应用了Chen等~(\citeyear{Chen-2021}) 中提出的加速技术,与一些针对我们算法设计的加速实现。这些实现方法在本节进行讨论。

\paragraph{存储布局}
对于LBM来说,在并行时一般是按方向的。即先计算$f_0$,之后计算$f_1$、$f_2$、……,以此类推。在这种情况下数组结构体 (Structure-of-arrays, SoA) 是更好的存储布局,因为SoA结构可以提升缓存利用率,从而提升整体性能~\cite{Chen-2021}。我们在实现中也同样使用SoA布局。

\paragraph{快速碰撞计算}
Chen等~(\citeyear{Chen-2021}) 还讨论了在碰撞过程中所需要的中心矩投影矩阵$\bm{M}(\bm{u})$的逆矩阵$\bm{M}(\bm{u})^{-1}$计算问题。作者们在其中提到逆矩阵应该使用解析式进行计算来保证碰撞的精度。然而由于解析式非常复杂,在GPU计算时,会占用过多的寄存器,导致GPU占用率低。但是我们注意到,如果将$\bm{M}(\bm{u})^{-1}$的解析式进行LU分解~\cite{fei2018three},分解后的三角矩阵中很多元素是非常接近于0的。所以我们可以将这些非常接近于0的元素置0,之后使用稀疏矩阵的数据格式进行存储。这样并不会影响整体计算的精度 (所造成的误差低于机器误差),但是寄存器使用量可以大幅减少。在同样的GPU上进行测试,我们的碰撞过程的效率约是Chen等~(\citeyear{Chen-2021}) 中碰撞过程的三倍。

\paragraph{并行的切削网格点识别与投影}
我们在上一小节讨论了我们算法中的一个关键步骤是将切削网格点投影到物体表面,这一过程可以通过寻找距离该点最近的固体采样点来近似。一般这一过程会将流体边界点作为并行单元,来搜索距其最近的采样点。但是这样会造成GPU负载不均衡,从而降低GPU占用率。
所以我们在这一过程中,将固体采样点作为并行单元。对于每一个采样点,我们计算该点到周围的切削网格点的距离 (见图~\ref{img:geometric-computing} (f)),并且比较这个距离与切削网格点当前记录的最近点的距离。如果距离更小的话则更新该采样点为最近点。因为多个线程可能会同时访问同一个流体点,这一过程需要使用原子比较计算。注意公式~\ref{eq:is_in_fluid} 依然需要满足,所以切削网格点状态可以被同时识别。
这样的好处是,每个固体采样点附近一定存在切削网格点,并且数量接近。这样可以大幅提升GPU的负载均衡,减少无用的线程造成的线程开销,提升GPU占用率。

\begin{figure}[!htbp]
  \centering
    \includegraphics[width=0.99\columnwidth]{figures/comparsion_car_design.png}
  \bicaption{汽车的空气动力学设计。通过我们的方法对没有尾翼的汽车进行仿真 (a),其结果与一个相似的车辆模型在风洞可视化的结果匹配 (b)。之后添加一个小型的扰流板 (图中红色方框) 到汽车尾部之后,仿真中的尾流发生显著变化 (c),与带有扰流板的风洞测试相似 (d)。这展示出我们的方法可以快速、有效地为气动外形设计服务。图中的速度模值单位为$km/h$。图 (b) 与 (d) 的版权来自 Feltham~(\citeyear{youtube_2014})。}{Aerodynamic design of a car model. Simulating the airflow around a car model without a spoiler using our solver (a) matches the wind tunnel visualization for a similar car model (b); A small and thin spoiler (in red box) added on the back of the car model (c) changes the wake flow of the car model in our simulation quite significantly, in line with a wind-tunnel test of a car model with a spoiler (d) --- indicating that our solver offers an efficient, yet predictive tool of air flows for computational shape design. The velocity magnitude is measured in $km/h$. Image (b) and (d) courtesy of Feltham~(\citeyear{youtube_2014}).}
  \label{img:comparsion_car_design}
\end{figure}

\begin{figure}[!htbp]
  \centering
    \includegraphics[width=0.99\columnwidth]{figures/car_wind_tunnel.png}
  \bicaption{车的风洞测试。车周围的气流可视化展示了车身的设计使得气流直到后备箱才发生边界层分离,这样可以有效减少风阻与震动。图片版权来自\textit{Auto Motor und Sport}, \copyright Frank Herzog/Sportauto.}{Wind-tunnel test of a car. A wind tunnel visualization of the airflow around a car shows that the design of the body delays boundary layer separation until the trunk, which reduces drag and vibration in practice. Image courtesy of \textit{Auto Motor und Sport}, \copyright Frank Herzog/Sportauto.}
  \label{img:car_wind_tunnel}
\end{figure}

\begin{figure}[!htbp]
  \centering
    \includegraphics[width=0.85\columnwidth]{figures/taylor-couette-setting-2d.png}
  \bicaption{二维泰勒-库埃特流。左图:两个以$\bm{O}$点为圆心的同心圆,半径分别为$r_1$、$r_2$,角速度为$\Omega_1$、$\Omega_2$,通过旋转在中间的圆环中形成泰勒-库埃特流。右图:二维泰勒-库埃特流的解析解速度场模值的可视化。}{2D Taylor-Couette flow. Left: the Taylor-Couette flow runs between two concentric circle boundaries locating at origin $\bm{O}$ with radii $r_1$ and $r_2$ whose rotating speeds are $\Omega_1$ and $\Omega_2$, respectively. Right: visualization of velocity magnitudes of the analytical solution between two rotating circles.}
  \label{img:taylor-couette-setting-2d}
\end{figure}

\section{对比与仿真结果}
此节中我们将我们的方法与图形学中近期提出的LBM仿真方法~\cite{Li-2020} 进行定量与定性的对比,包括与实际实验的对比。我们还针对计算效率,讨论基于Li等~\citeyear{Li-2020} 的性能改进~\cite{Chen-2021} 与我们方法的比较。
为了分析我们的边界处理方法,并验证其精度优于现有方法,我们进行了一系列的二维数值实验。这些我们将在本节进行讨论。

\paragraph{二维泰勒-库埃特 (Taylor-Couette) 流仿真}
我们首先对二维泰勒-库埃特流进行仿真。这是一个经典的二维流体基准测试,具体来说,测试的场景为两个同心圆 (厚度可不计) 分别以各自的速度旋转~\cite{xu2006immersed}。即在这个场景中,这两个圆既为薄物体,又为动态边界,所以十分适合对我们的方法进行精度的测试。该测试场景的示意可见图~\ref{img:taylor-couette-setting-2d} (a)。
两个圆的半径分别为$r_1 \!=\! 0.5$、$r_2 \!=\! 1.0$,角速度为$\Omega_1 \!=\! 1.0$、$\Omega_2 \!=\! -1.0$。雷诺数为$\text{Re} \!=\! 10$,对应运动粘度为$\nu \!=\! 0.1$。
在这样的设置下,两个圆之间的圆环会产生流场,并且这个流场存在解析解$\bm{u}^\text{ref}\!=\!(u_x^\text{ref},u_y^\text{ref})$。我们以两个圆的圆心为坐标轴原点,对于任意一个点$(x,y)$,我们可以用$r\!=\!\sqrt{x^2+y^2}$衡量该点到圆心的距离。那么对于$r\!\in\![r_1, r_2]$的点,流场的解析解为
\begin{equation}
u_x^\text{ref} = -\bigl(A + \frac{B}{r^2}\bigr)y, \hspace{4mm} u_y^\text{ref} = \bigl(A + \frac{B}{r^2}\bigr)x\;,
\end{equation}
\noindent 其中常数$A$与$B$的定义为:
\begin{equation}
A = \frac{\Omega_2 r_2^2 - \Omega_1 r_1^2}{r_2^2 - r_1^2}, \hspace{4mm} B= \frac{(\Omega_1 - \Omega_2)r_1^2r_2^2}{r_2^2 - r_1^2}.
\end{equation}
图~\ref{img:taylor-couette-setting-2d} (b) 展示了这个解析解速度分布的可视化。

\begin{figure}[!htbp]
  \centering
    \includegraphics[width=0.99\columnwidth]{figures/taylor-couette-compare-2d.png}
  \bicaption{二维泰勒-库埃特流仿真的对比。我们将不同求解器 (从左至右) 得到的不同分辨率 (从上至下) 的二维泰勒-库埃特流数值仿真结果进行可视化,可视化的值为相对误差,场景设置见图~\ref{img:taylor-couette-setting-2d}。图中从左至右依次为原始的简单反弹边界~\cite{Ladd-1994}、\cite{Li-2020} 中的动理学方法、我们的边界处理,以及ANSYS Fluent求解器。ANSYS Fluent在求解时使用了自适应的体网格,平均元素大小与其它求解器中的网格大小相当。}{Comparison of 2D Taylor-Couette flow simulation. We visualize the relative errors of the numerical solution to the 2D Taylor-Couette flow of Fig.~\ref{img:taylor-couette-setting-2d} computed by different solvers (columns) and different resolutions (rows). From left to right: the original bounce-back scheme~\cite{Ladd-1994}, the kinetic solver of~\cite{Li-2020}, our boundary treatment method, as well as the ANSYS Fluent solver using adaptive radial mesh for which the average element area matches the grid cell area used for the other solvers.}
  \label{img:taylor-couette-compare-2d}
\end{figure}

我们使用了不同的求解器,对该边界驱动的流体进行求解,求解器分别为 (a) 使用原始简单反弹边界~\cite{Ladd-1994} 的动理学方法、(b) DI-IBM~\cite{Li-2020}、(c) 我们的边界处理方法与 (d) 使用N-S方法的商业求解器ANSYS Fluent~\cite{Ansys-2014}。在使用ANSYS Fluent求解时,我们采用了自适应的体网格,平均元素大小与其它求解器中的网格大小相当。由于其使用的是贴体网格,ANSYS Fluent的误差应该会更小。
图~\ref{img:taylor-couette-compare-2d} 展示了不同求解器 (从左至右) 在不同分辨率 (从上至下) 下的相对误差的分布。从图中我们的方法相比其它方法拥有更小误差同时,也随着分辨率上升而更快地达到收敛。

我们还测量了不同求解器结果的加权$\ell_2$误差:
\begin{equation}\label{eq:error_metric}
\epsilon = \frac{\sum_i w(\bm{x}_i)\|\bm{u}(\bm{x}_i)-\bm{u}^{\text{ref}}(\bm{x}_i)\|_2}{\sum_i w(\bm{x}_i)\|\bm{u}^{\text{ref}}(\bm{x}_i)\|_2}\;,
\end{equation}
其中$\bm{x}_i$是流体域中的一个格点,$\bm{u}^{\text{ref}}$是解析解,权重$w(\bm{x}_i) \!=\! 4|r-\bar{r}|$,$\bar{r}\!=\!(r_1+r_2)/2$。这个权重存在的目的是更多地体现边界附近的误差,因为我们本质上在对比边界处理。图~\ref{img:taylor-couette-error-compare} 展示了不同求解器的误差随分辨率上升的变化情况。显然,我们的新边界处理比起其它两个动理学边界处理方法都有着更好的表现。并且,我们的方法在分辨率足够高时,也优于ANSYS Fluent。

\begin{figure}[!htbp]
  \centering
    \includegraphics[width=0.94\columnwidth]{figures/taylor-couette-error-compare.png}
  \bicaption{不同类型求解器的误差收敛情况。我们对二维泰勒-库埃特流进行仿真,并衡量了在不同分辨率下,不同类型的求解器的相对误差,以展示各个求解器的误差收敛情况。相对误差根据公式~\ref{eq:error_metric} 计算得到。}{Convergence for different types of solvers. We simulated the 2D Taylor-Couette flows with different types of solvers and measured the relative errors according to Eq.~\ref{eq:error_metric} to show the convergence for different solvers as resolution increases.}
  \label{img:taylor-couette-error-compare}
\end{figure}

\paragraph{与基于切削网格的二维N-S求解器的对比}
CG领域中的N-S方法已经发展得较为成熟,且拥有一系列不同的流固耦合方法。其中基于切削网格的方法是精度与计算效率均相对较高的方法。于是我们与~\cite{Azevedo-2016} 中的基于切削网格的N-S方法进行对比。该方法在切削网格构造了类似有限体积法中的网格以提升精度。我们使用该方法与我们的方法分别仿真了一个二维的薄板以角速度$\omega_s = 3\text{rad}/s$旋转的场景,并将得到的速度场展示在图~\ref{img:cut-cell-based-NS-solver} 中。由于Azevedo等~(\citeyear{Azevedo-2016}) 求解的是无粘度的欧拉方程,我们在我们的方法中也将粘度设为0 ($\nu\!=\!0$)。虽然Azevedo等的方法得到了不错的结果,但是在流场中只有大涡,即使流体本身是无粘的。我们注意到目前也有一些更加精准的N-S求解器,如~\cite{zehnder2018advection,qu2019efficient},但这些方法的计算效率是显著不如动理学方法的~\cite{Li-2020}。从上述分析可得,我们的方法拥有较为独特的优势,即提供了更为精准且高效的湍流仿真,同时可以对不同类型的复杂几何进行耦合。

\begin{figure}[!htbp]
  \centering
    \includegraphics[width=0.99\columnwidth]{figures/ns-compare-2d.png}
  \bicaption{与二维N-S求解器的对比。我们仿真一个二维的薄板在流体中旋转的场景,仿真分辨率为$256\times256$。顶图:由基于切削网格方法的N-S求解器~\cite{Azevedo-2016} 得到的速度场模值;底图:我们的方法获得的速度场模值。(a)、(c) 与 (b)、(d) 分别从仿真中的同一帧截取。虽然两个仿真方法的结果都在视觉上可以接受,但是我们的方法有着更多涡流的细节,证明了我们的方法更适合于湍流的仿真场景。}{Comparison with 2D N-S solver. We simulated a 2D thin plate spinning in a fluid using a grid resolution of $256\times256$. Top: velocity magnitude plot from a cut-cell-based NS solver~\cite{Azevedo-2016}; Bottom: velocity magnitude plot from our solver for the same frames as (a) and (b). Although both solvers retain good boundary velocities, ours produces much more detailed vortices, making it significantly more suitable for solving coupling in a turbulent flow scenario.}
  \label{img:cut-cell-based-NS-solver}
\end{figure}

\subsection{对比}
\begin{figure}[htb]
  \centering
    \includegraphics[width=0.99\columnwidth]{figures/comparsion_car_ib_ours.png}
  \bicaption{DI-IBM~\cite{Li-2020} (顶图) 与我们的方法 (底图) 进行气动仿真的对比,速度场截面使用颜色可视化 (单位为$km/h$)。我们的仿真更贴近图~\ref{img:car_wind_tunnel} 中风洞的实际实验,而DI-IBM在车顶即开始发生边界层分离,不切合实际情况。}{Comparison of aerodynamic simulations. DI-IB method~\cite{Li-2020} (top) vs. our result (bottom), using a visualization showing the color-coded velocity field magnitude (in $km/h$) of a cross section. While our simulation matches the wind-tunnel visualization of air flow around a car as Fig.~\ref{img:car_wind_tunnel} demonstrates, the DI-IB method, on the other hand, fails to have accurate boundary treatment, leading to unreasonable boundary layer separation at the top of the car.}
  \label{img:comparsion_car_ib_ours}
\end{figure}

\begin{figure}[!htb]
  \centering
    \includegraphics[width=0.99\columnwidth]{figures/comparison_delta_wing.png}
  \bicaption{三角翼仿真。通过我们的方法得到的薄板三角翼仿真结果 (左图) 与实际实验~\cite{Delery:2001} (右图) 的对比,展示出相同的前缘涡旋结构,表面我们的方法在薄结构边界层上的准确性。}{Delta-wing simulation. The airflow over a thin-shell delta-wing simulated with our hybrid coupling approach (left) matches experimental visualizations from~\cite{Delery:2001} (right), exhibiting the same spiral vortex structure near the leading edge of the wing and demonstrating the accuracy of our solver in capturing boundary layer flows around thin structures.}
  \label{img:comparison_delta_wing}
\end{figure}

\paragraph{与图形学中现有方法的对比}
最近,Li等~\citeyear{Li-2020} 提出了一个基于动理学方法的双向流固耦合湍流求解方法,该方法使用DI-IBM处理固体边界。然而这一方法有一定的局限性。首先因为DI-IBM在边界两侧施加惩罚力,其在求解亚网格物体的流固耦合时,会有流体穿透的现象。图~\ref{img:comparison_with_ib} 展示了一个三维的喷射流打向一个薄板,其中图 (a) 使用了Li等~\citeyear{Li-2020} 的方法,并出现了不应该存在的流体穿透现象 (图中红框)。图 (b) 使用了我们的方法,没有产生类似的现象。

另外,DI-IBM有精度不足的问题,使得在湍流仿真时得不到正确结果,尤其是在固体边界层上。我们对一个向前运动的汽车模型进行气流仿真来说明这一问题,见图~\ref{img:comparsion_car_ib_ours}。对于汽车外形来说,气流的分离点一般位于汽车尾部,并且可以通过扰流板等进一步影响尾流走向 (如图~\ref{img:comparsion_car_design} (d) 中的风洞实验)。但仿真结果显示,DI-IBM预测的边界分离点在车顶,完全不符合实际风洞实验,而我们的方法可以正确预测边界的分离点 (两次仿真除边界处理外使用同样的参数设置)。

为了展示如扰流板这样的小且薄的结构对汽车周围流场的影响,我们还在图~\ref{img:comparsion_car_design}中展示了有或无扰流板时汽车流场的仿真结果。从结果中我们可以看到,在添加了扰流板后,汽车的尾流发生了明显的变化。扰流板破坏了车身尾部表面的气流,产生湍流的同时减少了车辆尾部的升力,从而提升了车辆在高速时的操控性。这两个例子展示出了我们的方法针对于计算机图形学中已有LBM方法的优势。

\begin{figure}[!htbp]
  \centering
    \includegraphics[width=0.99\columnwidth]{figures/DJI_compare.png}
  \bicaption{推力的验证。我们通过旋翼旋转的气动仿真来测试旋翼的推力。我们在$288\!\times\!324\!\times\!288$分辨率下测试不同转速的误差结果见左图。右图为固定3600 rpm转速时,相对误差的收敛情况。注意我们采用的分辨率并不是非常高,所以结果的精度依然有提升空间。}{Thrust evaluation. From the simulation of the coupling between a rotating drone blade and the surrounding air, we can compute the estimated thrust values at different rotation speeds with a grid resolution of $288\!\times\!324\!\times\!288$ (left). The relative errors of our thrust value at 3600 rpm w.r.t. experimental measurements as the grid resolution increases (right) indicates convergence of our solver. Note that the tested grid resolutions are not very high, and more accurate results are expected for higher grid resolutions.}
  \label{img:DJI_thrust_compare}
\end{figure}

\paragraph{与实际实验的对比}
为了进一步验证我们方法在有亚网格物体时边界处理的精度,我们与现实世界中的实验进行对比。
首先我们与三角翼的风洞实验进行比较。该三角翼的后掠角度为$75^\circ$,攻角为$20^\circ$ (见图~\ref{img:comparison_delta_wing} (a))。在实验中,三角翼上方会产生稳定的螺旋涡流结构,提升气动升力 (这种升力被称为涡升力~\cite{anderson2010aircraft})。这种结构被广泛使用于现代飞行器的设计中,如图~\ref{img:teaser_concorde} 中的协和客机。该实验的可视化可见图~\ref{img:comparison_delta_wing} (b)~\cite{Delery:2001}。我们的仿真结果在视觉上与实验结果相匹配,展示出相似的螺旋涡流结构。除去这一定性的对比实验,我们还对旋翼的气动性能进行了定量分析。我们参照一项最近的专利~\cite{lin2020screw} 中的描述重建了旋翼模型,并测量旋翼在旋转时的推力 (计算方法参照~\cite{leishman_2016}),与专利中的数据对比。图~\ref{img:DJI_thrust_compare} (a) 展示了不同转速下的计算误差,显示出较好的一致性。值得注意的是,随着转速的提升,相对误差会升高。这一点的主要原因是在LBM中,随着参照速度的提升,LBM空间下粘度在不断减小,以至于我们在测试时使用的32位浮点数已无法有效保证碰撞过程的计算精度。我们还对收敛性进行了测试,即在固定转速下提升分辨率,结果可见图~\ref{img:DJI_thrust_compare} (b)。这一结果显示了我们方法误差随分辨率的收敛性。

\begin{figure}[!htbp]
  \centering
    \includegraphics[width=0.93\columnwidth]{figures/performance.png}
  \bicaption{性能对比。我们将我们的方法 (绿色) 与Li等~(\citeyear{Li-2020})、Chen等~(\citeyear{Chen-2021}) 中的动理学求解方法 (红色) 进行比较,比较用例为图~\ref{img:comparsion_car_ib_ours} 中所示的汽车仿真,仿真时长为1秒。在同一张NVIDIA RTX 3090 GPU上得到的测试结果显示我们的方法在所有分辨率上都有显著的性能提升。}{Performance comparison. We compare the efficiency of the kinetic solver from~\cite{Li-2020,Chen-2021} (red) with our kinetic solver (green) based on one second of simulation of the car model shown in Fig.~\ref{img:comparsion_car_ib_ours} and both executed on a same NVIDIA RTX 3090 GPU, showing significant performance improvement for all grid resolutions.}
  \label{img:Performance}
\end{figure}

\paragraph{性能对比}
在Li等~(\citeyear{Li-2020}) 与Chen等~(\citeyear{Chen-2021}) 中,作者已经通过对比,说明了动理学方法的计算效率是显著优于N-S方法的。这种优势在湍流流固耦合仿真中更加明显,因为需要的时间步长更小。由于我们的方法与这些工作的框架相同,结合上述的各项优化,使得我们的方法在性能上更有优势。我们在同一张NVIDIA RTX 3090 GPU上,进行了多个分辨率的对比,结果显示我们的方法在所有的分辨率上都有显著提升 (见图~\ref{img:Performance})。通过代码性能分析,我们分析出两个使性能得以提升的主要因素:第一个是基于LU分解的碰撞运算符简化~\cite{fei2018three} 显著提升了GPU占用率,第二个是使用了几何近似优化之后,我们的混合边界处理比起之前工作使用的DI-IBM,计算量要更小。这两点使得我们的GPU计算效率得以提升。

\subsection{仿真结果}
下面我们展示一系列的仿真结果。这些仿真结果都于一个14核Intel Xeon E5-2690 CPU、128 GB内存、NVIDIA RTX 3090的工作站上运行得到。我们通过一系列的单向与双向流固耦合算例,对一系列的复杂几何进行测试,以显示我们的方法处理不同类型物体的能力。我们还展示了一些可交互的仿真、与离线的高分辨率大规模仿真。

\begin{figure}[!htbp]
  \centering
    \includegraphics[width=0.99\columnwidth]{figures/result_tube.png}
  \bicaption{旋转的薄圆柱面。一个薄圆柱面分别在高粘度 (顶图) 与低粘度 (底图) 流体中绕轴旋转。在圆柱面内部和外部分别有烟雾粒子对流场进行可视化 (内部为蓝色,外部为红色)。左图与右图分别展示仿真在4s与7s后的烟雾形态。}{Rotating cylindrical thin shell. A cylindrical thin shell rotating around its axis in a very high (top) and very low viscosity (bottom) fluid respectively, where smoke particles scattered inside (blue) and outside (red) the thin-shell cylinder are advected in the flow (left: after 4 seconds; right: after 7 seconds) to demonstrate that boundary layers are well resolved.}
  \label{img:result-tube}
\end{figure}

\begin{figure}[!htbp]
  \centering
    \includegraphics[width=0.99\columnwidth]{figures/result_thin_shell_sub_grid.png}
  \bicaption{烟雾经过层叠的板子。烟雾吹过一系列下落的薄板。虽然薄板在分开时烟雾可以通过,当它们层叠在一起时却形成一个密闭的障碍物。我们边界处理中的亚网格近似可以同时处理这两种不同情况,当板子加速靠近时,中间的流体会被加速,而它们紧贴时会成为一个厚的物体。}{Smoke flow through stacked plates. Smoke is blown towards a falling stack of thin plates. While smoke freely flows between separate plates, they become an airtight obstacle when they are stacked on top of each other. Our boundary treatment based on subgrid approximation deals with both cases seamlessly: plates getting closer accelerate the flow in between them, while tightly stacked plates are treated as thick solids.}
  \label{img:result-thin-shell-sub-grid}
\end{figure}

\begin{figure}[!htbp]
  \centering
    \includegraphics[width=0.99\columnwidth]{figures/result_blowing_leaves.png}
  \bicaption{风吹叶子的仿真。在这个双向流固耦合结果中,一阵风从地面吹起将树叶吹向空中。}{Wind blowing leaves. In this two-way coupling example, a puff of wind from the ground drives a bunch of dried-up leaves up in the air.}
  \label{img:result-blowing-leaves}
\end{figure}

\paragraph{与薄板的耦合}
我们首先构造一个薄圆柱面,使得厚度远小于网格大小,之后令圆柱面旋转,产生剪切流 (见图~\ref{img:result-tube})。我们在低雷诺数 (高粘度) 与高雷诺数 (低粘度) 分别进行了测试,并展示出完全不同的流场特性。我们还测试了多个薄板相叠的算例。这些薄板最初是分开的,之后逐渐叠落在一起。我们的亚网格近似使得我们的方法可以仿真从起初分离的薄板到叠在形成一个固体的整个过程,而无需任何调整。结果可见图~\ref{img:result-thin-shell-sub-grid}。我们最后展示一个双向耦合的结果,仿真一阵风从地面吹起树叶的过程。树叶一开始被风吹起后,缓慢地落回地面,并因风产生旋转。结果可见图~\ref{img:result-blowing-leaves}。

\begin{figure}[!htbp]
  \centering
    \includegraphics[width=0.99\columnwidth]{figures/result_rotating_torus.png}
  \bicaption{旋转的圆环。一个非常细的圆环沿着一个垂直轴旋转,在两侧发出黄色和橘色的烟雾,展示出圆环旋转所带出的湍流。}{Rotating ring. A very thin ring rotating along a vertical axis and emitting yellow and orange smoke particles is enough to create a turbulent wake as evidenced by the volutes of smoke resulting from the motion.}
  \label{img:result-rotating-torus}
\end{figure}

\begin{figure}[!htbp]
  \centering
    \includegraphics[width=0.99\columnwidth]{figures/result_thin_rod_sub_grid.png}
  \bicaption{发梳仿真。一个包含着上百根硬毛地发梳 (左图) 在旋转的同时从左向右平移,带动周围气流运动。与有硬毛的情况 (右图) 相比,没有硬毛时的发梳 (中图) 运动产生的尾流更加平滑,涡旋也更大。这个结果展示了我们的边界处理方法,包括亚网格近似,处理复杂几何的能力。}{Hair brush. The translating and rotating motion of a hair brush containing hundreds of bristles (left) creates fine vortices in its wake as well as around the bristles (right) as evidenced by the evolution of the smoke passing around it, properly capturing the intricate fluid-solid interaction engendered by this complex geometry. Compared to the coupling with bristles (right), the smoke near the hair brush without bristles (middle) is much smoother, and the wake flows contain relatively large vortices, indicating the efficacy of subgrid approximation in handling such complex solid shapes.}
  \label{img:result-thin-rod-sub-grid}
\end{figure}

\paragraph{与细棒的耦合}
下面继续展示我们的求解器对细棒耦合的仿真结果。我们先展示一个环状物体在空气中定速旋转的结果,其直径远小于网格大小,见图~\ref{img:result-rotating-torus}。我们还对多个细棒组合在一起时的情况进行仿真,结果见图~\ref{img:result-thin-rod-sub-grid}。注意当固体表面有硬毛时,尾流的湍流程度有显著的差别。

\begin{figure}[!htbp]
  \centering
    \includegraphics[width=0.99\columnwidth]{figures/result_wind_turbine.png}
  \bicaption{风力涡轮机的仿真。常速的风吹过一个有三个扇叶的风力涡轮机,驱动涡轮机旋转,并由扇叶上产生的烟雾对涡旋尾流结构进行可视化。这个双向的流固耦合结果考虑了轴上的摩擦力来限制了最大的角速度。扇叶角速度随时间的变化可见底部的小图。}{Turbine. A large turbine emitting colored smoke particles at the three propellers of its large blade is simulated with a constant incoming air flow making it turn, creating a spiral vortex trail. This two-way coupling example also involves friction on the axis of the turbine to limit its maximum angular velocity, resulting in a time-varying but converging curve of angular velocity of the turbine shown in the bottom inset.}
  \label{img:result-wind-turbine}
\end{figure}

\begin{figure}[!htbp]
  \centering
    \includegraphics[width=0.99\columnwidth]{figures/teaser_concorde.jpg}
  \bicaption{飞机气动仿真。通过对协和客机翼面的气流仿真,我们展示出我们的方法可以同时处理含有薄板、细棒的薄物体以及厚物体的复杂几何。}{Aerodynamic simulation of an airplane. By simulting the flow over the wing of the Concorde airplane, we demonstrate that our method can handle complex solids containing thin shells, rods and thick objects.}
  \label{img:teaser_concorde}
\end{figure}

\paragraph{与复杂物体耦合}
一个更复杂但非常常见的情况是厚物体与薄物体同时存在,并共同构成一个复杂物体。我们的边界处理可以有效并统一地处理这种情况。图~\ref{img:result-wind-turbine} 展示了通过风带动一个三扇叶风力涡轮机模型旋转的仿真结果。这里风在带动扇叶旋转的同时,扇叶也影响空气流动,展现了我们方法处理复杂物体的双向耦合。图~\ref{img:teaser_concorde} 展示了协和式客机以攻角20度姿态在空中飞行的仿真结果。对于飞机这样的复杂物体,许多部件对于网格都是亚网格大小的,尤其是机翼部分。在这两次仿真中,得到的结果都与期望相符。

\begin{figure}[!htbp]
  \centering
    \includegraphics[width=0.99\columnwidth]{figures/result_design.png}
  \bicaption{快速交互设计。我们通过GUI操作模拟系统,来快速生成仿真结果。图中示例为一个快速旋转的旋翼,我们可以通过可视化即时的看到尾流情况。这样的快速交互式仿真可以用来辅助产品设计与验证。}{Fast simulation for interactive design. We operated our GUI-based simulation system to produce a quick preliminary result of a fast rotating rotor-blade where turbulent wake flows can be observed interactively, which is very useful for efficient product design and verification.}
  \label{img:result-design}
\end{figure}

\paragraph{快速仿真以进行交互式设计}
我们的流固耦合算法可足够高效,以进行交互式产品设计与测试。作为示例,我们在$200\!\times\!200\!\times\!200$分辨率下对一个旋翼旋转的情况进行仿真。图~\ref{img:result-design} 展示了仿真中的速度场截面。完成1/100 s的仿真所需的计算时间为0.4 s。这个时间包括数据的读取与可视化 (截面可视化在CPU上完成)。这意味着类似旋翼这类的产品可以使用我们的算法进行快速的气动性能评估,并可基于此优化设计。

\begin{figure}[!htbp]
  \centering
    \includegraphics[width=0.99\columnwidth]{figures/result_city.png}
  \bicaption{气流经过城市街区的高分辨率仿真。我们对气流经过城市街区进行高分辨率仿真 (($889\times333\times556$)),建筑中包含诸多细且尖锐的结构。一层相对薄的烟雾从左侧注入,在建筑间产生涡流细节。}{High-resolution simulation of airflow through a city neighborhood. We simulate the airflow passing around and over buildings using a high resolution grid ($889\times333\times556$), where the buildings contain both thin and sharp structures. A layer of smoke particles with relatively small thickness is coming from the left, creating fine vortical structures behind and in between the buildings.}
  \label{img:result-city}
\end{figure}

\paragraph{高分辨率仿真}
最后,我们展示一个高分辨率的大规模仿真结果,来展示我们方法的可扩展性。该仿真使用两个GPU同时计算,展示了大量烟雾穿过一个高密度城市街区的场景 (见图~\ref{img:result-city})。场景中包含诸多细且尖锐的物体对尾流产生影响。对于这样的大场景湍流仿真,5 s的仿真所需的计算时间为1.5小时左右。正如Li等~(\citeyear{Li-2020}) 所讨论的,达到这样的仿真效率对于N-S方法来说是十分困难的,即使同样使用多GPU进行加速。主要原是N-S方法中压力场求解这一步需要求解全局的线性方程组,在高分辨率下会严重影响计算效率。

\section{方法的局限性}
虽然我们的方法展示出了优秀的通用性与数值结果,我们的方法依旧有一定的局限性。
首先,由于我们使用了基于采样点的几何计算近似,当固体包含尖角的时候,是否需要应用反弹边界可能发生误判。增加采样点数量或使用自适应采样可以提升精度,但是也会同时提升显存需求。第二,对固体所受的力和力矩的计算也受制于对固体外形的近似。如我们的亚网格近似无法真实反应物体的几何形状,会影响固体受力计算的精度。最后,对于可变形的物体,我们的方法需求一个更高效的采样方法,以每帧更新物体表面的采样点。

\section{总结}
在本章中,我们提出了一个基于动理学方法的混合边界处理方法,以处理在流固双向耦合。我们的方法可以有效、统一地同时处理厚、薄物体,并可以稳定、快速地完成仿真,得到视觉上可信的结果。我们的方法充分利用了反弹边界与尖锐界面浸没边界法优点的互补性,包含双面反弹边界处理与单边的速度修正,与双向耦合所需的固体受力计算。
我们基于边界采样点提出了高效的几何近似算法与对应的GPU并行实现,以提升整体计算效率。我们的方法在效率与精度上都超越了现有的计算机图形学中的LBM算法,并在强湍流中依然可以捕捉到正确的物理现象。
我们展示了我们的方法与其它方法和实际实验的对比,和一系列的仿真结果,包括不同的几何形状在流体中的单向或双向耦合仿真,及我们的方法在快速交互式设计中的应用。
\chapter{自动化的流体仿真框架}
\label{sec:sig23}

% Sec 4.1
\section{背景与动机}
我们已经描述了应用于CG上的LBM方法,但是对于工业应用来说,以上的方法完全无法满足。在工业领域中,更多注重XXXXXX,风洞测试用的更多。在这一章中我们将介绍一个基于LBM的虚拟风洞测试系统,以克服这一问题。第~\ref{sec:siga21} 章所描述的混合方法虽然高效,但不适合虚拟风洞的应用,因为此时边界层已经非常薄,远小于网格大小,从而在网格层面假设速度是线性的已经不够精确。

\paragraph{风洞与虚拟风洞}
虽然汽车行业在早期,并不注重空气动力学的影响,许多汽车的造型都以方正和硬朗为特点,但随着汽车越来越注重经济性,尤其是现在进入电动汽车的时代后,空气动力学对汽车造型的设计影响越来越显著。从20世纪早期开始,航空工程师开始使用风洞来测试飞机,之后汽车的设计制造也开始使用类似的技术。随后更多的领域开始使用风洞测试进行空气动力学特性的测试,如高层建筑物、高速列车、船舰等的设计。虽然风洞实验对产品设计有着很强的指导意义,但是真实的风洞实验需要进行实际的物理建模,并且风洞本身的造价也十分高昂。这样的高成本与操作难度,使其难以被频繁应用。在计算机得以发展后,虚拟风洞随即出现,并成为一种更简单、高效、节约成本的气动设计解决方案。同时虚拟风洞可以更直观地提供可视化结果,如表面的高压与低压区分布、不同位置的气流的涡流程度等。这些数据可以在设计的迭代中节约大量的时间与经济成本。直到今天,虚拟风洞的效率依然深刻影响汽车、建筑、航空航天等领域的产品研发周期~\cite{HighriseBuildings,windScience}。

\paragraph{虚拟风洞的现状}
对于最常见的亚声速弱可压情况 (马赫数小于0.3时,此时流体依然可以用不可压模型进行近似描述),目前的虚拟风洞测试需要经过一个非常耗时的前处理阶段。最耗时的部分是基于物体模型构建贴体的计算网格,这一过程需要大量的人工调整,以保证网格质量。构建好计算网格后,使用基于有限体积或有限元的CFD求解器,来求解流体。即使在CPU集群上,进行这样的流体求解也可能需要数天才可完成一次仿真。在GPU平台上,使用现有软件,如西门子StarCCM+~\cite{Siemens},进行瞬态流求解,也是类似的效率。所以更多时候,在现有的虚拟风洞实验中会进行稳态求解,以节省计算时间。

\paragraph{使用LBM进行气动分析}
近些年来,LBM在进行高效湍流仿真上,已经取得了长足的进步。LBM在核心算法上的进展已经使其有能力在大规模并行结构上高效求解流体,并达到满足工业应用的精度范围~\cite{Li-2020,Lallemand:2021}。这使得LBM开始在CFD领域成为传统方法之外的一个新的选项 (我们注意到现有的工业软件,如PowerFLOW与XFlow~\cite{Simulia}均基于LBM开发,但其具体使用的方法模型并不明确)。在CG领域,虽然有LBM方法得到了较好的视觉结果~\cite{Li-2020,Lyu:2021},但其精度依然远比不上工业应用所需的精度。

\paragraph{我们的工作}
我们尝试通过解决一些LBM中的关键问题,提升LBM的整体能力,使得LBM可以成为一个统一的流体仿真框架,以应用到视觉特效、工业设计等多个领域。通过提升碰撞模型、边界处理的精度,结合多分辨率网格与GPU优化,我们构建了一个虚拟的亚声速弱可压风洞测试系统。该系统可以拥有与现有的CFD商业软件相似、甚至更高的精度与计算效率。

% Sec 4.2
\section{方法}
我们的系统相比于现有的LBM主要包含以下四个方面的改进:
\begin{itemize}
	\item 我们对累积量碰撞模型,通过局部熵值的最大化的原则进行了改进。改进后的碰撞模型在精度上有所提升,并可在$10^8$级别的雷诺数下保持稳定; 
	\item 我们提出了一个新的单点插值反弹边界来处理静态与动态物体边界,可以更精准地求解边界层流场;
	\item 我们提出了一个多分辨率网格构建算法,以自动且灵活地构建多分辨率网格,可以在不需花费过多人工前处理的情况下,完成高分辨率流体仿真
	\item 我们还提出了一系列的GPU优化,进一步提升运算效率。
\end{itemize}

我们将在下面依次介绍这四部分内容。

\subsection{累积量碰撞模型的高阶参数优化}
在碰撞过程中,高阶松弛系数对仿真 (尤其是湍流仿真) 有着很大的影响。Li等~(\citeyear{Li-2020}) 讨论了在中心矩碰撞方法中,进行高阶参数优化的方法。作者通过回归方法,自适应地调整参数,降低了数值色散和耗散误差,提高了数值精度。于是我们希望同样将高阶松弛系数的优化引入累积量碰撞模型中。由于累积量模型的空间变换是非线性的,我们无法直接应用Li等~(\citeyear{Li-2020}) 中的回归方法,或Kramer等~(\citeyear{Kramer-2019}) 中的熵优化方法。但是,我们可以通过一定的推导,将Kramer等~(\citeyear{Kramer-2019}) 中的熵优化方法推广至累积量碰撞模型中。下面我们介绍推导过程。

首先,我们先介绍熵优化的基本思想。分布函数的平衡态$f^\text{eq}_i$可以使熵函数$H(f_i)$取得最大值 (在密度、动量保持不变的前提下)。该熵函数$H(f_i)$定义为
\begin{equation}
\label{eq:entropy_func}
H(f_i)=-\sum_i f_i \log \left(\frac{f_i}{\omega_i}\right) \;,
\end{equation}
其中$\omega_i$是网格权重。所以,可以期待的是,使$H(f_i)$最大化可以使分布函数趋于稳定。这些在Kr{\"a}mer等~(\citeyear{Kramer-2019}) 中有所讨论。
为了优化过程更易求解,我们使用一个二次凹函数$\tilde{H}$对$H$进行近似。$\tilde{H}$称为伪熵 (pseudo-entropy) 函数~\cite{Kramer-2019},表达式为
\begin{equation}
\label{eq:pseudoentropy_func}
\tilde{H}(\bm{f})=-\sum_{i}\left(\frac{f_i^2}{\omega_i}-f_i\right)=\rho-\sum_{i} \frac{f_i^2}{\omega_i}\;.
\end{equation}
该式为$H(\bm{f})$在全局平衡态$f_i^{\mathrm{eq}}(\rho=1, \mathbf{u}=0)=\omega_i$处的泰勒展开.
我们回顾,在分布函数$\bm{f}\!=\!\{f_i\}_i$与中心矩$\bm{m}\!=\!\{m_i\}_i$之间存在线性变换关系:$\bm{f}\!=\!\bm{T}\bm{m}$。但与中心矩变换不同,累积量变换是非线性的,所以在累积量碰撞模型中应用熵优化依然是非常复杂的,不过我们通过仔细观察后,可以发现在中心矩$k_{\alpha\beta\gamma}$与累积量$k_{\alpha\beta\gamma}$之间有着关键的联系:当阶数$p\!\leq\!3$时,累积量与中心矩是完全相等的,而高阶累积量是对应的中心矩以及其它中心矩的和 (见第~\ref{sec:cumulant} 节)。由于0-2阶的累积量在碰撞过程中需要根据物理规律来确定松弛系数,3阶累积量的松弛系数在Geier等~(\citeyear{Geier-2017}) 已有过优化方法,我们可以基于此推导使熵最大化的高阶累积量松弛系数的\emph{解析解}。

我们用$\bm{k_l}=\{k_l, l\!\in\!\mathcal{L}\}$表示3阶及以下的累积量,如上述,这些累积量的松弛系数是已知的。
接下来我们可以将剩余的累积量,即4到6阶的,表示为$\bm{k}=\{k_h, h\!\in\!\mathcal{H}\}$。那么,伪熵的最大化问题可以被写作
\begin{equation}
    \label{eq:entropy_opt_problem}
    \bm{f}^{*} = \underset{\bm{k}}{\arg \max } \tilde{H}(\bm{f}).
\end{equation}
我们将中心矩$m_i$与对应的累积量$k_i$之间的差定义为$r_i$:
\begin{equation}
    r_i = m_i - k_i.
\end{equation}
那么$f_i$可被相应写作:
\begin{align}
    f_i &= \sum_{l \in \mathcal{L}} t_{il}m_l+\sum_{h \in \mathcal{H}} t_{ih}m_h \\
    &= \sum_{l \in \mathcal{L}} t_{il}(k_l + r_l)+\sum_{h \in \mathcal{H}} t_{ih}(k_h + r_h).
\end{align}
其中$t_{ij}$是$\bm{T}$的元素,即$\bm{T}=\{t_{ij}\}$。
由于累积量和中心矩在三阶及三阶前是相等的,所以有$r_l = 0,l \!\in\! \mathcal{L}$,使得
\begin{align}
    f_i &= \sum_{l \in \mathcal{L}} t_{il}k_l + \sum_{h \in \mathcal{H}} t_{ih}(k_h + r_h). \label{eq:fi_as_cumulants}
\end{align}
对于公式~\ref{eq:entropy_opt_problem},我们可以对下式求解来得到$k_h$:
\begin{equation}
    \label{eq:entropy_opt_derivative}
    \frac{\partial \tilde{H}(\bm{f})}{\partial k_h} = 0, h \in \mathcal{H}.
\end{equation}
公式~\ref{eq:entropy_opt_derivative} 的左手侧可以被展开写作
\begin{equation}
    \frac{\partial \tilde{H}(\bm{f})}{\partial k_h} = -\sum_i \frac{\partial f_i}{\partial k_h} \cdot \frac{2 f_i}{\omega_i}.
\end{equation}
将公式~\ref{eq:fi_as_cumulants}代入公式~\ref{eq:entropy_opt_derivative},公式~\ref{eq:entropy_opt_derivative} 可以被重新写作
\begin{equation}
    \label{eq:entropy_opt_expanded}
    \sum_i \frac{\partial f_i}{\partial k_h} \cdot \frac{\sum_{h \in \mathcal{H}} t_{ih}(k_h + r_h)}{\omega_i} = -\sum_i \frac{\partial f_i}{\partial k_h} \cdot \frac{\sum_{l \in \mathcal{L}} t_{il}(k_l + r_l)}{\omega_i},
\end{equation}
现在,如果我们定义$\bm{r} = \{r_h, h \!\in\! \mathcal{H}\}$,一个重要的发现是,存在一个可以用解析式表达的矩阵$\bm{\underline{L}}$使得
\begin{equation}
    \bm{r} = \bm{\underline{L}} \,\bm{k}_h \,+\,  \bm{\underline{n}}\;, \label{linearRelationKvsM}
\end{equation}
并且$\bm{\underline{L}}$中不含未知量。公式~\ref{linearRelationKvsM} 中的$\bm{\underline{n}}$也只含有已知的低阶累积量,所以$\bm{r}$与$\bm{k}_h$成线性关系。
我们将矩阵$\bm{T}$分解为$\bm{T} = [\bm{T}_l; \bm{T}_h]$,并定义$\bm{D} = [\frac{\partial f_i}{\partial k_h}] = \bm{T}_h(\bm{I} + \bm{\underline{L}})$。我们将单位矩阵记作$\bm{I}$,$\omega_i$组成的对角矩阵记为$\bm{W}$,公式~\ref{eq:entropy_opt_expanded} 则成为
\begin{equation}
    \bm{D}^T \bm{W}^{-1} \bm{T}_h ((\bm{I} + \bm{\underline{L}}) \bm{k}_h + \bm{\underline{n}}) = -\bm{D}^T \bm{W}^{-1} \bm{T}_l \bm{k}_l,
\end{equation}
并经过变换后,可得到$\bm{k}_h$的表达式
\begin{equation} \label{eq:solution}
	(\bm{I} + \bm{\underline{L}}) \bm{k}_h =  -(\bm{D}^T \bm{W}^{-1} \bm{T}_\text{h})^{-1}\bm{D}^T \bm{W}^{-1} \bm{T}_\text{l} \bm{k}_\text{l} - \bm{\underline{n}} \;.
\end{equation}
至此,我们得到了D3Q27网格结构下,通过熵优化进行累积量碰撞模式进行高阶参数优化的方法。该方法可以通过一个简单的线性求解得到$\bm{k}_h$,并且由于有解析解的存在,在求解的过程中并没有过高的计算代价。为了避免数值偏移,优化后的累积量要被限制于平衡态和碰撞前的数值之间。剩余的碰撞过程与第~\ref{sec:cumulant} 节中的描述一致。

\subsection{单点插值反弹边界处理}
\paragraph{静态边界处理}
\begin{figure}[htb]
  \centering
    \includegraphics[width=0.7\columnwidth]{figures/boundary.png}
  \bicaption{固体附近的边界处理。对于切削网格点 (图中标为橘色),它们的未知分布函数必须要通过边界处理决定。图中黄色点为固体内部点,青色点为流体点。对于$\bm{x}_b$点,$q$表达该点沿$\bm{c}_i$方向到边界的正则化距离,其中$\bm{c}_{\bar{i}}$为$\bm{c}_i$的反方向。}{Boundary treatment near solid object. For ``cut-cell'' boundary nodes marked in orange, their unknown distribution functions must be determined through boundary schemes instead of the regular streaming. Yellow nodes mark nodes inside the solid while cyan nodes are fluids nodes. Our boundary treatment for a node $\bm{x}_b$ uses the normalized distance $q$ to the boundary surface along a link direction of $\bm{c}_i$ with its inverse direction denoted as $\bm{c}_{\bar{i}}$.}
  \label{img:boundary}
\end{figure}

\begin{figure}[htb]
	\centering
	  \includegraphics[width=0.8\columnwidth]{figures/bnd_comp.png}
	\bicaption{边界处理的比较。对于$Re=400,000$风吹过球的场景,我们画出了使用不同边界处理进行仿真得到的球的阻力系数。仿真中的碰撞模型均为使用了熵优化的累积量碰撞模型。该场景在实际实验中得到的球的阻力系数$C_\text{d}\!=\!0.1$。虽然简单反弹边界依然在许多LBM中得到应用,但是其结果误差及波动均过大,无法得到可置信的结果。}{Comparing boundary treatments. We plot the variation over time of drag coefficient of a sphere at $Re=400,000$, for different boundary treatments but with the same entropy-optimized cumulant model. The experimental value is near $C_\text{d}\!=\!0.1$ for this drag crisis case; a simple bounce-back, still used in many LBM implementations, leads to unacceptable results, producing large prediction errors and wide force fluctuations.}
	\label{img:bnd_comp}
  \end{figure}

在虚拟风洞中,在计算域中通常会有一个物体以进行气动性能测试,所以边界处理的重要性是不言而喻的。边界处理的示意图见图~\ref{img:boundary}。其中橘色点为需要进行边界处理的流体点,黄色点为固体内部点。
我们在第~\ref{sec:boundary_treatment} 节中已经讨论了简单反弹边界和插值反弹边界的形式。其中简单反弹边界虽然构造形式非常简单,但是在边界形状复杂时,一般只有一阶精度。从而在计算固体受力时,误差及波动是非常剧烈的 (见图~\ref{img:bnd_comp} 中展示的风在高雷诺数下吹过球时,球的阻力系数$C_\text{d}$变化)。
插值反弹边界~\cite{Bouzidi-2001} 与其之后的变体~\cite{Yu-2003, Ginzburg-2003, Chun-2007} 则可以使边界求解的精度达到二阶或更高。然而,这些边界处理方法均需要相邻点的参与,使得计算不再完全局部。由于GPU并行计算中,不连续数据访问对并行效率有很大影响,这样的边界处理在GPU计算上的效率有所降低。此外,当边界点周围均为边界时,找不到相邻点会使这样的边界处理方法失效。
而最近,Tao等~\citeyear{Tao-2018-b}) 通过在边界上构造额外的分布函数,提出了一个单点的插值反弹边界方法。
具体地,如图~\ref{img:bnd_comp}中所描绘的,我们用$\bm{x}_{b}$表达固体边界附近需要边界处理的点,$\bm{c}_{i}$为指向$\bm{x}_{a}$的方向,$\bm{c}_{\bar{i}}$为相反的方向并与固体边界相交于$\bm{x}_{w}$。跟随之前的方法中的假设,我们认为这之中存在线性关系:
\begin{equation}
f_i(\bm{x}_b, t\!+\!1) = \frac{1}{1+q}f_{i}(\bm{x}_w, t\!+\!1)+ \frac{q}{1+q}f_{i}(\bm{x}_a, t\!+\!1) \;,
\end{equation}
其中$q=\|\bm{x}_b - \bm{x}_w\|/\|\bm{c}_i\|$是边界点到边界的正则化距离。
然而分布函数$f_{i}(\bm{x}_a, t+1)$可以从$\bm{x}_b$的前一时刻迁移过来,所以这里的线性插值实际上只涉及一个点的数据。
此外,边界上的未知分布函数$f_{i}(\bm{x}_w, t\!+\!1)$可以由平衡态$f_{i}^\text{eq}(\bm{u}_w(t), \rho_b(t))$ ($\bm{u}_w(t)$为边界点的速度) 与非平衡态$f_{i}^\text{neq}(\bm{x}_b, t)$的和构造。$f_{i}^\text{neq}(\bm{x}_b, t)$可以表达为
\begin{equation}
f_{i}^\text{neq}(\bm{x}_b, t) = f_{i}(\bm{x}_b, t) - f_{i}^\text{eq}(\bm{u}_b, t),
\end{equation}
即将$\bm{x}_b$的非平衡态反转。
然而由于这样的构造使用了低阶的泰勒展开近似,在高雷诺数仿真中精度并不足够。为了进一步提升精度,我们提出一种新的构造平衡态和非平衡态的方法。
首先,我们利用非平衡态的阶数比平衡态高一阶的情况~\cite{Chun-2007},意味着我们可以对$\bm{x}_b$点的非平衡态进行简单反弹,获得的解依然是二阶精度。用公式表达为
\begin{equation}
\label{eq:neq_bb}
f^\text{neq}_{i}(\bm{x}_b, t\!+\!1) = f^\text{neq}_{\bar{i}}(\bm{x}_b, t).
\end{equation}
对于平衡态部分,我们可以采用与Tao等~\citeyear({Tao-2018-b}) 同样的二阶精度单点插值方法
\begin{align}
f^\text{eq}_i(\bm{x}_b, t\!+\!1) &= \frac{1}{1+q}f^\text{eq}_{i}(\bm{x}_w, t\!+\!1) + \frac{q}{1+q}f^\text{eq}_{i}(\bm{x}_a, t\!+\!1).
\end{align}
虽然$f^\text{eq}_{i}(\bm{x}_w, t\!+\!1)$可以由$f_{i}^\text{eq}(\bm{u}_w(t), \rho_b(t))$近似,但我们仍然需要确定如何求得$f^\text{eq}_{i}(\bm{x}_a, t\!+\!1)$。
由于我们无法获得$t+1$时刻的宏观量,一种方法是通过宏观量$\bm{u}(\bm{x}_a, t)$与$\rho(\bm{x}_a, t)$来重建平衡态。这样的做法相当于将平衡态公式对$t$进行泰勒展开后,进行了0阶逼近。这样的逼近势必在高雷诺数仿真中对边界层附近有负面的影响。
所以,我们利用$f^\text{eq}_{i}(\bm{x}_a, t\!+\!1) \approx f_{i}(\bm{x}_a, t\!+\!1)$来进行逼近,以不进行任何截断。并且$f_{i}(\bm{x}_a, t\!+\!1)$可以直接从$\bm{x}_b$迁移获得。所以我们得到
\begin{equation}
\label{eq:eq_a}
f^\text{eq}_{i}(\bm{x}_a, t\!+\!1) \approx f^{*}_{i}(\bm{x}_b, t).
\end{equation}
结合公式~\ref{eq:neq_bb}与公式~\ref{eq:eq_a},未知分布函数$f_i(\bm{x}_b, t\!+\!1)$可以被写为
\begin{align}
f_i(\bm{x}_b, t\!+\!1) &= f^\text{eq}_i(\bm{x}_b, t\!+\!1) + f^\text{neq}_{i}(\bm{x}_b, t\!+\!1) \\
&= \frac{1}{1\!+\!q}f_{i}^\text{eq}(\bm{u}_w(t), \rho_b(t)) + \frac{q}{1\!+\!q}f^{*}_{i}(\bm{x}_b, t) + f^\text{neq}_{\bar{\imath}}(\bm{x}_b, t).\nonumber
\end{align}

\paragraph{动态耦合}
虚拟风洞中也需要一部分动态的单向耦合,如汽车上轮胎的旋转。
之前的LBM通常结合格点重填与反弹边界条件~\cite{Tao-2016},或使用浸没边界法~\cite{Li-2016, Li-2020}进行动态耦合。
格点重填结合反弹边界条件的方法在高雷诺数仿真中会在边界周围产生不正常的速度,而浸没边界法虽然更容易实现,但因为只有一阶精度,则达到同样精度需要更高的分辨率进行仿真,从而消耗更多的资源。
为了保证在薄边界层上保证更高的精度,在动态耦合时,我们依赖于上述的静态边界处理方法,不同点是现在各个切削速度方向的$\bm{u}_w$都需要求解。并且我们依然使用\emph{浸没}的模式来处理边界,以避免格点重填。即所有的格点我们都认为是流体点,没有固体点。当然我们注意到,要在每个时间步都进行求交并求解边界的速度,通常需要进行层级搜索~\cite{Karras-2012},而当分辨率非常高时,在GPU上进行这样的操作非常耗时。
于是我们提出了相应的GPU实现,以减轻求交与动态耦合的计算开销。这些算法我们在第 \TODO{XXX} 章中介绍。

\paragraph{固体受力计算}
对于上述的边界处理,我们依然可以用动量交换法计算固体受力~\cite{Ladd-1994, Mei-2002},即对于边界点$\bm{x}$有
\begin{equation}
    \Delta \bm{p}(\bm{x})= \sum_{j\in L(\bm{x})} \bm{c}_{j'}\,(f_{j}(\bm{x}) + f^*_{j'}(\bm{x})),\vspace*{-1.5mm}
\end{equation}
其中$L(\bm{x})$为$\bm{x}$点向外的速度方向 (由固体向流体方向),$f_{j}(\bm{x})$为通过边界处理方法得到的分布函数。则固体受到的合力为所有边界点上受力的和
\begin{equation}
    \bm{F} = \sum_{\bm{x}} \Delta \bm{p}(\bm{x}).\vspace*{-1mm}
\end{equation}
注意因为我们使用了浸没的思想,所以所有的切削网格点都应被视为边界点。

\subsection{多分辨率仿真}
\label{sec:multi-res}
如果我们在单分辨率网格上应用上述的算法,对于CG领域的应用应该足够。因为一般CG中的物体模型较为简单,且我们只需捕捉视觉现象。但应用于工程领域,虚拟风洞需要被应用至非常大的场景 (一般对于4m左右长度的汽车模型需要40m$\times$20m$\times$20m的计算域),且几何模型的精度需要有3mm或更高的解析度,以捕捉到足够小的流体特征。这些特征也许对湍流本身的视觉效果来说没有明显改变,但是它们对物理量的大小有关键的影响。使用单分辨率网格进行这样高分辨率的仿真无论从仿真时间或存储需求考虑,都是不切实际的。所以显然多分辨率网格在工业领域流体仿真应用中是不可或缺的~\cite{Hou-2019,Aultman-2022,Romani-2022}。

然而,我们已经讨论过,许多传统的基于有限体积法的N-S求解器均需要非结构化网格。对于工业精度的几何模型,构建这样的网格是十分耗时的。虽然针对非结构化网格,有自动化的网格构建方法,但是由于模型的高复杂性,网格中产生空洞或尖角的情况依然无法避免,这些会导致FVM的求解最后无法收敛。而对模型的检查、修复需要专业工程师的人工努力,这对实际工业设计中的效率有很大影响。

在LBM的文献中,多分辨率网格方法一般都基于八叉树的数据结构~\cite{EitelAmor-2013,Hasert-2014},但是要在GPU上完成高效且可扩展的实现是十分复杂的~\cite{Schornbaum-2016, Schornbaum-2018}.
Instead, we propose a fully automatic block-based multiresolution grid construction method for our LBM fluid simulator that is better adapted to efficient GPU implementation and can handle flows through extremely complex structures as described next.

% \paragraph{Multiresolution grid construction}
% Grid refinement is often guided by different criteria, among which incoming flow direction and distance to a model~\cite{Sandoval-2012,Li-2019} are arguably the most common choices.
% We thus designed our automatic multiresolution grid construction algorithm around these two factors.
% Given an axis-aligned bounding box (AABB) of the model, we first extend it by a certain distance $d^*$ representing the boundary layer thickness determined by the Reynolds number, in order to form a larger box-shaped region labeled $\Omega_n$ --- see the red box in Fig.~\ref{fig:grid_construction}.
% From the whole simulation domain $\Omega$, we then define the pure flow region $\Omega_f \!=\! \Omega \setminus \Omega_n$.
% Inside $\Omega_f$, we construct our multiresolution data structure based solely on the direction of the incoming flow by using axis-aligned grids of power-of-two resolutions which partially overlap (to ease grid transition); i.e., we go from coarse at the domain boundary to finest at $\Omega_n$ by ensuring that each uniform grid level is twice as fine as its parent while being offset from the center $\Omega_n$ along the incoming flow direction so as to cover more of the wake flow (where the turbulence must be well resolved) than in the front --- see how the regions from blue to yellow unevenly straddle the building in Fig.~\ref{fig:grid_construction}.
% As for inside $\Omega_n$, we refine the sampling once further compared to the finest grid level in $\Omega_f$, but this time we transition to a refinement based on the distance to the model: using an unsigned distance field of the model computed within $\Omega_n$, we use a once-refined uniform grid compared to the finest grid level in $\Omega_f$, and consider its nodes to be \emph{valid} only if they are within a distance $d^*$ of the model --- see the orange nodes in Fig.~\ref{fig:grid_construction}. Note that we use a mask per grid node to indicate whether or not it is a valid fluid node, and we adopt the efficient GPU-based approach of~\cite{Imre-2017} to compute the distance field. 
% The resulting multiresolution sampling is thus block-based and independent, as it does not form a tree-like hierarchy --- but it allows for a good transition between multiple levels of resolutions without the prohibitive memory storage required by the continuous-scale approach of \cite{Li-2019}, for instance.

% \paragraph{Dynamic objects}
% Because of the specific nature of a wind tunnel facility, we can restrict our handling of dynamic objects in a simulation with one-way coupling to mostly rotations, such as the wheels of a car spinning. From an axis-aligned bounding box of the rotating object, 
% we construct a grid with the finest resolution and set all its nodes as valid fluid nodes. More complex motions, such as the opening of a plane's wheel well with the landing gear coming out, can in fact be handled the same way, but one must then pick an axis-aligned bounding box that encloses the complete motion of the dynamic objects, which can be potentially wasteful in the number of finest nodes depending on the scenario at hand. We leave such very specific improvements to future work.

% \begin{figure}[t] %\vspace*{0mm}
% 	\centering \includegraphics[width=0.9\columnwidth]{images/grid_construction}\vspace*{-2.5mm} 
% 	\caption{\textbf{Multiresolution grid construction.} To keep memory size low when computing accurate predictions of physical quantities, multiresolution grids are automatically constructed, with a refinement guided by the incoming flow direction for all but the last grid level (left) --- explaining the offset of the different levels of grid compared to this architectural building so as to capture its wake accurately --- then by the distance to the object. All active fluid nodes are finally found by flooding to ensure that the flow goes through all the mesh openings larger than the finest grid resolution (right).\vspace*{-3mm} }
% 	\label{fig:grid_construction}
% \end{figure}

% \paragraph{Dealing with complex models}
% %Propagation of valid fluid nodes}
% We assumed until now that the model surface is closed, for which using a signed distance is sufficient to identify valid fluid regions.
% However, in real applications, many models are not truly closed: e.g., a car may have a grill wire as an external air inlet, or an architectural model can have small openings, still larger in size than the finest grid cell, to let the wind flow inside parts of the building.
% In this case, the above grid construction is not sufficient: we
% must run a flooding algorithm on the \emph{finest} grid level to properly tag as valid the nodes that are connected to the external flow. 
% Starting from a known fluid node ---  e.g., the node at the corner of the region $\Omega_n$ --- we check its 6 neighbor nodes and perform link-surface intersection to see whether a neighboring node is connected to the current fluid node, see Fig.~\ref{fig:propagation} in 2D as an illustration. 
% Only if a neighboring node is connected to the current fluid node do we mark it as a valid fluid node, and we propagate through the valid fluid nodes using a breath-first search order such that all fluid nodes inside $\Omega_n$ are visited.

% %Fig.~\ref{fig:grid_construction} shows the multi-resolution grid construction the right figure shows the fluid grids constructed automatically inside the geometry where the domain is connected to the exterior. 
% %m@: pointless, no? the figure does not show which one is valid, which was is not.

% \paragraph{Grid interpolation}
% Our multiresolution grid construction enforces that only \emph{two} grids that are one level apart can overlap.
% We must therefore define how to evaluate and update in time the distribution functions at a given node based on the values stored on these two levels.
% For this purpose, we strictly follow the approach described in~\cite{Lagrava-2012}. 
% The internal nodes of a coarser parent grid provide boundary values which allow the finer grid to be updated twice (since a twice-finer grid requires twice-smaller integration time steps in LBM), where the distribution functions at the boundary nodes of the finer grid are readily interpolated from the coarser grid at the current time step.
% Then, the boundary nodes of the coarser grid can be updated by interpolating from the already updated finer grid at the next time step, and so on. 
% The interpolation is done by separating the distribution functions into equilibrium and non-equilibrium parts, where the equilibrium part is computed by first interpolating macroscopic quantities before evaluating its equilibrium distribution values, while the non-equilibrium part is interpolated directly; both interpolations are done using high-order schemes~\cite{Lagrava-2012}. 
% This process is applied recursively from the coarsest grid to the finest grid in a cascaded serialized manner.
% Fig.~\ref{fig:vis_building} for instance shows an instantaneous macroscopic velocity field cross-section from the simulation of the same architecture model from Fig.~\ref{fig:grid_construction} using this interpolation approach.

% \begin{figure}[t] %\vspace*{0mm}
% 	\centering \includegraphics[width=0.55\columnwidth]{images/propagation}\vspace*{-3mm} 
% 	\caption{\textbf{Propagation of valid fluid nodes.} To find and tag all valid fluid nodes, we start from an already known fluid node, e.g., node $\bm{p}$ in 2D, then we check its immediate neighbors to see whether a link intersects a boundary. If no intersection is detected (e.g., link $\bm{p}\bm{q}$), the untagged node $\bm{q}$ is tagged as a new valid fluid node. This process repeats in a bread-first search order, enabling internal regions to be connected to the external fluid region.  \vspace*{-3mm}}
% 	\label{fig:propagation}
% \end{figure}
\chapter{应用至不同领域}
\label{chap:caa}


\chapter{仿真结果与验证}
\label{chap:results}

在本章中,我们展示我们的流体仿真方法在不同领域中的应用。其中,通过第~\ref{chap:siga21} 章中的方法得到的结果均于一个14核Intel CPU、128 GB内存、NVIDIA RTX 3090的工作站上仿真获得;通过第~\ref{chap:sig23} 章中的方法得到的结果均于一个20核Intel CPU、128 GB内存、NVIDIA A100 GPU的工作站上仿真获得。

\section{面向视觉动画的仿真结果}
下面我们首先展示一系列面向视觉动画的仿真结果。首先使用第~\ref{chap:siga21} 章描述的方法,我们对一系列的复杂几何进行单向与双向流固耦合仿真,以显示我们的方法处理不同类型物体的能力。我们还展示了一些可交互的仿真与离线的高分辨率大规模仿真,包括对一些实际物理现象的复现。

结果中的展示的烟雾均使用染色粒子对仿真产生的流场进行被动追踪生成,这些粒子最终会被转换成OpenVDB文件~\citep{Museth-2013},并使用RedShift渲染器~\citep{redshift} 进行渲染。

\begin{figure}[!htbp]
  \centering
    \includegraphics[width=0.99\columnwidth]{figures/result_tube.png}
  \bicaption[旋转的薄圆柱面]{旋转的薄圆柱面。一个薄圆柱面分别在高黏度 (顶图) 与低黏度 (底图) 流体中绕轴旋转。在圆柱面内部和外部分别有烟雾粒子对流场进行可视化 (内部为蓝色,外部为红色)。左图与右图分别展示仿真在4s与7s后的烟雾形态。}{Rotating cylindrical thin shell. A cylindrical thin shell rotating around its axis in a very high (top) and very low viscosity (bottom) fluid respectively, where smoke particles scattered inside (blue) and outside (red) the thin-shell cylinder are advected in the flow (left: after 4 seconds; right: after 7 seconds) to demonstrate that boundary layers are well resolved.}
  \label{img:result-tube}
\end{figure}

\begin{figure}[!htbp]
  \centering
    \includegraphics[width=0.99\columnwidth]{figures/result_thin_shell_sub_grid.png}
  \bicaption[烟雾经过层叠的板子]{烟雾经过层叠的板子。烟雾吹过一系列下落的薄板。虽然薄板在分开时烟雾可以通过,当它们层叠在一起时却形成一个密闭的障碍物。我们边界处理中的亚网格近似可以同时处理这两种不同情况,当板子加速靠近时,中间的流体会被加速,而它们紧贴时会成为一个厚的物体。}{Smoke flow through stacked plates. Smoke is blown towards a falling stack of thin plates. While smoke freely flows between separate plates, they become an airtight obstacle when they are stacked on top of each other. Our boundary treatment based on subgrid approximation deals with both cases seamlessly: plates getting closer accelerate the flow in between them, while tightly stacked plates are treated as thick solids.}
  \label{img:result-thin-shell-sub-grid}
\end{figure}

\begin{figure}[!htbp]
  \centering
    \includegraphics[width=0.99\columnwidth]{figures/result_blowing_leaves.png}
  \bicaption[风吹叶子的仿真]{风吹叶子的仿真。在这个双向流固耦合结果中,一阵风从地面吹起将树叶吹向空中。}{Wind blowing leaves. In this two-way coupling example, a puff of wind from the ground drives a bunch of dried-up leaves up in the air.}
  \label{img:result-blowing-leaves}
\end{figure}

\paragraph{与薄板的耦合}
我们首先构造一个薄圆柱面,使得厚度远小于网格大小,之后令圆柱面旋转,产生剪切流 (见图~\ref{img:result-tube})。我们在低雷诺数 (高黏度) 与高雷诺数 (低黏度) 分别进行了测试,并展示出完全不同的流场特性。我们还测试了多个薄板相叠的场景。这些薄板最初是分开的,之后逐渐叠落在一起。我们的亚网格近似使得我们的方法可以仿真从起初分离的薄板到叠在形成一个固体的整个过程,而无需任何调整。结果可见图~\ref{img:result-thin-shell-sub-grid}。我们最后展示一个双向耦合的结果,仿真一阵风从地面吹起树叶的过程。树叶一开始被风吹起后,缓慢地落回地面,并因风产生旋转。结果可见图~\ref{img:result-blowing-leaves}。

\begin{figure}[!htbp]
  \centering
    \includegraphics[width=0.99\columnwidth]{figures/result_rotating_torus.png}
  \bicaption[旋转的圆环]{旋转的圆环。一个非常细的圆环沿着一个垂直轴旋转,在两侧发出黄色和橘色的烟雾,展示出圆环旋转所带出的湍流。}{Rotating ring. A very thin ring rotating along a vertical axis and emitting yellow and orange smoke particles is enough to create a turbulent wake as evidenced by the volutes of smoke resulting from the motion.}
  \label{img:result-rotating-torus}
\end{figure}

\begin{figure}[!htbp]
  \centering
    \includegraphics[width=0.99\columnwidth]{figures/result_thin_rod_sub_grid.png}
  \bicaption[发梳仿真]{发梳仿真。一个包含着上百根硬毛的发梳 (左图) 在旋转的同时从左向右平移,带动周围气流运动。与有硬毛的情况 (右图) 相比,没有硬毛时的发梳 (中图) 运动产生的尾流更加平滑,涡旋也更大。这个结果展示了我们的边界处理方法,包括亚网格近似,处理复杂几何的能力。}{Hair brush. The translating and rotating motion of a hair brush containing hundreds of bristles (left) creates fine vortices in its wake as well as around the bristles (right) as evidenced by the evolution of the smoke passing around it, properly capturing the intricate fluid-solid interaction engendered by this complex geometry. Compared to the coupling with bristles (right), the smoke near the hair brush without bristles (middle) is much smoother, and the wake flows contain relatively large vortices, indicating the efficacy of subgrid approximation in handling such complex solid shapes.}
  \label{img:result-thin-rod-sub-grid}
\end{figure}

\paragraph{与细棒的耦合}
下面继续展示我们的求解器对细棒耦合的仿真结果。我们先展示一个环状物体在空气中定速旋转的结果,其直径远小于网格大小,见图~\ref{img:result-rotating-torus}。我们还对多个细棒组合在一起时的情况进行仿真,结果见图~\ref{img:result-thin-rod-sub-grid}。注意当固体表面有硬毛时,尾流的湍流程度有显著的差别。

\begin{figure}[!htbp]
  \centering
    \includegraphics[width=0.99\columnwidth]{figures/result_wind_turbine.png}
  \bicaption[风力涡轮机的仿真]{风力涡轮机的仿真。常速的风吹过一个有三个扇叶的风力涡轮机,驱动涡轮机旋转,并由扇叶上产生的烟雾对涡旋尾流结构进行可视化。这个双向的流固耦合结果考虑了轴上的摩擦力来限制了最大的角速度。扇叶角速度随时间的变化可见底部的小图。}{Turbine. A large turbine emitting colored smoke particles at the three propellers of its large blade is simulated with a constant incoming air flow making it turn, creating a spiral vortex trail. This two-way coupling example also involves friction on the axis of the turbine to limit its maximum angular velocity, resulting in a time-varying but converging curve of angular velocity of the turbine shown in the bottom inset.}
  \label{img:result-wind-turbine}
\end{figure}

\begin{figure}[!htbp]
  \centering
    \includegraphics[width=0.99\columnwidth]{figures/teaser_concorde.jpg}
  \bicaption[飞机气动仿真]{飞机气动仿真。通过对协和客机翼面的气流仿真,我们展示出我们的方法可以同时处理含有薄板、细棒的薄物体以及厚物体的复杂几何。}{Aerodynamic simulation of an airplane. By simulting the flow over the wing of the Concorde airplane, we demonstrate that our method can handle complex solids containing thin shells, rods and thick objects.}
  \label{img:teaser_concorde}
\end{figure}

\paragraph{与复杂物体耦合}
一个更复杂但非常常见的情况是厚物体与薄物体同时存在,并共同构成一个复杂物体。我们的边界处理可以有效并统一地处理这种情况。图~\ref{img:result-wind-turbine} 展示了通过风带动一个三扇叶风力涡轮机模型旋转的仿真结果。这里风在带动扇叶旋转的同时,扇叶也影响空气流动,展现了我们方法处理复杂物体的双向耦合。图~\ref{img:teaser_concorde} 展示了协和式客机以攻角20度姿态在空中飞行的仿真结果。对于飞机这样的复杂物体,许多部件对于网格都是亚网格大小的,尤其是机翼部分。在这两次仿真中,得到的结果都与期望相符。

\begin{figure}[!htbp]
  \centering
    \includegraphics[width=0.99\columnwidth]{figures/result_design.png}
  \bicaption[快速交互设计]{快速交互设计。我们通过GUI操作模拟系统,来快速生成仿真结果。图中示例为一个快速旋转的旋翼,我们可以通过可视化即时的看到尾流情况。这样的快速交互式仿真可以用来辅助产品设计与验证。}{Fast simulation for interactive design. We operated our GUI-based simulation system to produce a quick preliminary result of a fast rotating rotor-blade where turbulent wake flows can be observed interactively, which is very useful for efficient product design and verification.}
  \label{img:result-design}
\end{figure}

\begin{figure}[!htbp]
  \centering
    \includegraphics[width=0.99\columnwidth]{figures/result_city.png}
  \bicaption[气流经过城市街区的高分辨率仿真]{气流经过城市街区的高分辨率仿真。我们对气流经过城市街区进行高分辨率仿真 (($889\times333\times556$)),建筑中包含诸多细且尖锐的结构。一层相对薄的烟雾从左侧注入,在建筑间产生涡流细节。}{High-resolution simulation of airflow through a city neighborhood. We simulate the airflow passing around and over buildings using a high resolution grid ($889\times333\times556$), where the buildings contain both thin and sharp structures. A layer of smoke particles with relatively small thickness is coming from the left, creating fine vortical structures behind and in between the buildings.}
  \label{img:result-city}
\end{figure}

\begin{figure}[!htbp]
  \centering
    \includegraphics[width=0.99\columnwidth]{figures/comparison_delta_wing.png}
  \bicaption[三角翼仿真]{三角翼仿真。通过我们的图形方法得到的薄板三角翼仿真结果 (左图) 与实际实验~\citep{Delery:2001} (右图) 的对比,展示出相同的前缘涡旋结构,表面我们的图形方法在薄结构边界层上的准确性。}{Delta-wing simulation. The airflow over a thin-shell delta-wing simulated with our hybrid coupling approach (left) matches experimental visualizations from~\citep{Delery:2001} (right), exhibiting the same spiral vortex structure near the leading edge of the wing and demonstrating the accuracy of our solver in capturing boundary layer flows around thin structures.}
  \label{img:comparison_delta_wing}
\end{figure}

\begin{figure}[!htbp]
  \centering
    \includegraphics[width=0.99\columnwidth]{figures/teaser_f1.png}
  \bicaption[F1赛车的气动仿真]{F1赛车的气动仿真。这里,车身表面上显示了平均压力场 ($C_\text{p}$),车前的两个发射源中发出染色的粒子对车身及转动的轮胎后方的尾流进行可视化。该仿真使用了五层网格,最细的网格大小为4mm,F1赛车的车身为4.15m。通过我们的GPU加速实现,我们完成了2秒的仿真,在精度可用于工业仿真的同时,仿真耗费的计算时间不足1个小时。}{Aerodynamics of an F1 racing car. Here, the mean pressure field ($C_\text{p}$) is shown on the car body surface, while passively-advected dyed particles from two front sources show the wake flow behind the rotating wheels and the car body. Five-level multiresolution grids are used to capture an effective resolution of 4mm on the body of an F1 racing car measuring 4.15 meters. 
  With our optimized GPU implementation, a 2-second simulation of such a complex model takes less than one hour to compute, with an accuracy meeting current industrial standards for automotive aerodynamics. }
  \label{img:teaser_f1}
\end{figure}

\begin{figure}[!htbp]
\centering
  \includegraphics[width=0.99\columnwidth]{figures/vis_building.png}
\bicaption[建筑模型的气动仿真]{建筑模型的气动仿真。我们的虚拟风洞系统可以对包含多个通道在内的建筑结构进行仿真。从结果中我们可以清晰地看到模型内部的流场。图中通过颜色可视化的水平截面为速度场模值。}{Aerodynamics of an architectural model. Our virtual wind tunnel can simulate the airflow passing through a building structure containing covered passages inside. Visualized here is the velocity field magnitude for a horizontal cross-section, where the internal flow is clearly visible.}
\label{img:vis_building}
\end{figure}

\begin{figure}[!htbp]
\centering
  \includegraphics[width=0.99\columnwidth]{figures/vis_plane.png}
\bicaption[波音787客机的气动仿真]{波音787客机的气动仿真。我们在两个GPU上对缩比的波音787客机模型进行了高分辨率仿真,客机的攻角为$8^{\circ}$。客机表面通过颜色对表面压力场进行了可视化,并有从6个沿着机翼前缘不同位置发出的染色粒子对流体进行被动追踪。}{Aerodynamics of a Boeing-787 passenger aircraft. We conducted a high-resolution aerodynamic simulation on dual GPUs for a scaled aircraft model of Being 787 at an angle of attack of $8^{\circ}.$
The pressure field is color-mapped over the aircraft body surface, while passively-advected dyed particles injected from the six different locations along the leading edge of the main wing are visualized.}
\label{img:vis_plane}
\end{figure}

\begin{figure}[!htbp]
\centering
  \includegraphics[width=0.99\columnwidth]{figures/vis_pipe.png}
\bicaption[气流经由一个有喷嘴的管道流出]{气流经由一个有喷嘴的管道流出。我们使用我们的方法,基于多分辨率网格,可以仿真气流流过一个有很多喷嘴的管道。管道表面被渲染为半透明以看清楚其中的流场。烟雾粒子从管道的入口喷入,并逐渐充满管道后,从喷嘴中流出。注意管道周围的空气是以定速向右流动的,所以烟雾粒子从喷嘴中喷出后会继续向右流动。}{Airflow passing through a pipe with nozzles. With our multiresolution solver, we can simulate a flow passing through an irregular transparent pipe with various nozzles on its surface. By injecting smoke particles at the inlet of the pipe, it becomes clear that the flow gradually fills up the pipe while exiting from the nozzles. Note that the surrouding air was given an initial constant velocity, which blows the smoke rightward after it comes out.}
\label{img:vis_pipe}
\end{figure}

\paragraph{快速仿真以进行交互式设计}
我们的流固耦合算法可足够高效,以进行交互式产品设计与测试。作为示例,我们在$200\!\times\!200\!\times\!200$分辨率下对一个旋翼旋转的情况进行仿真。图~\ref{img:result-design} 展示了仿真中的速度场截面。完成1/100秒的仿真所需的计算时间约为0.4秒。这个时间包括数据的读取与可视化 (截面可视化在CPU上完成)。这意味着类似旋翼这类的产品可以使用我们的算法进行快速的气动性能评估,并可基于此优化设计。

\paragraph{大规模仿真}
最后,我们展示一个相对高分辨率的大规模仿真结果,来展示我们方法的可扩展性。该仿真使用两个GPU同时计算,展示了大量烟雾穿过一个高密度城市街区的场景 (见图~\ref{img:result-city})。场景中包含诸多细且尖锐的物体对尾流产生影响。对于这样的大场景湍流仿真,5秒的仿真所需的计算时间为1.5小时左右。正如~\citet{Li-2020} 所讨论的,达到这样的仿真效率对于N-S方法来说是十分困难的,即使同样使用多GPU进行加速。主要原是N-S方法中压力场求解这一步需要求解全局的线性方程组,在高分辨率下会严重影响计算效率。

\paragraph{实际物理现象的定性复现}
为了进一步验证我们方法在有亚网格物体时边界处理的精度,我们与现实世界中的实验进行定性地复现。
首先我们对三角翼的风洞实验进行复现。该三角翼的后掠角度为$75^\circ$,攻角为$20^\circ$ (见图~\ref{img:comparison_delta_wing} (a))。在实验中,三角翼上方会产生稳定的螺旋涡流结构,提升气动升力 (这种升力被称为涡升力~\citep{anderson2010aircraft})。这种结构被广泛使用于现代飞行器的设计中,如图~\ref{img:teaser_concorde} 中的协和客机。该实验的可视化可见图~\ref{img:comparison_delta_wing} (b)~\citep{Delery:2001}。我们的仿真结果在视觉上与实验结果相匹配,展示出相似的螺旋涡流结构。

虽然在第~\ref{chap:sig23} 章中,我们主要讨论了如何将该方法应用至工业计算领域,但我们也有一个重要的动机是通过提出一个高精度的流体仿真方法,弥合CG与CFD领域中计算方法的差异。为了展示该方法不止有能力应用于工业领域的仿真,也可以在视觉特效领域,仿真极为复杂的流体现象,以用于真实的视觉特效制作。我们接下来展示一系列使用第~\ref{chap:sig23} 章中的方法得到的面向视觉动画的仿真结果。

图~\ref{img:vis_plane}、\ref{img:teaser_f1} 与~\ref{img:vis_building} (包括下一节中的图~\ref{img:golf_ball_vis}、\ref{img:golf_ball_comp_single_res}、\ref{img:fastback}) 中都包含速度场截面的可视化,这里的可视化是将速度场的模值映射到了不同的颜色上。在映射时,整个网格的速度场通过格点上的速度线性插值获得。我们同时也在一些结果中展示了物体表面的压力系数$C_\text{p}$。为了获得物体表面的压力系数,我们需要将最贴近物体表面的压力场投影至物体表面。但是注意到因为切削网格造成的影响,不同切削网格点距物体表面的距离是不同的。如果我们只将距离物体表面最近的格点的压力投影至物体表面,它们的压力是不连续的。为了解决这一问题,我们从物体表面出发,沿法向方向走固定距离 (通常为一个网格大小),然后在该处通过插值获得压力值后,再投影至物体表面。这样我们可以获得更连续的物体表面压力可视化,如图~\ref{img:teaser_f1}。

根据模型的尺度、计算域的大小与最细网格大小的不同,整个仿真计算过程可能需要几个小时。其中我们的多分辨率网格构建过程一般需要不多于20分钟。

\paragraph{飞机模型仿真}
首先,我们对一个缩比的波音787飞机模型进行了高分辨率仿真 (见图~\ref{img:vis_plane})。该场景中,飞机的速度相对较低 (0.16马赫),攻角为8度。飞机的长度为6.16米,翼展为3.2米。该仿真在两个NVIDIA A100 GPU上完成,最细的网格分辨率为3.75毫米。完成1.7秒仿真消耗的计算时间约为1.8小时。

\paragraph{F1赛车仿真}
我们还对一个F1赛车模型进行了仿真,以展示我们的动态耦合 (见图~\ref{img:teaser_f1})。仿真过程中,轮胎在不停旋转,对赛车尾部的湍流产生影响。F1赛车的车身为4.15m,仿真时使用的计算域大小为$15m\!\times\!4m\!\times\!6m$,完成2秒仿真消耗的计算时间约为0.9小时。

\paragraph{建筑模型仿真}
我们之后对一个复杂的缩比建筑模型进行了气流仿真。这个模型中包含了一些连通的通道,需要我们通过传播算法来识别所有有效的流体点 (见图~\ref{img:grid_construction})。
该模型的包围盒大小为$4.34m\!\times\!1.06m\!\times\!3.76m$,仿真时使用的计算域大小为$25m\!\times\!8m\!\times\!25m$。完成3.5秒仿真消耗的计算时间约为1.9小时,速度场截面的可视化结果可见图~\ref{img:vis_building}。

\paragraph{管道流仿真}
为了进一步说明我们的方法求解复杂模型边界与拓扑的能力,我们对一个不规则的有喷嘴的管道进行气流仿真 (见图~\ref{img:vis_pipe})。其中,喷嘴将管道的内部与外界连通。
管道模型的包围盒大小为$1.3m\!\times\!2m\!\times\!0.83m$,仿真时使用的计算域大小为$7.84m\!\times\!12m\!\times\!4.96m$。完成6秒仿真消耗的计算时间约为0.9小时。
通过染色的烟雾粒子进行可视化,我们可以很清楚地看到气流通过左下的入口流入管道,并从喷嘴中流出。这展示了我们的方法可以很好地对复杂形状的几何完成自动网格构建。

\section{面向工业应用的仿真结果与对比}
我们现在讨论对本文第~\ref{chap:sig23} 章中所描述的虚拟风洞系统进行的面向工业应用的测试,以验证我们方法的精度与效率。
在仿真中,物体的放置位置根据场景需求而定。对于需要在空中的物体,一般我们放置在竖直方向的中间位置。在地面上的物体则会根据需求设置离地的距离 (如汽车轮胎有时需要陷入地面以弥补轮胎自身的弹性对仿真的影响)。在水平面上,一般物体都会被放置在距来流3分之1的位置 (沿来流方向)。在垂直于来流的方向上,一般物体被放在计算域中间以更好地捕捉尾流。
整个计算域的大小一般是模型包围盒的8到10倍,图~\ref{img:domain_setup} 展示了汽车仿真的计算域场景设置。

\label{chap:results_industry}
\begin{figure}[!htbp]
  \centering
    \includegraphics[width=0.99\columnwidth]{figures/domain_setup.png}
  \bicaption[典型的虚拟风洞测试]{典型的虚拟风洞测试。为了准确预测物理量,计算域 (线框中的) 在各个方向的大小应为模型本身的包围盒的约10倍。}{Typical virtual wind tunnel test. For accurate prediction of physical quantities, the computational domain (in wireframe) should have a size typically \~10 times the size of the model's bounding box, in each direction.}
  \label{img:domain_setup}
\end{figure}

为了验证我们的虚拟风洞系统,我们进行了三组对比实验。所有的实验都包含实际风洞的测试结果,以与我们的仿真结果进行定性与定量的对比。这三组实验分别为球的阻力危机、高尔夫球的气动特性与DrivAer标准汽车测试模型的气动特性。下面我们将依次描述这三组实验。

\begin{figure}[htb]
\centering
  \includegraphics[width=0.75\columnwidth]{figures/sphere_cd_comp.png}
\bicaption[球的阻力与雷诺数的关系]{球的阻力与雷诺数的关系。在我们的仿真中,球在雷诺数在Re=400,000 (阻力危机) 时,\citep{Tao-2018-b} 与我们的方法都展示出了一个突然的阻力下降,之后又稍有上升,与实际实验~\citep{Morrison-2013,Barati-2014} 一致。}{Drag of a sphere vs. Reynolds number. Just like real-life experiments~\citep{Morrison-2013,Barati-2014} exhibit a sudden drop in drag for a sphere at a Reynolds number around Re=400,000 (drag crisis), both \citep{Tao-2018-b} and our kinetic solver demonstrate a similar drop at roughly the same Re, followed by a partial drag recovery.}
\label{img:drag_comp}
\end{figure}

\begin{figure}[htb]
\centering
  \includegraphics[width=0.88\columnwidth]{figures/golf_ball_cd.png}
\bicaption[高尔夫球的阻力与雷诺数的关系]{高尔夫球的阻力与雷诺数的关系。图中绘制了高尔夫球的阻力系数与雷诺数之间的关系,与真实实验~\citep{Bearman-1976,Aoki-2010} 相比,我们的仿真给出了十分相近的结果。注意高尔夫球阻力突然下降 (阻力危机) 的位置与先前图~\ref{img:drag_comp} 所给出的光滑球相比,雷诺数要小得多。}{Drag of golf ball vs. Reynolds number.Compared to the real-world experiments reported in~\citep{Bearman-1976} and~\citep{Aoki-2010} for the drag coefficient of golf ball as a function of the Reynolds number Re, our wind tunnel simulator provides very similar evaluations. Note that the drop in drag coefficient (drag crisis)  happens far earlier than in the case of a smooth sphere, as expected --- see Fig.~\ref{img:drag_comp}.}
\label{img:golf_ball_cd}
\end{figure}

\begin{figure}[htb]
\centering
  \includegraphics[width=0.99\columnwidth]{figures/golf_ball_vis.png}
\bicaption[光滑球与高尔夫球在Re=100,000时的对比]{光滑球与高尔夫球在Re=100,000时的对比。即使光滑球 (图中顶部子图) 与高尔夫球 (图中底部子图) 大小相同,只有表面的坑洞区别时,在我们的虚拟风洞测试中,也展现出完全不同的速度场与表面压力 ($C_\text{p}$) 场 (左图)。当我们使用染色粒子对流场进行被动追踪时,我们可以更清楚直观地看到这两者尾流所产生的区别 (右图),以理解为什么高尔夫球相对光滑球可以飞得更远。}{Ping-pong vs. golf ball at Re=100,000. While a ping-pong ball (top) differs (up to scale) from a golf ball (bottom) only in the absence of tiny dimples on its surface, testing these two balls in our wind tunnel exhibits very different velocity and surface pressure ($C_\text{p}$) fields (left); consequently, the flows visualized via passively-advected dyed particles are dramatically different (right), providing a good intuition of why golf balls can travel much further.}
\label{img:golf_ball_vis}
\end{figure}

\begin{figure}[htb]
\centering
  \includegraphics[width=0.9\columnwidth]{figures/golf_ball_single_res.png}
\bicaption[多分辨率网格与单分辨率网格仿真的对比]{多分辨率网格与单分辨率网格仿真的对比。多分辨率网格可以对边界层有更准确的捕捉,在这个例子中可以将高尔夫球表面的小坑洞中的流体准确解算 (左图)。与总格点数相同的单分辨率网格 (右图) 相比,我们可以看到完全不同模式的湍流尾流。图中通过颜色可视化的为速度的模值,单位为$m/s$。}{Multi- vs. single-resolution simulation. A multiresolution simulation better resolves the boundary layer flow going within the small dimples of a golf ball (left), while a single-resolution simulation with the same total number of grid nodes cannot (right). As a consequence, we witness a very different behavior of the turbulent wake when the velocity magnitude (with a colormap indicating its value in $m/s$) is visualized.}
\label{img:golf_ball_comp_single_res}
\end{figure}

\subsection{球的阻力危机}
阻力危机是一个很特殊的空气动力学现象,我们用风吹过球的场景对这一现象进行简单描述。可以想象,球在风中是受到了一定的阻力的,通过实验人们发现这个阻力的大小与雷诺数有关。随着雷诺数的上升,起初阻力没有大的变化。但在到达某一个特定的雷诺数后,球受到的阻力会快速且剧烈地下降。这个雷诺数我们可以称为临界雷诺数。这一现象被成为阻力危机现象。想要在数值仿真中预测这一现象是十分具有挑战性的,尤其在没有湍流模型时,因为这一现象要求求解器对边界层必须求解得足够精准。
需要说明的是,球的阻力危机发生在很高的雷诺数条件下,并且受到实际实验条件限制,精准地获取阻力系数也是十分困难的。所以在我们参考的实际实验数据中~\citep{Morrison-2013, Barati-2014},不同的实验数据也有一定的差异。而图~\ref{img:sphere_wake_comp} (c) 中的实验可视化是通过在球上添加了一个额外的环来增加有效雷诺数,从而在低雷诺数环境中进行的等效实验。

我们的虚拟风洞系统可以定性地重现阻力危机这一现象 (见图~\ref{img:drag_comp}),即在正确的临界雷诺数,球受到的阻力产生明显且剧烈的下降。并且相比使用~\citep{Tao-2018-b} 的LBM求解器,我们的方法在临界雷诺数及之后的范围中预测的阻力系数更接近实验值。
我们注意到,我们预测出的阻力在临界区域中与实验值相比有一个小的偏移,使得在Re=200,000时我们的预测结果与实验值有一个较大的偏差。关于这一点,我们需要说明的是,即使是实际实验中,阻力系数也是在剧烈变化的。我们这里列出的实验获得的阻力系数值可以被认为是一个时间段内的平均。在这样一个流体正在发生从临界雷诺数前到临界雷诺数后转变的过程中,数值仿真得到的阻力系数也有着剧烈的抖动。在这样一个位置无论是实际实验,或数值仿真,进行准确的预测都是十分困难的。关于这一点,在~\citep{Geier-2017-b} 中有非常详尽的讨论。
我们同时注意到,我们的方法在临界雷诺数后的区域,对阻力的预测过大。实验值显示该区域的阻力系数$C_\text{d}$约为0.1,而我们的方法预测在$0.175$附近,但好于~\citep{Tao-2018-b} 中预测的$0.2$。由于这一区域是在极高的雷诺数下,通过直接数值仿真来准确求解边界层,对于现有的流体求解器,均是十分困难的~\citep{Tiwari-2020}。

\subsection{高尔夫球的气动特性}
我们还在虚拟风洞中测试了高尔夫在高速气流下的气动特性。我们使用的高尔夫球模型有362个坑洞 (美国职业高尔夫球巡回赛的赛事用球拥有最低322个,最高376个坑洞),坑洞深度与球的直径的比值为0.007。
如图~\ref{img:golf_ball_cd},我们通过仿真得到的阻力系数与已发表的实验数据~\citep{Bearman-1976, Aoki-2010}相吻合。这个结果同时可以验证我们的方法对于小尺度的几何变化的敏感性。在Re=100,000时,高尔夫球与表面光滑的球所造成的尾流截然不同。原因在于高尔夫球表面的小坑洞可以产生非常小的边界层涡流,并影响周围的流场,减小层流边界层在固体边界上的附着,这使得高尔夫球表面流场的分离点相比平滑球将更靠后,形成更加收窄的尾流形状,并降低自身阻力。这一点可见图~\ref{img:golf_ball_vis}。
注意我们的多分辨率网格是能捕捉到这样的小细节的关键。如果我们采用单分辨率网格,即使总网格数相同,也没有能力捕捉到高尔夫球表面坑洞所造成的薄边界层。这会造成仿真错误地预测边界分离点与尾流形状,并得到大幅升高的阻力系数,见图~\ref{img:golf_ball_comp_single_res} (右)。

\begin{figure}[!htbp]
  \centering
    \includegraphics[width=0.99\columnwidth]{figures/drivaer_model.png}
  \bicaption[DrivAer模型尾部和底盘的不同配置]{DrivAer模型尾部和底盘的不同配置。尾部的配置 (左图):快背式 (F)、阶背式 (E) 与方背式 (N)。底盘的配置 (右图):复杂底盘 (D) 与平滑底盘 (S)。图片来自~\citep{deltransient}。}{Rear end and underbody configurations of DrivAer. Rear end configurations (left): fastback (F), estate back (E) and notchback (N). Underbody configurations (right): detailed (D) and smooth (S). Image from~\citep{deltransient}.}
  \label{img:drivaer_model}
\end{figure}

\begin{figure}[!htbp]
  \centering
    \includegraphics[width=0.9\columnwidth]{figures/tum_validation_cd_curve.png}
  \bicaption[DrivAer汽车的阻力随时间的变化]{DrivAer汽车的阻力随时间的变化。我们画出我们仿真所给出的不同DrivAer汽车配置的阻力系数随时间的变化,包括有地面仿真与无地面仿真的情况。(GS表示有地面仿真,即汽车的轮胎旋转且地面运动的速度与轮胎的线速度一致。)}{Car drag over time. We plot the simulated drag coefficient in time for different DrivAer car configurations, with or without ground simulation (GS, meaning ground motion and rotating wheels are simulated).}
  \label{img:tum_validation_cd_curve}
\end{figure}
  
\begin{figure}[!htbp]
  \centering
    \includegraphics[width=0.9\columnwidth]{figures/tum_fastback.png}
  \bicaption[DrivAer fastback汽车的空气动力仿真]{DrivAer fastback汽车的空气动力仿真。我们对DrivAer fastback这样的标准测试车型进行气动仿真,图中展示了垂直的平均速度场截面,与模型表面的平均压力 (平均均指随时间平均)。轮胎在有地面仿真 (蓝色方框内) 与无地面仿真 (红色方框内) 时的平均表面压力可视化也在图中展示。}{DrivAer Fastback aerodynamic simulation.
  A vertical cross-section shows the magnitude of the mean velocity flow around the DrivAer fastback benchmark car model, while the mean pressure over the model surface colors the mesh.
  Mean pressure distributions without (red inset) and with (blue inset) ground simulation are also visualized on the wheels.}
  \label{img:fastback}
\end{figure}
  
\begin{table}[!htbp]
    \centering
  \bicaption[DrivAer汽车模型的阻力计算精度]{DrivAer汽车模型的阻力计算精度。我们对我们仿真得到的不同DrivAer车型配置的阻力系数$C_\text{d}$与实验值进行了比较。GS表明地面仿真,即有GS时地面运动且轮胎旋转。}{Drag estimation accuracy of DrivAer car model. We compare our estimates of the drag coefficient $C_\text{d}$ for different DrivAer configurations with experimental data, with or without ground simulation (GS, meaning ground motion and rotating wheels are simulated).}
  \begin{tabular}{*{4}{c}}
        \toprule
    车型配置 & 仿真所得$C_\text{d}$ & 实际实验所得$C_\text{d}$ & 相对误差 \\
        \midrule
    快背式 (无GS) & 0.2849 & 0.284 & 0.32\%\\
    阶背式 (无GS) & 0.2851 & 0.286 & -0.31\%\\
    方背式 (无GS) & 0.316 & 0.318 & -0.63\%\\
    快背式 (有GS) & 0.2811 & 0.275 & 2.22\%\\				
    阶背式 (有GS) & 0.283 & 0.277 & 2.17\%\\				
    方背式 (有GS) & 0.3089 & 0.319  & -3.17\%\\
        \bottomrule
  \end{tabular}
  \label{tab:drivaer_result}
\end{table}

\subsection{DrivAer汽车模型的气动特性}
最后,我们对DrivAer汽车模型进行气动仿真。DrivAer汽车模型是慕尼黑工业大学 (Technische Universit\"at M\"unchen) 所制造的一个标准汽车测试模型,其主要目的是为了研究汽车外形对空气动力学特征的影响,同时验证数值仿真算法的能力~\citep{Heft-2011, Heft-2012}。这一模型在汽车工业受到广泛认可。

我们在测试中使用了三种不同的车型配置,三种配置的主要区别是汽车尾部的造型区别,分别被称为快背式 (fastback)、阶背式 (notchback) 与方背式 (estateback),三种配置的几何示意可见图~\ref{img:drivaer_model}。我们在测试中使用的模型的几何精度为3 $mm$,带有发动机盖、轮胎、轮腔、后视镜与复杂底盘,且不考虑发动机舱内的流动。
所有的测试中,气流的流速 (即车速) 都被设置为57.6 $km/h$。并且,对每一个车型配置,我们都测试了有和无地面仿真 (ground simulation, GS) 这两种情况。在有地面仿真的情况中,地面的速度是与车速等同的,且轮胎也在以同样的线速度旋转。而没有地面仿真的情况中,地面和轮胎都维持静止。
这两种情况可以很直观地显示出,轮胎与地面的速度差别,对于风洞实验与气动结果的影响是巨大的。

我们将仿真得到的阻力数据画在了图~\ref{img:tum_validation_cd_curve} 中,这里阻力是随时间变化的。从图中可以看出,在流场经历过初始化带来的不稳定阶段后,阻力在约1 $s$后趋于稳定。我们将1.2 $s$至2 $s$区间内的阻力系数平均来得到相应车型配置的阻力系数,并于实际风洞实验值~\citep{Heft-2012b} 相对比,得到的结果可见表~\ref{tab:drivaer_result}。
注意我们的仿真结果是经过了一次平均值修正后的结果。平均值修正即我们将所有仿真结果与对应实验值误差的平均值减去,这样我们得到的可以认为是误差的变化趋势。这是工业设计中一个常见的做法。因为对于汽车这样的复杂模型,数值实验存在系统的偏差,包括我们之前讨论的地面与车身之间的流场,都会使计算得到的物理量发生整体的偏移。通过平均值修正我们可以弥补这种偏移。
图~\ref{img:fastback} 中展示了平均速度与压力场,平均的时间区间同样为$1.2s$至$2s$。图中分别展示了静止与旋转轮胎的表面压力,以显示有地面仿真和无地面仿真时的区别。
与我们的预想相一致,在没有地面仿真时,我们的方法精度更高,所得结果的误差平均约为$0.4\%$。这一结果远小于工业仿真中通常要求的$3\%$误差标准 (这一误差标准是我们与汽车制造商沟通获得的)。在有地面仿真时,我们的最大误差为$3.17\%$,也足以满足工业仿真中的需求 (在有地面仿真时,误差标准可以更宽容,一般在$5\%$)。

\paragraph{与现有工业软件的对比}
目前有许多商业的工业流体仿真软件,与我们的虚拟风洞系统所描述的功能类似。其中西门子 (Siemens) 的StarCCM+~\citep{Siemens} 与达索 (Dassault Syst{\`e}mes) 的PowerFLOW~\citep{powerflow}是两个较有代表性且被广泛使用的商业流体仿真分析软件,并分别通过基于FVM的N-S求解器与LBM求解器实现流体仿真。两者一般都运行于CPU集群环境。
因为StarCCM+在进行非稳态仿真时,计算量非常大,仿真时间可达数周,所以StarCCM+更经常被用于稳态求解。而稳态求解的问题是,如面对轮胎旋转这样的场景时,边界必须采取一定的近似。这与实际的物理情况并不相符。PowerFLOW克服了这些问题,但是只能在CPU集群上使用限制了LBM方法的并行优势。
而我们的方法既继承了动理学方法的优势 (即PowerFLOW软件的优势),同时又可以在GPU上解算,使得硬件需求大幅下降。如我们使用NVIDIA A100 GPU (拥有6,912个CUDA核) 对DrivAer的快背式模型进行2s的仿真,在没有地面仿真时只需约3个小时即可完成仿真,即每秒的仿真需要10,356 GPU核时。
\citet{James-2018} 使用PowerFLOW与StarCCM+分别对DrivAer的方背式模型进行了类似的仿真。在结果达到收敛时,PowerFLOW的仿真在96个CPU的集群上需要约80小时,StarCCM+在128个CPU的集群上需要约13小时。
假设上述CPU拥有至少8个核心的情况下,通过计算我们可以得知在核数相当时,我们的方法是更高效的。当然因为CPU平台与GPU平台直接比较性能是很困难的,我们这里只非常定性地对效率进行了描述。
此外,更引人注目的是,在~\citet{James-2018} 所报告的PowerFLOW的计算结果中,并没有显示出有地面仿真与无地面仿真时,阻力系数有明显区别。这与实验数据相违背。其它的结果与实验值~\citep{Heft-2012b} 相对比时,大约均在3\%的误差水平。
我们需提请注意的是,在该工作中,作者使用了没有后视镜的模型,而我们使用了带有后视镜的模型。此外作者没有提供任何数值修正相关的信息。
这证明了在DrivAer测试中,精度层面上,我们与现有的工业软件是大致相当的。当然我们也认为我们需要与PowerFLOW与StarCCM+等软件进行更直接的对比,以更明确地确定我们的方法与现有工业软件的优劣。这些我们留在未来的工作中。

\section{气动声学的仿真结果}
\subsection{平面波的传播仿真}
平面波是三维空间中最简单的一种波。将某一时刻一个波振动相位相同的点连起来,我们可以得到一系列的同相面 (波阵面),若我们得到的波阵面均为平面,且这些平面垂直于波的传播方向,我们称这样的波动为平面波 (见图)。我们先对平面波进行仿真,以测试我们的方法声波仿真的能力。在没有物体存在下平面波的传播相对较为简单,且已经被许多工作验证在LBM下该现象可以被准确捕捉~\cite{doi:10.2514/6.2009-3395, viggen2009lattice}。所以我们这里验证在有物体散射时的平面波,具体地来说,我们考虑一个刚性的单位球 (半径$r=1$ $m$的球) 在计算域中时的平面波传播仿真,以同时验证我们的边界处理精度。

\begin{figure}[!htbp]
  \centering
    \includegraphics[width=0.7\columnwidth]{figures/plane_wave.png}
  \bicaption[平面波示意图]{平面波示意图。箭头指向波的传播方向,不同颜色的平面标识波阵面。}{Schematics of plane wave. The arrow indicates wave propagation direction, the planes with different colors indicate wave fronts.}
  \label{img:plane_wave}
\end{figure}

考虑某个时刻的平面波沿着$+x$方向传播,可以表达为
\begin{equation}
  p_{inc}=p_0 e^{ikx},
\end{equation}
其中$x$为空间的位置,$i$为虚数单位,$k=\omega/c$为频率是$\omega$,$c=343$ $m/s$时的波数。这个入射的平面波可以被表示为一系列的球谐函数 (spherical harmonics)~\cite{rayleigh1896theory}:
\begin{equation}
  p_{\text {inc }}=p_{o} \sum_{l=0}^{N}(2 l+1) i^{l} j_{l}(k r) P_{l}(\cos \theta),
\end{equation}
其中$j_{l}$是球贝塞尔函数 (spherical Bessel function),$P_{l}$是$l$阶勒让德多项式 (Legendre polynomials)。

而因为球体的存在,该平面波会被散射。我们假设$p_{\text {scattered}}$是被球散射的声压,则$p_{\text {scattered}}$应该满足亥姆霍兹方程 (Helmholtz equation) 与索末菲辐射条件 (Sommerfeld radiation condition)。则该散射波可以被表达为
\begin{equation}
  p_{\text {scattered }}=\sum_{l=0}^{N} A_{l} h_{l}(k r) P_{l}(\cos \theta),
\end{equation}
其中$h_{l}$是$l$阶第一类球面汉克尔函数 (spherical Hankel function of the first kind),$A_{l}$是一个待求的系数。该系数是通过设定球面上散射压力的法向导数与球面上入射压力的法向导数相反,从而使球面上的总法向速度为零 (固体的边界条件) 来确定的,即
\begin{equation}
  \frac{\partial p_{\text {inc }}}{\partial r}+\frac{\partial p_{\text {scattered }}}{\partial r}=0, \mathrm{r}=\mathrm{r_0}.
\end{equation}
从而可以求得
\begin{equation}
  A_{l}=-p_{0}(2 l+1) i^{l} \frac{l j_{l-1}(k a)-(l+1) j_{l+1}(k a)}{l h_{l-1}(k a)-(l+1) h_{l+1}(k a)}.
\end{equation}
则场中任意一点的总声压为入射的声压和散射的声压的和
\begin{equation}
  p_{\text {total }}=p_{\text {inc }}+p_{\text {scattered }}.
  \label{eq:plane_wave_gt}
\end{equation}
我们在仿真中设置$p_0=1$,$k=2.74$,此时公式~\ref{eq:plane_wave_gt} 所描述的声压解析解的可视化可见图~\ref{img:plane_wave_result} (a)。我们使用第~\ref{chap:sig23} 章中的方法,在$120\times120\times120$的单分辨率网格上进行测试。在仿真时,除波射入面以外的计算域边界条件均使用周期边界条件 (periodic boundary condition)。得到的声压可视化结果可见图~\ref{img:plane_wave_result} (b)。

通过计算可得,通过LBM仿真得到的结果与解析解的平均声压误差约为3.7\%,具体的误差分布可见图~\ref{img:plane_wave_result} (c)。结果证明即使在相对较低的分辨率,LBM依然可以较为准确地进行声学现象仿真。

\begin{figure}[!htbp]
  \centering
    \includegraphics[width=0.99\columnwidth]{figures/plane_wave_result.png}
  \bicaption[平面波被刚性球散射的仿真结果]{平面波被刚性球散射的仿真结果。(a) 平面波被刚性球散射时,声场的解析解可视化。(b) 使用LBM对该场景仿真得到的声场可视化。(c) 仿真误差的可视化。}{Simulation result of plane wave scattering by a rigid sphere. (a) Analytical solution of plane wave scattering by a rigid sphere. (b) Visualization of simulated acoustic field using LBM. (c) Visualization of simulation error.}
  \label{img:plane_wave_result}
\end{figure}

\subsection{起落架模型的气动声学仿真}
了解和预测民航客机机身的噪声对于提升民航客机的舒适度非常重要。由于高涵道比发动机的使用大大降低了发动机噪音,起落架和增升裝置的噪音开始受到关注。具体地讲,在客机进近时起落架会放出,此时起落架发出的声能 (acoustic energy) 约占飞机总声能的三分之一。除了实际风洞试验,计算气动声学为更好地了解噪声产生机制提供了一条很有前景的途径。在此背景下,由空中客车 (Airbus) 提供支持的LAGOON (LAnding-Gear nOise database for CAA validatiON) 项目尝试提供一个行业标准的起落架模型,以用于计算气动声学的数值方法研究~\citep{doi:10.2514/6.2008-2816, doi:10.2514/6.2009-3277},LAGOON的几何模型见图~\ref{img:landing_gear_model}。

\begin{figure}[!htbp]
  \centering
    \includegraphics[width=0.9\columnwidth]{figures/landing_gear_model.png}
  \bicaption[不同视角下LAGOON起落架模型的示意图]{不同视角下LAGOON起落架模型的示意图。 (a) 等轴测投影; (b) $x-z$平面; (c) $y-z$平面。图片来自~\citep{doi:10.2514/6.2022-2850}。}{Schematics of the LAGOON landing gear at different views. (a) Isometric view; (b) $x-z$ plane; (c) $y-z$ plane. Image from~\citep{doi:10.2514/6.2022-2850}.}
  \label{img:landing_gear_model}
\end{figure}

本节对LAGOON模型进行初步的气动声学实验,以验证我们的方法在气动声学上依然可以满足工业标准的要求,所获得的不同方面的结果在下面列出。在实验中,我们设定的场景与CEPRA19风洞中进行实际风洞实验时的场景~\citep{doi:10.2514/6.2015-2993} 相同,即起落架模型置于0.23 Ma (78.99 $m/s$) 的气流中,起落架的直径$D=0.3 \quad m$,雷诺数$Re=1.541\times 10^6$。

我们首先展示一些流场的可视化。在图~\ref{img:landing_gear_pressure} 中我们展示起落架在$y=0$平面上的瞬时压力分布,图中清楚地显示出了在高速的流体中,起落架所造成的高压区与低压区。图~\ref{img:landing_gear_pressure} 中我们展示起落架在Q值为$1.0\times 10^6 \quad s^{-2}$时的等值面。这些Q值的等值面显示涡度模值远大于应变速率模值的位置。从图中我们可以看到流体是高度不稳定的,且有很多小的湍流涡旋从起落架上激发。

\begin{figure}[!htbp]
  \centering
    \includegraphics[width=0.99\columnwidth]{figures/landing_gear_pressure.png}
  \bicaption[LAGOON起落架在$y=0$平面上的瞬时压力分布]{LAGOON起落架在$y=0$平面上的瞬时压力分布。}{Instantaneous pressure distribution of LAGOON landing gear in the $y=0$ plane.}
  \label{img:landing_gear_pressure}
\end{figure}

\begin{figure}[!htbp]
  \centering
    \includegraphics[width=0.99\columnwidth]{figures/landing_gear_q_criterion.png}
  \bicaption[LAGOON起落架在Q值为$1.0\times 10^6 s^{-2}$时的等值面]{LAGOON起落架在Q值为$1.0\times 10^6 s^{-2}$时的等值面。等值面的颜色来自瞬时速度的模值。}{Iso-surfaces of Q-criterion at the value of $1.0\times 10^6 s^{-2}$ of LAGOON landing gear. The iso-surfaces are colored by instantaneous velocity magnitude.}
  \label{img:landing_gear_q_criterion}
\end{figure}

我们接下来展示一些与声音更直接相关的可视化与数值结果。首先在图~\ref{img:landing_gear_divergence} 中我们展示起落架在$y=0$平面上的胀量场分布,以突出声波的分布。从图中我们可以看到两处很明显的声波扰动 (acoustic perturbation) 来源:一处是起落架轮胎的下部,一处是起落架撑杆的顶部。并且我们也可以看到在起落架撑杆的后方 (相对于来流方向),有很强的湍流对流 (turbulent convection)。为了进一步定性分析,我们在起落架模型的右轮胎上取一个采样点 (见图~\ref{img:landing_gear_frequency} (a)),并测量该点的压力变化。之后使用加窗平均周期图法 (Welch法~\citep{1161901}) 对压力序列进行处理以获得频谱,见过可见图~\ref{img:landing_gear_frequency} (b)。图中蓝色线条为我们测得的结果,红色线条为CEPRA19风洞中的实际实验结果~\citep{doi:10.2514/6.2015-2993}。结果显示我们可以成功预测在约$1000Hz$与$1500Hz$出现的声调,与实验相吻合。

\begin{figure}[!htbp]
  \centering
    \includegraphics[width=0.99\columnwidth]{figures/landing_gear_divergence.png}
  \bicaption[LAGOON起落架在$y=0$平面上的胀量场分布(灰阶)]{LAGOON起落架在$y=0$平面上的胀量场分布(灰阶)。}{Dilatation field in the $y=0$ plane of LAGOON landing gear in grayscale.}
  \label{img:landing_gear_divergence}
\end{figure}

\begin{figure}[!htbp]
  \centering
    \includegraphics[width=0.99\columnwidth]{figures/landing_gear_frequency.png}
  \bicaption[LAGOON起落架轮胎采样点的功率谱密度]{LAGOON起落架轮胎采样点的功率谱密度。在右轮胎上的采样点K1 (a) 测得压力序列后进行信号分析,得到该点的功率谱密度 (b)。其中蓝色线条为我们测得的结果,红色线条为CEPRA19风洞中的实际实验结果~\citep{doi:10.2514/6.2015-2993}。结果显示我们可以成功预测在约1000 Hz与1500 Hz出现的声调,与实验相吻合。}{Power spectral density of a sample point on the LAGOON landing gear tire. We mearsure the pressure sequence from a sample point K1 on the right tire (a) and obtain its power spectral density via signal analysis (b). Blue curve shows our mearsurement and red curve shows the result from CEPRA19 wind tunnel testing. It demonstrates that we can successfully predict the tonal emergence at about 1000 Hz and 1500 Hz, which matches the experiment.}
  \label{img:landing_gear_frequency}
\end{figure}
\chapter{总结与展望}
\label{chap:conclusion}

\section{总结}
本文在格子玻尔兹曼方法的框架下,提出了新的碰撞模型、边界处理及相关的前处理方法与GPU加速,进一步提升格子玻尔兹曼方法的精度、稳定性与易用性,实现湍流的准确、高效仿真。本文所提出的方法可以被应用于包括计算机图形学、计算流体力学及计算气动声学等不同领域,弥合不同领域仿真方法的差异,并推动LBM在这些领域中的应用与发展。

\section{展望}
我们考虑基于目前的工作,可以在以下三个方面进一步提升方法的仿真能力,并将其影响扩展到更多的研究和行业。

\paragraph{进一步的精度及性能提升}
虽然本文展示了一系列的验证案例,如物理上的阻力危机现象,但我们可以看到目前的方法与实际实验结果依然存在一定的误差。为了改善这一情况,我们认为未来可以从以下方面,在不大幅增加计算资源的前提下,对精度及性能进一步提升。首先,目前的多分辨率网格构建没有考虑到局部的单独网格细化,对尺寸小但对流场有重要影响的结构进行局部的细化可以进一步优化计算资源的分配以提升精度。此外,目前的多分辨率网格数据过渡仍会在网格边界产生不正确的流场,这对物理量的计算,尤其是气动声学这种对扰动极其敏感的应用,有着一定的影响。提升多分辨率网格过渡的连续性,对精度的提升是十分重要的。最后,针对于性能,我们认为LBM仍在实现层面有进一步的效率提升空间,具体地讲可以通过减少仿真时需求数据的吞吐量,来减少读写存储的开销,从而加速计算。在这一方向上使用更好的迁移算法~\citep{Moritz-2022} 与浮点数表示方法~\citep{PhysRevE.106.015308} 均是值得探索的方向。

\paragraph{与可变形物体耦合} 
我们目前的方法主要专注于与刚性物体的耦合,而许多场景下必须考虑物体的形变对仿真的影响。如布料与弹性物体在流体中的动态,血流仿真中血管的变形等。现有的方法中大多考虑使用浸没边界法结合拉格朗日点对边界进行近似~\citep{doi:10.1142/S0219876221500705, WU2017103},这一想法与我们在第~\ref{chap:siga21} 章中的方法是类似的。但是如何在动态物体上快速并均匀地采样以提升仿真效率,及如何将更高精度的方法如插值反弹边界应用于可变形物体,仍需要更深入的研究。

\paragraph{扩展到不同流体}
本文主要展示了气体 (烟雾) 与物体耦合时的仿真,但是我们也注意到。许多仿真场景还需求其它不同种类的流体,如研究汽车涉水的车身污染问题,或水下机器人在虚拟环境中的训练时,就涉及到液体的仿真。LBM在解两相流时,也同样有着高效、准确的优势,并且已经在视觉动画领域展现出一定的优势~\cite{Wei:2022, Wei:2023}。我们认为可以基于这些工作进一步将现有方法扩展至多相流体,并验证其在应用于工业制造领域的精度及性能表现,以进一步提升现有流体仿真方法的适用范围。

\makebiblio

\backmatter
\begin{acknowledgement}
    感谢上海科技大学,让我真正感受到了大学的魅力。

    感谢我的导师刘晓培教授,对资质驽钝的我不倦教导,才令我得以小有所成。

    感谢Mathieu Desbrun教授与郑昌熙教授,能与您二位合作并建立友谊是我莫大的荣幸。

    感谢所有教导过我、帮助过我的老师们。我在上课时总是走神,所以你们的智慧和知识我尚只得其万一,但这依然是我这一生中最宝贵的财富。

    感谢上海科技大学FLARE实验室的全体成员,尤其我的师兄,李伟博士与柏凯博士。每次与你们交流都使我受益匪浅。感谢周宣辰,为我的答辩奔波忙碌。

    感谢我的父母与挚友们,在我最艰难与迷茫时给我的支撑与陪伴。
\end{acknowledgement}

\ifgraduate
\begin{resume}
  吕超阳,男,1996年4月生,河南郑州人。

  2014年9月——2018年6月,就读于哈尔滨工业大学 (威海) 软件学院,并获得工学学士学位。

  2018年9月——2024年1月,在上海科技大学信息科学与技术学院攻读工学博士学位。  
\end{resume}

\begin{publications}
  \begin{enumerate}
    \item Lyu Chaoyang, Li Wei, Desbrun Mathieu, et al. Fast and versatile fluid-solid coupling for turbulent flow simulation[J]. ACM Transactions on Graphics, 2021, 40(6): Article 201. (中国计算机学会推荐国际学术期刊A类;该文汇报于SIGGRAPH Asia 2021,该会议为中国计算机学会推荐国际学术会议A类)
    \item Lyu Chaoyang, Bai Kai, Wu Yiheng, et al. Building a virtual weakly-compressible wind tunnel testing facility[J]. ACM Transactions on Graphics, 2023, 42(4): Article 125. (中国计算机学会推荐国际学术期刊A类;该文汇报于SIGGRAPH 2023,该会议为中国计算机学会推荐国际学术会议A类)
  \end{enumerate}
\end{publications}

\begin{publications*}
  \begin{enumerate}
    \item ACM Transactions on Graphics, 2021. (发表于SIGGRAPH Asia 2021,中国计算机学会推荐国际学术会议A类,本文作者为第一作者)
    \item ACM Transactions on Graphics, 2023. (发表于SIGGRAPH 2023,中国计算机学会推荐国际学术会议A类,本文作者为第一作者)
  \end{enumerate}
\end{publications*}

% \begin{patents}
%   无。
% \end{patents}

% \begin{patents*}
%   无。
% \end{patents*}

% \begin{projects}
%   个人参与的科研项目、获奖情况…… (仅非匿名环境显示)
% \end{projects}
\fi

\end{document}
